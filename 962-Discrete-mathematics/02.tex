\documentclass[12pt,a4paper]{article}
\usepackage{multirow}
\usepackage{rotating}
\usepackage{graphicx}
\usepackage{amsmath}
\usepackage{amsfonts}
\usepackage{amssymb}
\usepackage{graphicx}
\pagestyle{empty}
%\usepackage[bottom=0.5in,headheight=0pt,headsep=0pt]{geometry}
%\addtolength{\topmargin}{0pt}
\usepackage{xepersian}
\linespread{1.2}
\settextfont{XB Niloofar}

\begin{document}
\begin{center}
	بسمه تعالی
\end{center}
\begin{center}
	\textbf{
		پاسخ سری دوم تمرین‌ها
		-
		- درس ریاضیات گسسته - دانشگاه صنعتی شریف}
	\\
	علیرضا توفیقی محمدی - رشته علوم کامپیوتر - شماره‌دانشجویی: ۹۶۱۰۰۳۶۳
\end{center}
\section{تمرین چهارم}
\subsection{قسمت الف}
\subsubsection{صورت سوال}
تعداد دنباله‌های $2n$تایی مانند 
$x_1, \dots, x_{2n}$
 از اعداد $+1$ و $-1$  بیابید به‌طوری که:
\[
(x_1 + x_2 \geq 0), \dots, 
(x_1 + x_2 + \dots + x_{2n-1} \geq 0), 
(x_1 + x_2 + \dots + x_{2n} = 0)
\]
\subsubsection{پاسخ}
هرکدام از جواب‌های این معادله تناظری یک‌به‌یک با مسیر‌های مشبکه در جدولی $n\times n$ بالای خط $x=y$ نمی‌روند دارند.

بدین صورت که که اگر $x_i$ برابر با +۱ بود، حرکت $i$ ام مسیر مشبکه را به‌سمت راست و در غیر اینصورت به سمت بالا تعریف کنیم و بالعکس.

حال تابع یک‌به‌یک را ارائه کردیم. تنها کافی‌ست دو چیز را نشان دهیم:\\
\\
\textbf{
۱. مسیر‌هایی که جواب‌های معادله به آن‌ها متناظر شده‌اند بالای خط $y=x$ نمی‌روند.
}

برای اینکار از برهان خلف استفاده می‌کنیم، فرض کنید جوابی مانند 
$(x_1, x_2, ..., x_{2n})$
وجود دارد که به مسیری مشبکه متناظر شده به بالای خط $y=x$ رفته است. فرض کنید در حرکت $i$ام به بالای خط $y=x$ برود. در این صورت در این لحظه تعداد حرکت‌های بالا باید بیشتر از تعداد حرکت‌های رو به راست باشد. چون $+1$ به راست و $-1$ به بالا متناظر شد معادل با این است که در $x_1, ..., x_i$ تعداد $-1$ها بیشتر از $+1$ هاست که این بدین معنی است که
$\sum_{j=1}^i x_j < 0$
و با فرض مسئله در تناقض است.
\\
همچنین چون در پایان $\sum_{i=1}^{2n} x_i = 0$ شده پس تعداد $+1$ ها با $-1$ ها برابر است یعنی تعداد کل راست‌ها با بالا‌ها برابر بوده و در آخر به نقطه‌ی $(n, n)$ در مسیرمشبکه می‌رسیم.
\\
\\
\textbf{
	۲. جواب‌هایی که متناظر با مسیر‌های مشبکه شدند در شروط
 سوال صدق می‌کنند.
}

باز از برهان خلف استفاده می‌کنیم، فرض کنید $i$ ای وجود دارد که $x_1+x_2 + ... +x_i < 0$ شده باشد.  یعنی در بین $x_1, ..., x_i$ تعداد $-1$ها بیشتر از $+1$ها بود، در این صورت چون حرکت راست معادل با $+1$ و حرکت بالا معادل با $-1$ بود یعنی تعداد حرکت‌های رو به بالا بیشتر از حرکت‌های رو به راست بوده و یعنی در حرکت $i$ام به بالای خط $y=x$ می‌رویم و این با فرض مسئله مخالف است.
\\
همچنین چون در آخر به نقطه‌ی $(n, n)$ می‌رسیم یعنی $n$ بار راست و $n$ بار بالا رفتیم یعنی $n$تا $+1$ و $n$ تا $-1$ داریم پس $\sum_{i=1}^{2n} x_i = 0$ است.
\\
\\
پس تناظر یک‌به‌یک ساخته شد و تعداد جواب‌های معادله برابر با $C_n$ است.

\subsection{قسمت ب}
\subsubsection{صورت سوال}
به چند روش می‌توان اعداد ۱ تا $2n$ را در یک جدول $2\times n$ قرار داد به‌طوری که اعداد هر سطر از چپ به راست و اعداد هر ستون از بالا به پایین صعودی باشند.
\subsubsection{پاسخ}
هر جدول صورت سوال را می‌توانیم در $2n$ مرحله بسازیم، بدین گونه که در ابتدا یک جدول $2\times n$ خالی از اعداد داریم و در مرحله‌ی $i$ام یکی از دو سطر را انتخاب کرده و عدد $i$ام را به اولین خانه‌ی خالی از سمت چپ آن اضافه کنیم با این شرط که تعداد اعداد قرارگرفته از سطر دوم از سطر اول بیشتر نشود.

برای اثبات ادعا‌ی بالا یک جدول دلخواه که شرایط مسئله را داشته باشد و عدد دلخواه $i$ از ۱ تا $2n$ را در نظر بگیرید؛ اولا همه‌ی عددهای سمت چپ $i$ کمتر از $i$ اند زیرا هر سطر صعودی است. همچنین اگر $i$ در سطر دوم باشد، ستونی که عدد $i$ در آن قرار گرفته و ستون‌های چپ آن در سطر اول شامل اعداد کمتر از $i$ اند، زیرا باید ستونی که عدد $i$ در آن قرار گرفته صعودی و سطر اول صعودی باشد. پس اگر اعداد را یکی یکی اضافه کنیم، عدد $i$ در سطر خودش در چپ‌ترین جای خالی قرار می‌گیرد و چون تمام ستون‌های سمت چپ عدد $i$ و خود ستون عدد $i$ در سطر اول از اعداد کمتر از $i$ تشکیل شده‌اند، تعداد اعداد قرارگرفته در ستون اول بیشتر مساوی ستون دوم است.
\\
همچنین اگر یک روش ساختن با $2n$ مرحله که در بند نخست توضیح داده شده را در نظر بگیرید، اعداد هر سطر و هر ستون صعودی اند. پس همه‌ی جدول‌ها ساخته می‌شوند و هیچ جدول نادرستی ساخته نمی‌شود.

حال ادعا می‌کنیم هر روش ساختن معادل با مسیر مشبکه‌ای از $(0, 0)$ به $(n, n)$ است که از خط $y=x$ بالاتر نرود.

برای این منظور اگر عدد $i$ ام در سطر بالا قرار گرفت حرکت $i$ام مشبکه را به سمت راست و اگر عدد $i$ ام در سطر پایین قرار گرفت جرکت $i$ام مسیر مشبکه را به سمت بالا در نظر بگیرید و بالعکس.
\\
به سادگی می‌توان نشان داد که تابع بالا یک تابع یک‌به‌یک و پوشاست پس تناظری یک‌به‌یک بین جدول‌ها و $C_n$ یافت شد پس تعداد جدول‌ها برابر با $C_n$ است.

\section{تمرین پنجم}
\subsection{صورت سوال}
یک $n$ ضلعی منتظم را با راس‌های $A_1, A_2, \dots, A_n$ درنظر بگیرید. به چند روش می‌توان $k$ تا از راس‌های این $n$ضلعی را انتخاب کرد به‌طوری که هیچ دو راس انتخاب‌شده‌ای مجاور نباشند؟
\subsection{پاسخ}
\subsubsection{لم}
\textbf{
تعداد حالت‌های انتخاب $k$ عدد از اعداد ۱ تا $n$ که هیچ دوتایی متوالی نباشند برابر با 
$\binom{n-k+1}{k}$
است.
}

برای اثبات هر یک از این انتخاب‌ها برابر با دنباله‌هایی $n$تایی از $k$تا یک و $n-k$تا صفر است که هیچ دو یکی کنار هم نباشند. برای این انتخاب نیست ابتدا $n-k$ صفر را قرار می‌دهیم، سپس $k$ تا یک را باید در بین این ۰‌‌ها یا در ابتدا یا انتهای آن‌ها قرار دهیم و در هر جایگاه حداکثر یک عدد یک قرار دهیم (تا مجاور نشوند) که $n-k+1$ جایگاه داریم و باید $k$ تا از آن‌ها را انتخاب کنیم یعنی $\binom{n-k+1}{k}$.
\subsubsection{مسئله‌ی اصلی}
اگر $k=0$ باشد، جواب برابر با ۱ است. در ادامه فرض می‌کنیم $k \geq 1$.
\\
اگر $n=1$ باشد، برای $k =  1$ جواب ۱ و در غیر اینصورت جواب ۰ است.
\\
اگر $n=2$ باشد، برای $k=1$ جواب ۲ و در غیراینصورت ۰ است.

حال برای 
$n \geq 3, k \geq 1$
با حالت‌بندی مسئله اصلی را حل می‌کنیم.

اگر $A_1$ انتخاب شود، $A_2$ و $A_n$ حتما انتخاب نمی‌شوند و از بین $A_3, A_4, ..., A_{n-1}$ باید $k-1$ تا انتخاب شوند که هیچ کدام متوالی نباشند که چون این حالت دیگر دور تشکیل نمی‌دهند خطی است و طبق لم برابر با 
$\binom{n-3-(k-1)+1}{k-1} = \binom{n-k-1}{k-1}$
حالت داریم.

اگر $A_1$ انتخاب نشود، باید از بین $A_2, ..., A_{n}$ دقیقا $k$ تا را انتخاب کنیم که چون باز خطی می‌شود طبق لم برابر با
$\binom{(n-1)-k+1}{k} = \binom{n-k}{k}$
حالت داریم.

پس در کل
$\binom{n-k-1}{k-1} + \binom{n-k}{k}$
حالت داریم.
\end{document}