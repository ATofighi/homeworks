\documentclass[12pt,a4paper]{article}
\usepackage{multirow}
\usepackage{rotating}
\usepackage{graphicx}
\usepackage{amsmath}
\usepackage{amsfonts}
\usepackage{amssymb}
\usepackage{graphicx}
\pagestyle{empty}
%\usepackage[bottom=0.5in,headheight=0pt,headsep=0pt]{geometry}
%\addtolength{\topmargin}{0pt}
\usepackage{xepersian}
\linespread{1.2}
\settextfont{XB Niloofar}

\begin{document}
\begin{center}
	بسمه تعالی
\end{center}
\begin{center}
	\textbf{
		پاسخ سری چهارم تمرین‌ها
		- درس ریاضیات گسسته - دانشگاه صنعتی شریف}
	\\
	علیرضا توفیقی محمدی - رشته علوم کامپیوتر - شماره‌دانشجویی: ۹۶۱۰۰۳۶۳
\end{center}
\section{تمرین نهم}
\subsection{صورت سوال}
چه تعداد جایگشت از حروف 
\lr{A}
تا
\lr{P}
داریم که نتوان با حذف تعدادی از حروف آن کلمات
\lr{BAD}،
\lr{DEAF}
و
\lr{APE}
را ساخت؟
اگر کلمه‌ی 
\lr{LEADING}
نیز غیرمجاز بود چطور؟
\subsection{پاسخ}
برای حل این مسئله از اصل متمم و اصل شمول و عدم شمول استفاده می‌کنیم.
$A$ 
را مجموعه‌ی تمام جایگشت‌های ممکن و $A_i$ را مجموعه‌ی تمام جایگشت‌هایی که زیردنباله‌ای از حروف آن کلمه‌ی $i$ام از کلمات صورت سوال را بسازد در نظر می‌گیریم.

\[
\text{Answer} = |A \backslash (A_1 \cup A_2 \cup A_3)|
 = |A| - |A_1 \cup A_2 \cup A_3|
\]
\[
\rightarrow \text{Answer} = |A| - |A_1| - |A_2| - |A_3| + |A_1\cap A_2| + ... - |A_1 \cap A_2 \cap A_3|
\]
از طرفی 
$|A| = 16!$
و
$A_1 = \binom{16}{3} \times 13!, A_2 = \binom{16}{4} \times 12! , 
A_3 = \binom{16}{3} \times 13!$
همچنین چون در دو کلمه‌ای اول در اولی A قبل از D و در دومی D قبل از A آمده پس نمی‌توان جایگشتی را ساخت که هم شامل کلمه‌ی اول و هم شامل کلمه‌ی دوم باشد. به طریق مشابه همین استدلال می‌توان گفت جایگشتی که همزمان شامل کلمه‌ی دوم و سوم باشد نیز نداریم. و حال تنها کافی‌ست جایگشت‌هایی که شامل کلمه‌ی اول و سوم هستند را حساب کنیم.
در این جایگشت‌ها باید B قبل از A بیاید، D بعد از A بیاید و Pو E نیز بعد از A بیایند و P نیز قبل از E بیاید. که تعداد جایگشت‌ها برابر با
$ \binom{5}{16}\times 3 \times 11!$
است پس جواب مسئله برابر با
\[
16! - \binom{16}{3} \times 13! - \binom{16}{4} \times 12! - \binom{16}{3} \times 13! + \binom{16}{5}\times 3 \times 11!
\]
است.

حال به حل بخش دوم مسئله می‌پردازیم، 
مشابه بخش قبل جواب مسئله برابر با
\[
\text{Answer} = |A| - |A_1| - |A_2| - |A_3| - |A_4| + |A_1\cap A_2| + ... + |A_1 \cap A_2 \cap A_3 \cap A_4|
\]
در مورد کلمه‌ی چهارم؛
$A_4 = \binom{16}{7} \times 9!$
است، همچنین مشابه استدلال بخش قبل جایگشتی شامل کلمه‌ی چهارم و کلمه‌ی دوم یا سوم نداریم و تنها کافی‌ست 
$A_1 \cap A_4$
را شمارش کنیم، در این حالت نیز چون باید ترتیب هر دو حرف در این دو کلمه حفظ شود پسوند آن 
\lr{ADING}
بوده و پیشوند آن ۳ حالت (
\lr{BLE}, \lr{LBE}, \lr{LEB})
دارد، پس پاسخ مسئله برابر با 
\[
16! - \binom{16}{3} \times 13! - \binom{16}{4} \times 12! - \binom{16}{3} \times 13! - \binom{16}{7}\times 9! + \binom{16}{5}\times 3 \times 11! + \binom{16}{8}\times 3 \times 8!
\]
است.

\section{تمرین دهم}
\subsection{صورت سوال}
فرض کنید 
$n \geq 0$
عددی صحیح است، برای هر عدد حقیقی $x$ ثابت کنید:
\[
\sum_{i = 0}^n (-1)^i \binom{n}{i} (x-i)^n = n!
\]
\subsection{پاسخ}
تابع 
$f(x) = \sum_{i = 0}^n (-1)^i \binom{n}{i} (x-i)^n - n!$
را در نظر بگیرید، این تابع، یک چند جمله‌ای با درجه‌ی $n$ است، پس اگر فقط برای $x$ های بزرگتر از $n$ و صحیح ثابت کنیم که مقدار 
 $f(x) = 0$ 
 است، چون بینهایت ریشه به‌دست آوردیم، پس برای همه‌ی نقاط باید تابع برابر با صفر باشد.
 \\
 پس تنها کافی‌ست برای 
 $x \geq n$
 و صحیح ثابت کنیم 
\[f(x) = 0 \rightarrow \sum_{i=0}^n (-1)^i \binom{n}{i} (x-i)^n = n!\]
 است. برای اینکار ادعا می‌کنیم دو طرف معادله برابر با تعداد توابع از 
 $\{1, 2, ..., n\}$
 به
 $\{1, ..., x\}$
 است که مجموعه‌ی 
 $\{1, ..., n\}$
 زیرمجموعه‌ی برد آن‌ها باشد.
 
از طرفی چون در دامنه $n$ عضو داریم و برد باید شامل حداقل $n$ عضو باشد، شامل دقیقا $n$ عضو است و تعداد این توابع برابر با تعداد توابع پوشا از 
$\{1, ..., n\}$
به خودش است که برابر با 
$n!$
است.

از طرفی این تعداد را می‌توان با اصل شمول و عدم شمول محاسبه نمود؛
برای اینکار $A$ را برابر با کل توابع و 
$A_i$
را برابر با توابعی که شامل $i$ نیستند تعریف می‌کنیم بدین صورت:
\[
\text{Ans} = |A \backslash (A_1 \cup ... \cup A_n)| = |A| - |A_1 \cup ... \cup A_n|
\]
است. که بنابر اصل شمول و عدم شمول برابر است با:
\[
|A| - |A_1| - |A_2| - ... - |A_n| + .... + (-1)^n |A_1 \cap ... \cap A_n|
= |A| + \sum_{\emptyset \neq I \subseteq \{1, ..., n\}} (-1)^{|I|} \bigcap_{i\in I} A_i
\]
\[
= x^n - \sum_{i = 1}^n \binom{n}{i}(x-i)^n = \sum_{i=0}^n (-1)^i\binom{n}{i}(x-i)^n
\]
که ثابت شد و حکم کلی نیز ثابت شد.
\end{document}