\documentclass[12pt]{article}
\usepackage{HomeWorkTemplate}
\usepackage{circuitikz}
\usepackage{tikz}
\usepackage{float}
\usepackage{amsmath}
\usepackage{xepersian}
\usetikzlibrary{arrows,automata}
\usetikzlibrary{circuits.logic.US}
\settextfont{XB Niloofar}
\newcounter{problemcounter}
\newcounter{subproblemcounter}
\newcommand{\E}{\mathbb{E}}
\linespread{1.2}
\setcounter{problemcounter}{1}
\setcounter{subproblemcounter}{1}
\newcommand{\grade}[1]{\textbf{(#1 نمره)}}
\newcommand{\problem}[1]
{
\subsection*{
مسأله‌ی
\arabic{problemcounter} 
\stepcounter{problemcounter}
\setcounter{subproblemcounter}{1}
#1
}
}
\newcommand{\subproblem}{
\textbf{\harfi{subproblemcounter})}\stepcounter{subproblemcounter}
}
\newcommand{\n}{

\null

}
\begin{document}
\handout
{نظریه‌ی زبان‌ها و اتوماتا}
{}
{علیرضا توفیقی محمدی}
{سری ۴}
{شماره‌دانشجویی: 96100363}
\problem{} % 1
\subproblem{} % 1.1
با توجه به اینکه می‌دانیم قدرت ماشین تورینگ با حافظه‌ی پنهان و قابلیت توقف هد با ماشین تورینگ بدون آن‌ها یکسان است، فرض می‌کنیم ماشین تورینگمان کشی به طول ۴ دارد و همچنین هد قابلیت توقف دارد.

حال برای حل این سوال $\Gamma$ را به شکل زیر تعریف می‌کنیم:
$$
\Gamma = \{B, 0, 1, +, X\}
$$
حال مجموعه حالت‌ها را به شکل زیر تعریف می‌کنیم:
$$
Q = \{q_0, q_1, q_2, q_3, q_4, f\}, F = \{f\}
$$
که $q_0$ حالت اولیه است که باید اینقد به راست برود تا به کوچک‌ترین رقم عدد اول برسد و سپس به $q_1$ برود.

حال به ساختن تابع جرئی انتقال حالت می‌پردازیم: (پارامتر دوم ورودی و خروجی همان کش است و منظور از علامت سوال هرچیزی عضو $\Gamma^4$ است.)
$$
\delta(q_0, ?, 0) = (q_0, ?, 0, R)
$$$$
\delta(q_0, ?, 1) = (q_0, ?, 1, R)
$$$$
\delta(q_0, ?, +) = (q_1, (0,0,0,0), +, L)
$$$$
\delta(q_1, (y,w,z, 0), x) = (q_2, (y, x, z, 0), X, R), w,x,y,z \in \{0, 1\}
$$$$
\delta(q_1, (y,w,z, 1), x) = (q_4, (y, x, z, 1), X, S), w,x,y,z \in \{0, 1\}
$$$$
\delta(q_1, (y,w,z, 0), B) = (q_2, (y, 0, z, 0), X, R), w,y,z \in \{0, 1\}
$$$$
\delta(q_1, (0,w,z, 1), B) = (f, (0, 0, 0, 0), B, L), w,z \in \{0, 1\}
$$$$
\delta(q_1, (1,w,z, 1), B) = (f, (0, 0, 0, 0), 1, L), w,z \in \{0, 1\}
$$$$
\delta(q_2, ?, x) = (q_2, ?, x, R), x \in \{0, 1, +\}
$$$$
\delta(q_2, ?, B) = (q_3, ?, B, L)
$$$$
\delta(q_3, (x, y, z, 0), w) = (q_4, (x, y, w, 0), B, L), w,x,y,z \in \{0, 1\}
$$$$
\delta(q_3, (x, y, z,0), +) = (q_4, (x, y, 0,1), B, L), x,y,z \in \{0, 1\}
$$$$
\delta(q_4, ?, x) = (q_4, ?, x, L), x \in \{0,1,+\}
$$$$
\delta(q_4, ?, x) = (q_4, ?, x, L), x \in \{0,1,+\}
$$$$
\delta(q_4, (x, y, z, v), X) = (q_1, (1,0,0, v), w, L), v,x,y,z \in \{0, 1\}, w = x \oplus y \oplus z, x+y+z > 1
$$$$
\delta(q_4, (x, y, z, v), X) = (q_1, (0,0,0, v), w, L), v,x,y,z \in \{0, 1\}, w = x \oplus y \oplus z, x+y+z \leq 1
$$
در واقع ۴ حرف روی حافظه‌ی پنهان به ترتیب نیاز به دو به یک شدن، رقم مورد پردازش برای عدد اول، رقم مورد پردازش برای عدد دوم و اینکه آیا پردازش عدد دوم تمام شده یا نه است و روش کار ماشین تورینگ به این شکل است که ابتدا به کوچیک‌ترین رقم عدد اول رفته و آن‌را در خودش ذخیره کرده و جای آن X می‌گذارد. سپس کوچیکترین جای رقم دوم را پیدا کرده و در حافظه‌ی پنهان ذخیره کرده و پاک می‌کند. حال به چپ بر می‌گردد تا به جایی که $X$ گذاشته برسد و مقدار آن را حساب می‌کند و همین کار را برای رقم سمت چپ $X$ انجام می‌دهد.

\subproblem{} % 1.2
چون قدرت ماشین تورینگ دوشیاره با یک شیاره برابر است، از ماشین تورینگ دوشیاره استفاده می‌کنیم.

ماشین تورینگ را به شکل زیر می‌سازیم که با رفتن به حالت نهایی باید زبان را بپذیرد:

$$
Q = \{q_0, q_1, q_2, q_{3, 0}, q_{3, 1}, q_{4, 1}, q_5, q_6, a, r\}
$$$$
F = \{a\}
$$

$$
\forall x \in \{0, 1\}, y \in \{0, 1, B\}: \delta(q_0, (x, y)) = (q_0, (x, y), R)
$$$$
\delta(q_0, (>, B)) =  (q_1, (>, B), R)
$$$$
\forall x \in \{0, 1\}: \delta(q_1, (x, B)) = (q_1, (x, B), R)
$$$$
\delta(q_1, (B, B)) = (q_2, (B, B), L)
$$$$
\delta(q_2, (0, B)) = (q_{3,0}, (B, B), L)
$$$$
\delta(q_2, (1, B)) = (q_{3,1}, (B, B), L)
$$$$
\delta(q_2, (>, B)) = (q_5, (B, B), L)
$$$$
\forall y, x \in \{0, 1\}: \delta(q_{3,x}, (y, B)) = (q_{3,x}, (y, B), L)
$$$$
\forall x \in \{0, 1\}: \delta(q_{3,x}, (>, B)) = (q_{4,x}, (>, B), L)
$$$$
\forall z, y, x \in \{0, 1\}: \delta(q_{4,x}, (y, z)) = (q_{4,x}, (y, z), L)
$$$$
\forall x \in \{0, 1\}, y \in \{0, 1, B\}: \delta(q_{4,x}, (B, y)) = (q_{4,x}, (0, y), R)
$$$$
\forall y, x \in \{0, 1\}: \delta(q_{4,x}, (y, B)) = (q_0, (y, x), L)
$$$$
\forall y, x \in \{0, 1\}: \delta(q_{4,x}, (y, B)) = (q_0, (y, x), L)
$$$$
\forall x,y \in \{0, 1\}: \delta(q_5, (x, y)) = (q_5, (x, y), L)
$$$$
\forall x \in \{0, 1\}: \delta(q_5, (x, B)) = (q_5, (x, 0), L)
$$$$
\forall x \in \{0, 1\}: \delta(q_5, (B, x)) = (q_5, (0, x), L)
$$$$
\delta(q_5, (B, B)) = (q_6, (B, B), R)
$$$$
\delta(q_6, (0, 1)) = (r, (0, 1), R)
$$$$
\delta(q_6, (1, 0)) = (a, (0, 1), R)
$$$$
\forall x \in \{0, 1\}: \delta(q_6, (x, x)) = (q_6, (x, x), R)
$$

روش کار به این صورت است که ابتدا رشته‌ی دوم را به زیر رشته‌ی اول منتقل کرده و رشته‌ها را هم‌طول می‌کنیم، سپس به چپ‌ترین رقم رفته و یکی یکی حرف‌های زیرهم را مقایسه می‌کنیم.

\subproblem{} % 1.3

ماشین تورینگ را به شکل زیر می‌سازیم که با رفتم به حالت نهایی زبان را می‌پذیرد.

$$
Q = \{q_0, q_1, q_2, q_3, q_4, f\}
$$
$$
F = \{f\}
$$
$$
\delta(q_0, 1) = (q_4, 1, R)
$$$$
\delta(q_0, 0) = (q_1, B, R)
$$$$
\delta(q_1, 0) = (q_1, 0, R)
$$$$
\delta(q_1, 1) = (q_1, 1, R)
$$$$
\delta(q_1, B) = (q_2, B, L)
$$$$
\delta(q_2, 1) = (q_3, B, L)
$$$$
\delta(q_3, 0) = (q_3, 0, L)
$$$$
\delta(q_3, 1) = (q_3, 1, L)
$$$$
\delta(q_3, B) = (q_0, B, R)
$$$$
\delta(q_4, 1) = (q_4, 1, R)
$$$$
\delta(q_4, B) = (f, B, L)
$$

روش کار به این صورت است که هر دفعه از سمت چپ ۰ پاک کرده سپس از سمت راست یک پاک می‌کند و دوباره به چپ می‌رود و ... و در نهایت باید فقط یک باقی بماند.

\subproblem{} % 1.4

ماشین تورینگ را به شکل زیر می‌سازیم که با رفتن به حالت نهایی زبان را می‌پذیرد.

$$
Q = \{q_0, q_a, q_b, p_a, p_b, q_3, f\}
$$
$$
F = \{f\}
$$
$$
\delta(q_0, a) = (q_a, B, R)
$$$$
\delta(q_0, b) = (q_b, B, R)
$$$$
\delta(q_0, B) = (f, B, R)
$$$$
\forall x \in \{a, b\}: \delta(q_a, x) = (q_a, x, R)
$$$$
\forall x \in \{a, b\}: \delta(q_b, x) = (q_b, x, R)
$$$$
\delta(q_a, B) = (p_a, B, L)
$$$$
\delta(q_b, B) = (p_b, B, L)
$$$$
\delta(p_a, a) = (q_3, B, L)
$$$$
\delta(p_b, b) = (q_3, B, L)
$$$$
\delta(p_a, B) = (f, B, L)
$$$$
\delta(p_b, B) = (f, B, L)
$$$$
\forall x \in \{0, 1\}: \delta(q_3, x) = (q_3, x, L)
$$$$
\delta(q_3, B) = (q_0, B, R)
$$

روش کار به این صورت است که هر دفعه یک کاراکتر را از سمت چپ حذف کرده و همان کاراکتر را از سمت راست حذف می‌کند. اگر یکی از دو سمت بلنک دیدیم یعنی رشته پالیندرم است.

\subproblem{} % 1.5	

ماشین تورینگ را به شکل زیر می‌سازیم که با متوقف شدن زبان را می‌پذیرد.
$$
Q = \{q_0, q_{1, a}, q_{2, a}, q_{3, a}, q_2, f\}
$$
$$
F = \{f\}
$$
$$
\delta(q_0, X) = (q_0, X, R)
$$$$
\forall x \in \{a, b, c\}: \delta(q_0, x) = (q_{1, x}, X, L)
$$$$
\forall x \in \{a, b, c\}, y \in \{a, b, c, X\}: \delta(q_{1, x}, y) = (q_{1, x}, y, L)
$$$$
\forall x \in \{a, b, c\}: \delta(q_{1, x}, B) = (q_2, x, R)
$$$$
\forall x \in \{a, b, c\}: \delta(q_2, x) = (q_2, x, R)
$$$$
\delta(q_2, X) = (q_0, X, R)
$$$$
\delta(q_0, B) = (f, B, L)
$$$$
\delta(f, X) = (f, B, L)
$$

روش کار به این صورت است که هر دفعه یک حرف از رشته‌ی اصلی را خوانده و آن‌را به $X$ تبدیل کرده و این حرف را سمت چپ نوار که بلنک است می‌نویسد.
\problem{} % 2
\subproblem{} % 2.1
چون بررسی اول بودن عدد بازگشتی و تعداد اعداد اول حداکثر ۱۰۰ رقمی متناهی اند، پس ماشین تورینگی که همه‌ی آن‌ها را پیدا کند و روی نوار بنویسد در زمان متناهی متوقف می‌شود و در نتیجه بازگشتی است.

\subproblem{} % 2.2
بازگشتی برشمردنی است، زیرا اعداد دوقلو نمی‌دانیم که متناهی اند یا نامتناهی.

\subproblem{} %2.3
چون الگوریتمی برای فهمیدن اینکه عدد اول است یا خیر داریم. پس ماشین تورینگ آن بازگشتی است.

\subproblem{} % 2.4
چون می‌توان $L_1 = \emptyset$ در نظر گرفت، آنگاه $L_1 \cup L_2 = L_2$ و بازگشتی برشمردنی است.

\problem{} % 3

چون انسان قادر نیست در هر ثانیه بیشتر از $c$ کاراکتر بنویسد و تعداد کل‌انسان‌ها حداکثر $k$ و عمر بشر حداکثر $t$ است، پس هر رمان حداکثر $ckt$ کاراکتر دارد.

پس کافی‌است ماشین تورینگی بسازیم که تمام رشته‌های $ckt$ حرفی را بنویسد. برای اینکار کافی است ماشین تورینگی بسازیم که یک رشته می‌گیرد و آن‌را به‌اضافه‌ی یک می‌کند، همچنین یک ماشین تورینگ دیگر بسازیم که یک رشته بگیرد و آن‌را در سمت راست نوار با یک فاصله کپی کند. 

\problem{} % 4
فرض کنید ماشین تورینگ $T$ را داریم که کد یک ماشین تورینگ و یک رشته را ورودی گرفته و بررسی می‌کند که آیا این ماشین تورینگ کاراکتر $\#$ را می‌نویسد را خیر.

می‌توانیم الگوریتمی ارائه دهیم که یک ماشین تورینگ گرفته، ابتدا تمام $\#$های آن‌را به حرفی مانند $X$ تبدیل کند که در ماشین تورینگ استفاده نشده است، سپس به ازای هر $q \in Q$ و $f \in F$ و $x, y \in \Gamma$ و $D \in \{L, R\}$ در ماشین تورینگ که
$\delta(q, x) = (f, y, D)$
است، این تابع را به
$\delta(q, x) = (f, \#, D)$
تبدیل کند. در ماشین تورینگ جدید، حرف $\#$ نوشته می‌شود اگر و تنها اگر ماشین به حالت نهایی برود. حال ماشین تورینگ $T'$ را می‌سازیم که یک ماشین تورینگ را به عنوان ورودی و یک رشته بگیرد، سپس تغییرات بالا را روی آن ایجاد کند و سپس هد را به ابتدای رشته آورده و سپس به حالت اولیه‌ی $T$ رفته و عمل ماشین $T$ را روی ماشین تورینگ انجام دهد. ماشین $T$ تشخیص می‌دهد ماشین تورینگ نوشته شده روی نوارش کاراکتر $\#$ می‌نویسد یا خیر و ماشین تورینگ نوارش حرف $\#$ می‌نویسد اگر و تنهااگر ماشین تورینگ به حالت نهایی برود؛ پس $T'$ می‌تواند تشخیص دهد که ماشین تورینگ ورودی به حالت نهایی می‌رود یا خیر. پس به تناقض رسیدیم و فرض خلف باطل و حکم ثابت است.
\problem{} % 5
\subproblem{} % 5.1
غلط است. زیرا این ماشین تورینگ می‌تواند هر ماشین تورینگی را شبیه‌سازی کند که چون خودش نیز ماشین تورینگ است پس باید بتواند خودش را نیز شبیه سازی کند.

\subproblem{} % 5.2
غلط است. چون تعداد هردو شمارا است، پس تعداد هردو برابر با $|\mathbb{N}$ بود و باهم برابر است.
\subproblem{} % 5.3
خیر، زیرا زبان همه‌ی ماشین‌هایی که خودشان را می‌پذیرند بازگشتی برشمردنی است، اما متمم آن بازگشتی برشمردنی نیست.

\problem{} % 6

\problem{} % 7
اولا واضح است که ماشین‌تورینگ یک‌طرفه قدرت کمتر‌مساوی ماشین تورینگ دوطرفه را دارد زیرا هر ماشین تورینگ یک طرفه را با همان دلتا و حالت‌ها می‌توان با یک ماشین تورینگ دوطرفه شبیه سازی کرد.

حال ثابت می‌کنیم به ازای هر ماشین تورینگ دوطرفه مثل $M$ یک ماشین تورینگ یک طرفه مثل $M'$ داریم که $L(M) = L(M')$.
برای اینکار فرض کنید 
$M = (Q, \Sigma, \Gamma, q_0, F, \delta, B)$
یک ماشین تورینگ دلخواه است که 
$X \notin \Sigma, \Gamma$, $p_i \notin Q$.
و این ماشین تورینگ هیچ‌گاه $B$ نمی‌نویسد.

حال ماشین‌تورینگ یک طرفه‌ی $M'$ را به شکل زیر در نظر بگیرید:
$$
M' = (Q', \Sigma, \Gamma', p_0, \delta', B)
$$
$$
Q' = Q \cup \{p_0, p_2\} \cup \{p_{1, x} | x \in \Gamma \}\cup \{p_{3, q} | q \in Q \}
$$$$
\Gamma' = \Gamma \cup \{X\}
$$
$$
\forall q \in Q, x \in \Gamma: \delta'(q, x) = \delta(q, x)
$$
$$
\forall q \in Q : \delta'(q, X) = (p_{3, q}, B, R)
$$
$$
\forall q \in Q, x \in \Gamma : \delta'(p_{3, q}, x) = (p_{4, q, X}, x, L)
$$
$$
\forall x \in \Sigma: \delta'(p_0, x) = (p_{3, q_0}, x, R)
$$
$$
\forall q \in Q, x, y \in \Gamma'\\ \{B\}: \delta'(p_{4, q, y}, x) = (p_{4, q, x}, y, R)
$$
$$
\forall q \in Q, y \in \Gamma'\\ \{B\}: \delta'(p_{4, q, y}, B) = (p_{5, q}, y, L)
$$
$$
\forall q \in Q, x \in \Gamma: \delta'(p_{5, q}, x) = (p_{5, q}, x, L)
$$
$$
\forall q \in Q: \delta'(p_{5, q}, X) = (q, X, R)
$$

که این ماشین تورینگ به این صورت کار می‌کند که ابتدا یک $X$ در ابتدای نوار گذاشته و بقیه‌ی نوار را یک واحد به سمت راست شیفت می‌دهد، سپس هر وقت به $X$ رسید، یک بلنک بین $X$ و رشته می‌سازد و روی بلنک قرار می‌گیرد. پس هرگاه بخواهد به چپ برود به این $X$ رسیده و بلنک ساخته و اشکالی به وجود نمی‌آورد و زبان این ماشین با زبان ماشین قبلی یکسان است و مسئله حل می‌شود.

\problem{} % 8
\end{document}
