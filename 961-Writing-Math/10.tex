\documentclass[12pt,a4paper]{article}
\usepackage{multirow}
\usepackage{rotating}
\usepackage{graphicx}
\usepackage{amsmath}
\usepackage{amsfonts}
\usepackage{amssymb}
\usepackage{graphicx}
\pagestyle{empty}
\usepackage[bottom=0.5in,headheight=0pt,headsep=0pt]{geometry}
\addtolength{\topmargin}{0pt}
\usepackage{xepersian}
\linespread{1.2}
\settextfont{XB Niloofar}

\begin{document}
\begin{center}
	بسمه تعالی
\end{center}
\begin{center}
	\textbf{پاسخ تمرین تحویلی دهم - درس ریاضی‌نویسی - دانشگاه صنعتی شریف}
	\\
	علیرضا توفیقی محمدی - رشته علوم کامپیوتر - شماره‌دانشجویی: ۹۶۱۰۰۳۶۳
	\\
	امیرحسین قربانی - رشته ریاضی - شماره‌دانشجویی: ۹۶۱۰۰۲۴۴
	\\
	علیرضا عظیمی‌نیا - رشته علوم کامپیوتر - شماره‌دانشجویی: ۹۶۱۰۰۴۵۸
	\\
	امیرحسین زارع - رشته علوم کامپیوتر - شماره‌دانشجویی: ۹۶۱۰۰۳۹۶
	\\
	امیرحسین ندیری - رشته علوم کامپیوتر - شماره‌دانشجویی: ۹۶۱۰۰۵۳۳
\end{center}

\section*{پاسخ}
\section*{۱}
\subsection*{۱.۲}
در هر ماه افرادی که به تعداد بلیط‌داران اضافه می‌شوند ۴ برابر می‌شود، تعداد افراد اضافه‌شده در ماه صفرم ۱ نفر، ماه بعد ۴ نفر، ماه بعد ۱۶ نفر و ... است پس تعداد کل افرادی که در ماه $n$ ام بلیط دارند برابر با:
$$
\sum_{i = 0}^n 4^i = \frac{4^{n+1}-1}{4-1}
$$
است، که باید اولین $n$ای که مقدار بالا از 
$50\times 10^6$
بیشتر شود را پیدا کنیم که این $n = 13$ صادق است پس ۱۳ ماه یعنی یک سال و یک ماه طول می‌کشد تا همه‌ی متقاضیان صاحب بلیط شوند.
\subsection*{۱.۳}
آخرین گروه دارای دوچرخه نیستند یعنی $4^{n}$ نفر که $n=13$ بود یعنی 
67108864 
نفر صاحب بلیط‌اند اما دوچرخه ندارند.
\subsection*{۱.۴}
فقط افراد ماه آخر دارای دوچرخه نیستند پس افراد ماه یکی‌مانده به آخر دارای دوچرخه هستند ولی برخی معرفی‌شدگان آن‌ها صاحب دوچرخه نیستند یعنی
$4^{n-1} = 4^{12} = 16777216$
نفر.
\subsection*{۱.۵}
همچون سوال ۱.۲ مسئله را حل می‌کنیم و 
$n = 17$
به دست می‌آید.
\subsubsection*{۱.۵.۲}
تعداد افرادی که بلیط دارند ولی دوچرخه ندارند $4^n = 4^{17} = 17179869184$.
\subsubsection*{۱.۵.۳}
تعداد افرادی که دوچرخه دارند ولی برخی از معرفی‌شدگان آن‌ها دوچرخه ندارند:
$n = 4^{16} = 4294967296$
\subsection*{۱.۶}
بله، زمان پیدا کردن مشتری برای همه‌ی افراد یکسان نیست و همچنین با گذر زمان این زمان به دلیل محدود شدن جامعه‌ی هدف افزایش می‌یابد.
\section*{۲}
\subsection*{۲.۱}
\begin{itemize}
	\item $n_0 \geq n_1$
	\item $n_0 \geq n_2$
	\item بین $n_0$ و $n_1$ رابطه‌ای نیست.
\end{itemize}
\subsection*{۲.۲}
تعداد افرادی که در هفته‌ی $i$ ام عضو شرکت شده‌اند 
$2^i$
نفر است پس پس از $n$ هفته، 
$2^{n+1}-1$
نفر عضو سیستم می‌شوند.
\\
عمق ۱ تا ۴ از پایین سرشان کلاه رفته که این تعداد 
$2^n + 2^{n-1} + 2^{n-2} + 2^{n-3}$
نفرند پس جواب مسئله 
$\frac{{den}2^n + 2^{n-1} + 2^{n-2} + 2^{n-3}}{2^{n+1}-1}$
است.
\subsection*{۲.۳}
افراد عمق ۵ ام از پایین جایزه می‌گیرند ولی زیرشاخه‌های آن‌ها جایزه نمی‌گیرند پس
$2^{n-4}$
نفر جایزه گرفته‌اند ولی زیرشاخه‌ی مستقیمی دارند که سرش کلاه رفته پس 
$\frac{2^{n-4}}{2^{n+1}-1}$
جواب مسئله است.
\subsection*{۲.۴}
تعداد کل افراد $2^{n+1}-1$ نفر است که از عمق ۵ام از پایین به بالا صاحب جایزه می‌شوند که این افراد 
$2^{n-3}-1$
 نفر است پس سود شرکت برابر با
 $2^{n+1}-1 - 10\times (2^{n-3}-1)$
 است.
\subsection*{۳.۱}
اگر تعداد افرادی که بدون شاخه هستند را $n_0$، تعداد افرادی که یک شاخه دارند را $n_1$ و بقیه را $n_2$ بنامیم. در هفته‌ی نخست:
$$
n_0 = 1, n_1 = 0, n_2 = 0
$$
است، هفته‌‌های بعد به صورت زیر است:
$$
1: n_0 = 1, n_1 = 1, n_2 = 0$$$$
2: n_0 = 1, n_1 = 1, n_2 = 1$$$$
3: n_0 = 2, n_1 = 1, n_2 = 2$$$$
4: n_0 = 3, n_1 = 2, n_2 = 3$$$$
5: n_0 = 5, n_1 = 3, n_2 = 5$$$$
6: n_0 = 8, n_1 = 5, n_2 = 8$$$$
7: n_0 = 13, n_1 = 8, n_2 = 13$$$$
...
$$
اگر دقت کنیم می‌بینیم که $n_0$ و $n_1$ و $n_2$ دنباله‌ی فیبوناچی است که اثبات آن هم راحت است و از اثبات صرف نظر می‌کنیم.
\\
پس در هفته‌ی ۲۰ ام این تعداد برابر با 28656 است.
\subsection*{۳.۲}
با نوشتن جمله‌ی عمومی اعداد فیبوناچی نتیجه می‌گیریم که این تعداد هرهفته حدود
$\frac{1+\sqrt{5}}{2}$
است.
\subsection*{۳.۳}
دادیم.
\subsection*{۳.۴}
هر فرد باید ۶ هفته صبر کند تا به سود رسانی برسد زیرا هفته‌ی ۶ ام این تعداد بیشتر از ۱۵ است.
\\
همچنین طبق قسمت ۳.۲ افراد پس از هر هفته حدود
$\frac{1+\sqrt{5}}{2}$
برابر می‌شوند پس اگر در هفته‌ی 
$n-6$
ام تعداد $x$ نفر باشد، در هفته‌ی $n$ ام تعداد 
$x \times (\frac{1+\sqrt{5}}{2})^6$
می‌شود که همه غیر از $x$ نفر ضرر می‌کنند پس کسر
$$
\frac{(\frac{1+\sqrt{5}}{2})^6 - 1}{(\frac{1+\sqrt{5}}{2})^6}
$$
از افراد ضرر می‌کنند.
\section*{۴}
\subsection*{۴.۱}
طبق نوشته‌ی متن، وقتی هنوز پدیده‌ی اشباع رخ نداده‌است، نمودار رشد یک نمودار نمایی است پس تابعی مانند 
$y = a^{bx}\times c$
است.
حال اگر از این نمودار لگاریتم بگیریم خواهیم داشت:
$$ \log y = \log (a^{bx}\times c) \rightarrow  \log y = bx \log a + \log c$$
که یک نمودار خطی است.  پس قسمت اول نمودار دوم باید به صورت خطی باشد.
\subsection*{۴.۲}
وقتی اشباع رخ می‌دهد که سرعت رشد اعضا کاهش شدید پیدا کند، این کاهش شدید در هفته‌ی ۳۵ام رخ داده‌است پس زمان اشباع باید حدودا هفته‌ی ۳۵ ام باشد.
\subsection*{۴.۳}
می‌توان زمان اشباع را با کمک مشتق نمودار بیان کرد، در زمان اشباع سرعت افزایش اعضا کاهش شدید پیدا می‌کند پس می‌توان زمان اشباع را اولین زمانی دانست که مشتق نمودار  کمتر از یک‌دهم تغییرات متوسط اعضا از ابتدا تا آن موقع باشد.
\subsection*{۴.۴}
هرچه نرخ بیشتر باشد، سرعت رشد هم بیشتر است اما رابطه‌ای خطی بین آنها برقرار نیست.
\subsection*{۴.۵}
برای کسی که به شرکت می‌پیوند زمان حداکثر سرعت رشد زمان باخت است، زیرا از این پس تقریبا نمی‌توان کسی را به شرکت اضافه کرد و پول داده‌شده پس داده نمی‌شود.
\\
برای شرکت این زمان زمانی است که دیگر نباید پولی به اعضا بدهد و البته دیگر سود‌دهی قابل توجهی نیز نخواهد داشت و باید به فکر کسب و کاری دیگر باشد.
\subsection*{۴.۶}
اگر $x$ نفر عضو شرکت باشند در یک جمعیت $2n$ نفره احتمال اینکه یک نفر که عضو شرکتی نیست پیدا کنیم قبل از فرهنگ‌سازی برابر با 
$\frac{2n-x}{2n}$
بود، حال این احتمال برابر با 
$\frac{n-x}{2n}$
است که احتمال کمتر از یک‌دوم می‌شود، پس به همین ترتیب تعداد افراد در روز نیز کمتر از یک‌دوم می‌شود (چون تمام احتمال‌ها کمتر از یک‌ذوم حالت قبل است پس تعداد افرادی هم که پیدا می‌شوند باید کمتر از یک‌دوم حالت قبل باشد)  پس در روز اشباع حالت قبل، تعداد افراد حالت جدید کمتر از یک‌دوم حالت قبل است پس نمودار مربعی غلط و نمودار مثلثی درست است.
\subsection*{۴.۷}
در این مسئله فقط افراد بدون زیرشاخه با حداکثر مدت عضویت $k$ هفته و با یک زیرشاخه با حداکثر مدت یک‌زیرشاخه‌دار شدن $k$ هفته دست به تلاش برای عضو کردن دیگران می‌زنند.

برای $k=1$ این مقدار حدود ۶۰۰۰۰۰ و برای $k=2$ این مقدار حدود  ۹۰۰۰۰۰ و برای $k\geq35$ این مقدار همان ۱۰۰۰۰۰۰ است (چون در هفته‌ی ۳۵ ام جامعه اشباع می‌شود.)
حال تابع
$f(k) = n - \frac{n}{(k+1)(k+1)}$
تقریبا جوابی همچون ۳ داده‌ای که داریم می‌دهد.
\subsection*{۴.۸}
اولا طبق استدلال ۴.۶ زمان رسیدن به آستانه‌ی اعضا طولانی تر می‌شود. ثانیا آستانه‌ی اعضا بسیار کاهش می‌یابد.
\subsection*{۴.۹}
در حالت واقعی تر افراد علاوه بر اینکه سعی می‌کنند برای خودشان شاخه پیدا کنند، سعی برای پیدا کردن زیرشاخه برای زیرشاخه‌هایشان نیز می‌کنند تا ضریب تعادل خود را ایجاد کنند. می‌توان این حالت را نیز به مسئله اضافه کرد.
\subsection*{۴.۱۰}
تاثیر تلوزیون به صورت ثابت و ۱/۲ جامعه در نظر گرفته شد ولی همچون رشد تعداد افراد متقاضی، تعداد افراد آگاه نیز رشد می‌کنند.
\\
همچنین می‌توان برای پول اولیه‌ی ثبت‌نام و جایزه نیز حالت‌های دیگری در نظر گرفت و رشد آن را بررسی کرد.
\end{document}
