\documentclass[12pt,a4paper]{article}
\usepackage{multirow}
\usepackage{rotating}
\usepackage{graphicx}
\usepackage{amsmath}
\usepackage{amsfonts}
\usepackage{amssymb}
\usepackage{graphicx}
\pagestyle{empty}
\usepackage[bottom=0.5in,headheight=0pt,headsep=0pt]{geometry}
\addtolength{\topmargin}{0pt}
\usepackage{xepersian}
\linespread{1.2}
\settextfont{XB Niloofar}

\begin{document}
\begin{center}
	بسمه تعالی
\end{center}
\begin{center}
	\textbf{پاسخ تمرین تحویلی هشتم - درس ریاضی‌نویسی - دانشگاه صنعتی شریف}
	\\
	علیرضا توفیقی محمدی - رشته علوم کامپیوتر - شماره‌دانشجویی: ۹۶۱۰۰۳۶۳
\end{center}
\section*{پرسش}
یک میز داریم که $2^n$ صندلی دارد که روی هر کدام از صندلی ها چراغی قرار گرفته، چشم استاد بسته‌شده و هر دفعه تعدادی از جایگاه‌های صندلی را انتخاب کرده و وضغیت چراغ‌های روی آن را تغییر می‌دهد. اما پس از هر مرحله دانشجویان چراغ‌های روی میزها را به صورت ساعت‌گرد به مقدار دلخواه می‌چرخانند.\\
همچنین درصورتی که همه‌ی چراغ‌ها روشن شود میز بوق می‌زند. ثابت کنید اگر وضعیت اولیه‌ی چراغ‌ها نامشخص باشد، استاد می‌تواند طی تعدادی محدود حرکت کاری کند که میز حتما بوق بزند.
\section*{پاسخ}
صورت سوال معادل حالت زیر است:

 می‌خواهیم دنباله‌ای از زیرمجموعه‌های اعداد ۱ تا 
$2^n$
بدیم که در صورتی که در مرحله‌ی $i$ام، وضعیت کلید روی صندلی‌ها با شماره‌ی اعضای مجموعه‌ی $i$ام از دنباله را تغییر دهیم و مرحله‌ای وجود داشته باشد که در آن همه‌ی چراغ‌ها روشن شوند.
\\
معادل بودن دو حکم کامل مشخص است و نیازی به توضیح برای آن نمی‌بینم.

\textbf{تعریف}
به یک دنباله جذاب می‌گوییم اگر آن دنباله یکی از دنباله‌های جواب مسئله باشد یعنی با اجرای آن حتما مرحله‌ای وجود داشته باشد که در آن همه‌ی چراغ ها روشن شوند.
\\
\subsection*{لم}
اگر دنباله‌ی $A$ جذاب باشد، دنباله‌ی $A'$ که با اضافه کردن تعدادی عضو $\{\}$ به دنباله به وجود می‌آید نیز جذاب است.
\\
\\
\textbf{اثبات:}
فرض کنید قبلا مجموعه‌ی $X$ داشتیم و سپس مجموعه‌ی $Y$ حال ادعا می‌کنیم با اضافه شدن تعداد تهی در بین $X$ و $Y$ هیچ تفاوتی در جواب ایجاد نمی‌شود.
\\
دلیل این است که در این مراحل جایگشت دوری وضعیت چراغ‌ها هیچ تغییری نمی‌کند چون فقط دانشجویان چراغ را می‌چرخانند، حال فرض کنید چرخش $i$ ام $a_i$ واحد ساعتگرد چراغ‌ها را چرخانده ایم، این دقیقا معادل حالتی است که در مراحل اول تا $k-1$ (که $k$ تعداد مجموعه‌های تهی اضافه شده‌است) لامپ‌ها را نچرخانیم و سپس
$\sum_{i=1}^k a_k$
واحد ساعتگرد بچرخانیم. پس وضعیت با تعدادی مجموعه‌ی تهی به وضعیتی بدون آن مجموعه‌های تهی منتاظر شد، پس چون آن وضعیت جذاب بود پس $A'$ نیز جذاب است.
\subsection*{حکم مسئله}
برای اثبات حکم روی $n$ استقرا می‌زنیم.
\subsubsection*{پایه}
حکم برای $n=0$ برقرار است. اگر یک چراغ داشته باشیم، دنباله‌ی $A = {1}$ یک دنباله‌ی جذاب و جواب مسئله است، چون یا در ابتدا تنها لامپی که داریم روشن است، یا خاموش است که در یک حرکت آن‌را روشن کرده و در هر دو حالت حالتی را می‌بینیم که همه‌ی چراغ ها روشن‌اند.
\subsubsection*{فرض}
برای $2^k$ چراغ دور یک میز، مجموعه‌ای جذاب مثل $A$ وجود دارد.
\subsubsection*{حکم}
برای 
$2^{k+1}$
چراغ دور میز مجموعه‌ای مثل $B$ وجود دارد که جذاب است.
\\
\\
برای اثبات حکم صندلی‌ها را به ترتیب ساعتگرد از میزی دلخواه از اعداد ۱ تا 
$2^{k+1}$
شماره گذاری می‌کنیم.
\\
حال وضعیت چراغ روی صندلی $i$ام را $l_i$ می‌نامیم.
\\
دو میز فرضی $X$ و $Y$ در نظر می‌گیریم که روی هر کدام $2^k$ صندلی و روی هر صندلی یک چراغ قرار گرفته است.
\\
چراغ های هر میز را به ترتیب ساعتگرد با شماره‌های ۱ تا $2^k$ شماره‌گذاری می‌کنیم و چراغ $i$ ام از $X$ را با $x_i$ و چراغ $i$ام از $Y$ را با $y_i$ نمایش می‌دهیم.
\\\\
حال وضعیت $x_i$ را معادل با وضعیت $l_i$ در نظر می‌گیریم، همچنین اگر وضعیت دو چراغ 
$l_i$ و
$l_{2^k + i}$
یکسان بودند وضعیت $y_i$ را روشن و در غیر اینصورت خاموش در نظر می‌گیریم.
\\
حال طبق فرض استقرا دنباله‌ی جذابی برای میزی با شرایط مسئله و $2^k$ صندلی وجود دارد، این دنباله را $A$ بنامیم و مجموعه‌ی $i$ام آن‌را $A_i$ و $|A| = m$
\\
حال دنباله‌ی $H$ را به این صورت می‌سازیم که 
$$H_i = \{x | x \in A_i \} \cup \{x + 2^k | x \in A_i \}$$
\\
حال دنباله‌ی $B$ را با الگوریتم زیر می‌سازیم:
\\
دنباله‌ی خالی $B = H$ را در نظر بگیرید و سپس $m$ مرحله فرایند زیر را طی کنید:
\\
در مرحله‌ی $i$ ام ابتدا $A_i$ را به انتهای $B$ اضافه کرده و سپس کلیه‌ی اعضای $H$ را به ترتیب به انتهای $B$ اضافه می‌کنیم.
\\\\
حال ادعا می‌کنیم $B$ یک دنباله‌ی جذاب برای حالت 
$2^{k+1}$
است.
\\
برای اثبات چند قدم را طی می‌کنیم:
\\
\textbf{:قدم ۱}
هر چرخش در میزاصلی معادل چرخش به همان میزان در میز $Y$ است.
\\
برای راحتی کار یک چرخش $x$تایی ساعتگرد را معادل $x$ چرخش یکتایی ساعتگرد در نظر می‌گیریم، حال ثابت می‌کنیم یک چرخش یکتایی معادل است، پس $x$تا چرخش یکتایی نیز معادل بوده و یک چرخش $x$تایی معادل است.
اگر لامپ روشن را با ۱ و خاموش را با ۰ نشان دهیم، در میز اصلی این وضعیت قبل از چرخش به صورت زیر است:\\
$l_1, l_2, ..., l_{2^{k+1}}$
\\
و بعد از چرخش به صورت زیر می‌شود:\\
$l_{2^{k+1}}, l_1, l_2, ..., l_{2^{k+1}-1}$
\\\\
حال اگر $x | y$ به معنی نقیض یای مانع‌الجمع منطقی $x$ و $y$ باشد  (نقیض ایکس اور)وضعیت لامپ $y_i$ به صورت زیر است:
\\
$x_i = l_i | l_{2^k + i}$
\\
پس دنباله‌ی وضعیت لامپ‌ها قبل از چرخش به صورت زیر بوده:
\\
$l_1 | l_{2^k + 1}, l_2 | l_{2^k + 2}, \dots, l_{2^k} | l_{2^{k + 1}}$
\\
و بعد از چرخش به صورت زیر می‌شود:
\\
$l_{2^{k+1}} | l_{2^k}, l_1 | l_{2^k + 1}, \dots, l_{2^k - 1} | l_{2^{k + 1} - 1}$
\\
که چون $a| b = b | a$ است پس دنباله‌ی بالا معادل یک واحد چرخش دنباله‌ی قبل به صورت ساعت گرد است و قدم اول ثابت شد.
\\
\textbf{قدم دوم}
حال دنباله‌ی $B$ را در نظر بگیرید، در دنباله‌ی $B$ هر مرحله یک بار $A_i$ انجام می‌دهند، این تغییر روی $A_i$ باعث تغییر همین عنصر ها روی $Y$ می‌شود زیرا دقیقا یک عنصر از $w$ و $2^k+w$ را تغییر می‌دهد. همچنین بقیه چون هم $w$ و هم $2^w+k$ می‌دهد هیچ تغییری روی $Y$ پس $H$ روی $Y$ همانند دنباله‌ای از $A$ است که تعدادی عضو تهی به آن اضافه کرده اند و در قدم قبل چرخش را برای $Y$ ثابت کردیم پس طبق فرض استقرا وضعیتی وجود دارد که در آن همه‌ی لامپ‌ها در $Y$ روشن می‌شوند یعنی وضعیت لامپ دلخواه $w$ با $2^k+w$ یکسان می‌شود.
\\
حال اولین دفعه‌ای که مراحل ناتهی که این اتفاق می‌افتد را در نظر بگیرید، حال طبق فرایند ساخت $H$ در این مرحله ما مراحل با یکسان بودن $w$ و $2^k+w$ را روی میز اصلی اجرا می‌کنیم، این مراحل هیچ تغییری روی $Y$ انجام نمی‌دهند پس $Y$ بازهم روشن می‌ماند. 
\\
اگر $Y$ روشن به طور کامل روشن باشد هر چرخش در میزاصلی معادل چرخش به همان میزان در میز $X$ است.
\\
برای راحتی کار یک چرخش $x$تایی ساعتگرد را معادل $x$ چرخش یکتایی ساعتگرد در نظر می‌گیریم، حال ثابت می‌کنیم یک چرخش یکتایی معادل است، پس $x$تا چرخش یکتایی نیز معادل بوده و یک چرخش $x$تایی معادل است.
اگر لامپ روشن را با ۱ و خاموش را با ۰ نشان دهیم، در میز اصلی این وضعیت قبل از چرخش به صورت زیر است:\\
$l_1, l_2, ..., l_{2^{k+1}}$ 
\\
همچنین چون تمام لامپ‌های $Y$ روشن است داریم:
\\
$l_i = l_{2^k + i}$
\\
پس وضعیت لامپ‌ها به صورت زیر است:
$$l_1, l_2, \dots, l_{2^k}, l_1, l_2, \dots, l_{2^k}$$
یک چرخش ساعتگرد وضعیت را به صورت زیر می‌کند:
$$l_2, l_3, \dots, l_{2^k}, l_1, l_2, \dots, l_{2^k}, l_1$$
\\
حال $x_i$ برابر وضعیت لامپ در صندلی $i$ ام بود پس $x_i$ در قبل از انجام این فرایند به شکل:
$$l_1, l_2, \dots l_k$$
و بعد از انجام فرایند به شکل:
$$l_2, l_3, \dots, l_k, l_1$$
شد پس یک چرخش در میز اصلی یک چرخش در $X$ را به دنبال داشت.
\\
حال چون چرخش در $X$ نیز برقرار است پس $X$ ویژگی‌های یک میز با شرایط مسئله را دارد پس طبق فرض استقرا دنباله‌ی $A$ یک دنباله‌ی جذاب برای آن است، حال $m$ حرکت بعد از روشن شدن $Y$ را در نظر بگیرید، این $m$ حرکت روی $X$ دقیقا تاثیر $A$ را دارند (اگر در مرحله‌ی $i$ ام $A$ صندلی‌های $A_i$ را تغییر می‌داد، دنباله‌ی $m$ حرکت بعدی در $H$ نیز دقیقا همین کار را می‌کند زیرا مجموعه‌ی صندلی‌های $A_i$ و صندلی‌های $2^k$ تا بیشتر را تغییر می‌داد که چون $X$ وضعیت صندلی‌های ۱ تا $2^k$ است فقط مجموعه صندلی‌های $A_i$ در آن تغییر می‌کند.) پس این $m$ حرکت در لحظه‌ای باعث روشن‌شدن کامل $X$ می‌شوند.
\\
چون این حرکات تغییری روی $Y$ ایجاد نمی‌کردند و $Y$ تمام روشن بود پس در یک لحظه هم $Y$ تمام روشن شد و هم $X$.
\\
 پس همه‌ی لامپ‌های صندلی ۱ تا $2^k$ روشن بوده و لامپ $u$ ام از لامپ‌های بعدی با لامپ $u-2^k$ که روشن بود وضعیت یکسان دارد پس در این حالت همه‌ی لامپ ها روشن می‌شود و پس $H$ جذاب است و حکم ثابت شد.
 \\
 \\

پس حکم برای همه‌ی $n$های حسابی ثابت شد و مسئله حل شد.

\end{document}



