\documentclass[12pt,a4paper]{article}
\usepackage{multirow}
\usepackage{rotating}
\usepackage{graphicx}
\usepackage{amsmath}
\usepackage{amsfonts}
\usepackage{amssymb}
\usepackage{graphicx}
\pagestyle{empty}
\usepackage[bottom=0.5in,headheight=0pt,headsep=0pt]{geometry}
\addtolength{\topmargin}{0pt}
\usepackage{xepersian}
\linespread{1.2}
\settextfont{XB Niloofar}

\begin{document}
\begin{center}
	بسمه تعالی
\end{center}
\begin{center}
	\textbf{پاسخ تمرین تحویلی هفتم - درس ریاضی‌نویسی - دانشگاه صنعتی شریف}
	\\
	علیرضا توفیقی محمدی - رشته علوم کامپیوتر - شماره‌دانشجویی: ۹۶۱۰۰۳۶۳
\end{center}
\section*{پرسش}
$n$ 
پمپ بنزین روی یک مسیر دایره‌ای داریم که مجموع بنزین روی آنها برای حرکت یک دور کافی‌ست، ماشینی با باک خالی داریم، ثابت کنید این ماشین می‌تواند با شروع از پمپ‌بنزین مناسب یک دور کامل بزند.
\section*{پاسخ}
\subsection*{لم:}
\textbf{اگر 
$n > 1$
باشد، پمپ‌بنزینی وجود دارد که ماشین با شروع از آن می‌تواند به صورت ساعت‌گرد به پمپ بنزین بعدی برسد.}
\\
برای اثبات از برهان خلف استفاده می‌کنیم، فرض کنید پمپ بنزین ها با شماره‌های 
$1, 2, \cdots, n$
به ترتیب ساعت‌گرد نام‌گذاری شده‌باشند و $n+1 = 1$ و هیچ پمپ بنزینی مثل $i$ وجود نداشته باشد که بتوان با بنزینش از $i$ به $i+1$ رفت.
در این صورت بنزین هر پمپ بنزین کمتر از مقدار مورد نیاز برای مسافت آن پمپ بنزین تا پمپ بنزین بعدی آن است پس مجموع بنزین پمپ‌بنزین‌ها کمتر از مجموع مسافت هر پمپ بنزین به پمپ بنزین بعدی است که نتیجه می‌گیریم مجموع بنزین پمپ بنزین‌ها کمتر از یک دور است و این با فرض مسئله در تناقض است.\\
پس فرض خلف باطل و لم ثابت شد.
\subsection*{مسئله‌ی اصلی}
مسئله‌ی اصلی را با کمک استقرا روی $n$ که همان تعداد پمپ‌بنزین‌هاست ثابت می‌کنیم.
\\
همچنین فرض می‌کنیم ماشین فقط به صورت ساعتگرد حرکت می‌کند.
\\
\subsubsection*{پایه}
حکم برای $n=1$ برقرار است، زیرا تنها یک پمپ بنزین داریم که بنزین آن برای یک دور کافی است. پس با شروع از آن به صورت ساعتگرد می‌توان یک دور حرکت کرد.
\subsubsection*{فرض}
اگر 
$n=k$
 پمپ بنزین داشتیم که مجموع بنزین آنها برای یک دور کافی بود، ماشین می‌توانست با شروع از پمپ‌بنزینی مناسب کل مسیر را دور زند.
\subsubsection*{حکم}
حال با کمک فرض استقرا ثابت می‌کنیم گزاره‌ی بالا برای
$n=k+1$
برقرار است.
\\
پمپ‌بنزین ها با با شماره‌های ۱ تا $k+1$ به صورت ساعت‌گرد شماره‌گذاری می‌کنیم.
\\
چون $k+1 > 1$ است، طبق لم حداقل یک پمپ بنزین مثل پمپ بنزین $i$ وجود دارد که از $i$ می‌توان به $i+1$ رفت. 
$(n+1 = 1)$
\\
اگر بنزین پمپ بنزین $j$ام را با $a_j$ نشان دهیم، پمپ بنزین $i+1$ ام را در نظر نگرفته و بنزین آن را داخل پمپ‌بنزین $i$ ام می‌ریزیم، یعنی بنزین پمپ بنزین $i$ام برابر با $a_i + a_{i+1}$ می‌شود.
\\
حالت قبلی را حالت $k+1$ و حالت جدید را حالت $k$ می‌نامیم.
\\
حال بازهم مجموع بنزین پمپ‌بنزین‌ها برای یک دور کامل کافی بوده و همچنین تعداد پمپ بنزین‌ها $k$ تا شد، پس طبق فرض استقرا ماشین می‌تواند از پمپ‌بنزینی مناسب همچون $x$ شروع کند و در حالت جدید یک دور کامل بزند.
\\
ماشین در این حرکت در لحظه‌ای به پمپ‌بنزین $i$ رسیده‌است، تا این مدت هیچ تفاوتی بین حالت $k$ام ساخته شده و حالت $k+1$ای که داشتیم نیست، پس می‌تواند به همین طریق در $k+1$ نیز حرکت کند.\\
فرض کنید در این لحظه ماشین دارای بنزین $l$ باشد  (لحظه‌ی قبل از رسیدن به پمپ $i$ام)و مسیر بین $a_i$ و $a_i+1$ نیاز به $o$ بنزین داشته باشد.
\\
چون می‌توانستیم از $i$ به $i+1$ برسیم پس $a_i > o$ است، پس $l+a_i - o > 0$ است پس ماشین در حالت $k+1$ می‌تواند تا پمپ بنزین $i+1$ نیز حرکت کند و در لحظه‌ی عبور از آن بنزین ماشین برابر با 
$l+a_i - o + a_{i+1}$
می‌شود.\\
حال لحظه‌ای که ماشین در حالت $k$ به جایگاه پمپ بنزین $a_{i+1}$ (که در این حالت پمپ بنزینی در آنجا نیست) برسد مقدار بنزینش برابر با 
$l+ a_i + a_{i+1} - o = l + a_i - o + a_{i+1}$
است که این با حالت بالا برابر است، پس ادامه‌ی مسیر نیز کاملا مشابه است و ماشین همانگونه که می‌تواند در حالت $k$ ادامه‌ی مسیر را طی کند در حالت $k+1$ نیز می‌تواند ادامه‌ی مسیر را طی کند، پس  در حالت $k+1$  ماشین با شروع از $x$ می‌تواند یک دور کامل بزند، پس حکم استقرا ثابت و حکم مسئله به روش استقرا ثابت شد.
\end{document}



