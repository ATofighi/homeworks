\documentclass[12pt,a4paper]{article}
\usepackage{multirow}
\usepackage{rotating}
\usepackage{graphicx}
\usepackage{amsmath}
\usepackage{amsfonts}
\usepackage{amssymb}
\usepackage{graphicx}
\pagestyle{empty}
\usepackage[bottom=0.5in,headheight=0pt,headsep=0pt]{geometry}
\addtolength{\topmargin}{0pt}
\usepackage{xepersian}
\linespread{1.2}
\settextfont{XB Niloofar}

\begin{document}
\begin{center}
	بسمه تعالی
\end{center}
\begin{center}
	\textbf{پاسخ تمرین تحویلی نهم - درس ریاضی‌نویسی - دانشگاه صنعتی شریف}
	\\
	علیرضا توفیقی محمدی - رشته علوم کامپیوتر - شماره‌دانشجویی: ۹۶۱۰۰۳۶۳
\end{center}
\section*{پرسش}
$n$ 
شکلات داریم، دو نفر به صورت نوبتی روی این شکلات بازی انجام می‌دهند، هر فرد می‌تواند از ۱ تا حداکثر نصف تعداد شکلات‌های باقی‌مانده را برداشته و بخورد. کسی که نتواند حرکتی کند بازنده‌ی این مسابقه‌ست. چه کسی برنده می‌شود؟
\section*{پاسخ}
برای یک حالت بازی با $n$ شکلات این حالت را حالت برد می‌گوییم و با \lr{W} نشان می‌دهیم اگر شروع‌کننده‌ی آن حالت بتواند طوری بازی کند که برنده‌ی بازی شود و در غیر این‌صورت آن‌را با
\lr{L}
نشان می‌دهیم.
\\
\textbf{تعریف:}
به یک عدد جالب می‌گوییم اگر توانی از دو مانند $2^k$ وجود داشته باشد که یک واحد از این عدد بیشتر باشد.
\\
به طور مثال ۱،۳،۷،۱۵ چهار کوچکترین اعداد جالب طبیعی هستند.
\\
\textbf{تعریف:}
جالبی یک عدد را بزرگترین عدد جالب کوچکتر مساوی آن عدد تعریف می‌کنیم.
\\
\textbf{ادعا:}
اگر $n$ جالب باشد حالت باخت و در غیر اینصورت حالت برد است.
\\
\\
برای اثبات از استقرا استفاده می‌کنیم:
\subsection*{پایه}
برای $n=1$ عدد یک جالب است و همچنین حالت باخت است، زیرا شروع‌کننده‌ی بازی باید حداکثر ۰.۵ شکلات برداشته پس نمی‌تواند شکلاتی را بردارد و بازنده‌ی بازی می‌شود.
\subsection*{فرض}
اگر $n<k$ باشد، آنگاه اگر $n$ جالب باشد بازی با $n$ شکلات حالت باخت و در غیراین‌صورت حالت برد است.
\subsection*{حکم}
حال حکم بالا را برای $n=k$ ثابت می‌کنیم:
\\
اگر جالبی عدد $n$ را با 
$2^m - 1$
نشان دهیم
\subsubsection*{حالت ۱}
اگر 
$n \neq 2^m-1$
باشد:
\\
پس در این صورت داریم
$n = 2^m - 1 + r$
که $r > 0$ است، همچنین 
$r \leq 2^m-1$
\\
(زیرا اگر 
$r > 2^m-1$
باشد در این صورت
\\
$n > 2^m-1 + 2^m-1 = 2^{m+1}-2 \rightarrow n \geq 2^{m+1} \geq 2^{m+1}-1$
\\
که عدد جالبی $n$ باید حداقل $2^{m+1}-1$ می‌شد که تناقض است.)
\\
پس چون 
$r \leq 2^m-1$
است، نتیجه‌می‌گیریم 
$r \leq \frac{n}{2}$
پس نفر اول می‌تواند $r$ شکلات برداشته و بازی را به حالت $n-r = 2^m-1$ شکلاته تبدیل کند، حال چون $n-r < n$ است و $n-r$ جالب است، طبق فرض استقرا این حالت، حالتِ باخت است پس نفر اول توانست بازی را به حالت باخت تبدیل کند پس این حالت حالت برد نفر اول بود.
\subsubsection*{حالت دوم}
اگر $n=2^m-1$ باشد، آنگاه نفر اول می‌تواند این حالت را به تمام حالت‌های $n-1$، $n-2$ و ... و 
$n - \lfloor\frac{n}{2}\rfloor = 2^m-1 - (2^{m-1}-1) = 2^{m-1}$
تبدیل کند، همچنین عدد جالب قبل از $2^m-1$ عدد $2^{m-1}-1$ است که از تمام اعداد بالا کوچک‌تر است، پس تمام حالات‌بالا چون حالت کمتری از $n$ اند و همچنین عدد جالب نیستند پس طبق فرض استقرا حالت برد اند، پس نفر اول هرکاری کند به حالتی برد برای شروع‌کننده‌ی جدید بازی (نفر دوم) می‌رسد پس این حالت حالت باخت است.
\\
\\
پس هر دو حالت حل شد و حکم استقرا ثابت و گام استقرا ثابت و حکم برای تمام $n$ های طبیعی ثابت شد.
\\
\\
\\
پس اگر $n$ برابر عددی به فُرم 
$2^m - 1$
بود، چون حالت باخت است، نفر دوم استراتژی برد دارد و در غیر این‌صورت نفر اول استراتژی برد دارد و مسئله حل شد.

\end{document}



