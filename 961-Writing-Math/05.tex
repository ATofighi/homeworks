\documentclass[12pt,a4paper]{article}
\usepackage{multirow}
\usepackage{rotating}
\usepackage{graphicx}
\usepackage{amsmath}
\usepackage{amsfonts}
\usepackage{amssymb}
\usepackage{graphicx}
\pagestyle{empty}
\usepackage[bottom=0.5in,headheight=0pt,headsep=0pt]{geometry}
\addtolength{\topmargin}{0pt}
\usepackage{xepersian}
\linespread{1.2}
\settextfont{XB Niloofar}

\begin{document}
\begin{center}
	بسمه تعالی
\end{center}
\begin{center}
	\textbf{پاسخ تمرین تحویلی پنجم - درس ریاضی‌نویسی - دانشگاه صنعتی شریف}
	\\
	علیرضا توفیقی محمدی - رشته علوم کامپیوتر - شماره‌دانشجویی: ۹۶۱۰۰۳۶۳
\end{center}
\section*{پرسش}
جسمی داریم که هر سطح مقطع گذرنده از آن دایره‌است، ثابت کنید این جسم کره است.
\section*{پاسخ}
\subsection*{لم}
\textbf{
هر خط گذرنده از فضا یا هیچ چیزی از جسم جدا نمی‌کند، یا یک پاره‌خط از جسم جدا می‌کند. (نقطه را حالتی خاص از پاره خط در نظر می‌گیریم.)
}
\\
اثبات:
\\
دو حالت داریم، یا خط هیچ اشتراکی با جسم ندارد که در این صورت هیچ چیزی از جسم جدا نمی‌کند، یا با جسم اشتراک دارد، حال صفحه‌ای دلخواه که از این خط می‌گذرد در نظر بگیرید، چون خط با جسم اشتراک داشت پس این صفحه نیز با جسم اشتراک دارد، پس طبق فرض مسئله، سطح مقطع دایره‌ای از جسم جدا می‌کند، چون دایره اشتراک صفحه و جسم و بخش جدا شده توسط خط اشتراک خط و جسم است، چیز جدا شده از جسم زیر مجموعه‌ای از دایره است و بخشی از خط است که از دایره جدا کرده است پس وتری از دایره است که یک پاره خط است و لم ثابت شد.
\subsection*{مسئله‌ی اصلی}
از بین تمام خط‌هایی که با جسم اشتراک دارند و طبق لم پاره‌خطی از جسم جدا می‌کنند، خطی را که بزرگترین طول پاره‌خط جداشده را دارد در نظر بگیرید، طول این پاره خط را $d$ در نظر می‌گیریم. صفحه‌ای گذرنده از این خط دایره‌ای با قطری همچون $2r$ در نظر بگیرید:
\begin{itemize}
\item
 اگر $2r>d$ باشد آنگاه خطی وجود دارد که از قطر دایره می‌گذرد و پاره‌خطی به طول $2r$ از جسم جدا می‌کند، پس پاره‌خطی یافت می‌شود که طولی بیشتر از $d$ از جسم جدا می‌کند که این با بیشینه بودن $d$ در تناقض است.
 \item
 اگر $2r < d$ باشد، دایره کمانی به طول $d$ دارد که این با قضیه‌ی «بزرگترین کمان دایره همان قطر است» در تناقض است.
\end{itemize}
پس این دو حالت رد شد و پس $d = 2r$ است.
\\
حال از این صفحات دایره‌هایی با اشتراک یک خط داریم که اجتماع همه‌ی این دایره‌ها طبق یک قضیه‌ی هندسی کره است. این کره را $C$ و مرکز آن‌را $O$  و شعاع آن‌را $R$ در نظر می‌گیریم.
این کره بخشی از جسم ماست، حال فرض کنید جسم ما برابر با این کره نباشد و نقره‌ای مانند $X$ از جسم وجود داشته باشد که داخل کره نباشد یعنی $OX > R$
حال $OX$ را از طرف $O$ ادامه می‌دهیم تا کره را در نقطه‌ی دیگری همچون $Y$ قطع کند، می‌دانیم
$OY = R$
و 
$OY+OX = XY$
پس داریم
$XY > 2R = d \rightarrow XY > d$
\\
حال خط گذرنده از $XY$ را در نظر بگیرید، این چون دارای ۲ نقطه است طبق لم باید پاره خطی از جسم بگذراند که $X$ و $Y$ باید جزئی از این پاره ‌خط باشند، پس باید طول این پاره‌خط حداقل $XY$ باشد که $XY > d$ که با بیشینه بودن $d$ در تناقض است.
پس این حالت نیز رخ نمی‌دهد و جسم ما تنها شامل کره‌ی $C$ است پس جسم ما یک کره است و حکم ثابت شد.
\end{document}



