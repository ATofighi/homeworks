\documentclass[12pt,a4paper]{article}
\usepackage{multirow}
\usepackage{rotating}
\usepackage{graphicx}
\usepackage{amsmath}
\usepackage{amsfonts}
\usepackage{amssymb}
\usepackage{graphicx}
\pagestyle{empty}
\usepackage[bottom=0.5in,headheight=0pt,headsep=0pt]{geometry}
\addtolength{\topmargin}{0pt}
\usepackage{xepersian}
\linespread{1.2}
\settextfont{XB Niloofar}

\begin{document}
\begin{center}
	بسمه تعالی
\end{center}
\begin{center}
	\textbf{پاسخ تمرین تحویلی دوم - درس ریاضی‌نویسی - دانشگاه صنعتی شریف}
	\\
	علیرضا توفیقی محمدی - رشته علوم کامپیوتر - شماره‌دانشجویی: ۹۶۱۰۰۳۶۳
\end{center}
\section*{پرسش}
هتلی داریم که دو گروه جاندار در آن زندگی می‌کنند، انسان و خون‌آشام؛ هر کدام می‌توانند سالم یا دیوانه باشند.
می‌دانیم انسان همیشه راست می‌گوید، خون‌آشام همیشه دروغ می‌گوید. جاندار سالم باور درست به مسائل دارد و جاندار دیوانه باور غلط به مسائل دارد.
\\
بر این اساس یک انسان سالم و خون‌آشام دیوانه واقعیت را می‌گویند و انسان دیوانه و خون‌آشام سالم خلاف واقعیت را می‌گویند.
\subsection*{قسمت نخست}
لوسی و مینا دو خواهر هستند که با آن‌ها صحبت می‌کنیم، می‌دانیم یکی انسان و دیگری خون‌آشام است.
\begin{itemize}
	\item \textbf{لوسی:} ما هر دو دیوانه‌ایم.
	\item \textbf{مینا:} این حرف درست نیست.
\end{itemize}
کدام یک انسان و کدام‌یک دیوانه‌است؟
\subsection*{پاسخ قسمت نخست}
%\begin{enumerate}
%	\item صورت مسئله $\leftarrow$ انسان سالم و خون‌آشام دیوانه واقعیت را می‌گویند و خون‌آشام سالم و انسان دیوانه خلاف واقعیت را.
%	\item صورت مسئله $\leftarrow$ لوسی گفته‌است که ما هر دو دیوانه‌ایم
%	\item صورت مسئله $\leftarrow$ مینا گفته است که حرف لوسی درست نیست.
%	\item ۲،۳ $\leftarrow$ مینا گفته‌است که «حداقل یکی‌شان دیوانه نیست»
%	\item صورت مسئله $\leftarrow$ انسان راست می‌گوید.
%	\item صورت مسئله $\leftarrow$ خون آشام دروغ می‌گوید.
%		\item صورت مسئله $\leftarrow$ جاندار سالم باور درست دارد.
%	\item صورت مسئله $\leftarrow$ جاندار دیوانه باور غلط دارد.
%	\item لوسی ۴ حالت دارد:
%	\begin{enumerate}
%		\item \textbf{حالت ۱:}
%		 اگر لوسی انسان سالم باشد:
%		\item ۱، آ $\leftarrow$ لوسی واقعیت را گفته‌است.
%		\item ۲، ب $\leftarrow$ هردو دیوانه اند.
%		\item ج $\leftarrow$ لوسی دیوانه است.
%		\item آ، د $\leftarrow$ تناقض و این حالت رخ نمی‌تواند بدهد.
%	\end{enumerate}
%	\begin{enumerate}
%		\item \textbf{حالت ۲:}
%		اگر لوسی انسان دیوانه باشد:
%		\item ۱، آ $\leftarrow$ لوسی غیرواقع گفته‌است.
%		\item ۲، ب $\leftarrow$ حداقل یک نفر دیوانه نیست
%		\item آ، ج $\leftarrow$ مینا سالم است.
%		\item ج، ۴ $\leftarrow$ مینا واقعیت را گفته‌است.
%		\item د، ه $\leftarrow$ مینا راست گفته‌است.
%		\item ۵، ۶ $\leftarrow$ مینا انسان است.
%		\item 
%	\end{enumerate}

%\end{enumerate}
...
\end{document}