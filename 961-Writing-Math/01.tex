\documentclass[12pt,a4paper]{article}
\usepackage{multirow}
\usepackage{rotating}
\usepackage{graphicx}
\usepackage{amsmath}
\usepackage{amsfonts}
\usepackage{amssymb}
\usepackage{graphicx}
\pagestyle{empty}
\usepackage[bottom=0.5in,headheight=0pt,headsep=0pt]{geometry}
\addtolength{\topmargin}{0pt}
\usepackage{xepersian}
\linespread{1.2}
\settextfont{XB Niloofar}

\begin{document}
\begin{center}
	بسمه تعالی
\end{center}
\begin{center}
	\textbf{پاسخ تمرین تحویلی نخست - درس ریاضی‌نویسی - دانشگاه صنعتی شریف}
	\\
	علیرضا توفیقی محمدی - رشته علوم کامپیوتر - شماره‌دانشجویی: ۹۶۱۰۰۳۶۳
\end{center}
\section*{پرسش}
سه اتاق داریم که در یکی دختری زیبا و در دوتای دیگر ببری گرسنه قرار دارد. روی سردر اتاق‌ها به ترتیب گزاره‌های زیر نوشته شده‌است:
\begin{itemize}
	\item \textbf{اتاق۱:}
		یک ببر گرسنه در این اتاق است.
	\item \textbf{اتاق ۲:}
	یک دختر زیبا در این اتاق است.
	\item \textbf{اتاق ۳:}
	یک ببر گرسنه در اتاق ۲ است.
\end{itemize}
اگر بدانیم حداکثر یکی از گزاره‌ها درست است، در کدام اتاق دختری زیبا قرار دارد؟
\section*{پاسخ}
\subsection*{لم}
\textbf{دقیقا گزاره‌ی یکی از دو اتاق ۲ و ۳ درست است.}
\\
گزاره‌ی اتاق ۲ دو حالت دارد:
\begin{enumerate}
\item
اگر درست باشد، آنگاه در اتاق ۲ دختری زیبا‌ست پس در آن ببری گرسنه نیست و گزاره‌ی اتاق ۳ غلط می‌شود.
\item
اگر غلط باشد، آنگاه در آن یک دختر زیبا نیست، پس در آن یک ببر گرسنه است و گزاره‌ی اتاق ۳ درست خواهد بود.
\end{enumerate}
همان‌طور که مشاهده شد در حالت نخست گزاره‌ی اتاق ۲ درست و ۳ غلط و در حالت دوم گزاره‌ی اتاق ۲ غلط و ۳ درست بود پس در هر حالت دقیقا یکی از این دو گزاره درست بودند و لم ثابت شد.
\subsection*{راه حل مسئله}
طبق لم، گزاره‌ی یکی از دو اتاق ۲ و ۳ درست است، همچنین می‌دانیم طبق فرض مسئله حداکثر گزاره‌ی یکی از سه اتاق درست است پس گزاره‌ی اتاق ۱ قطعا غلط است و در آن ببری گرسنه قرار نگرفته، پس باید در آن دختری زیبا باشد.
\\
همچنین چون طبق فرض مسئله در یکی از ۳ اتاق دختری زیبا بود و در دوتای دیگر ببری گرسنه و در بند بالا ثابت کردیم در اتاق ۱ دختری زیباست پس در اتاق ۲ و ۳ ببری گرسنه قرار گرفته، پس در اتاق ۲ یک دختر زیبا نیست و گزاره‌ی اتاق ۲ غلط و گزاره‌ی اتاق ۳ صحیح است.
\\
پس اتاق ۱ را انتخاب می‌کنیم.
\end{document}