\documentclass[12pt,a4paper]{article}
\usepackage{multirow}
\usepackage{rotating}
\usepackage{graphicx}
\usepackage{amsmath}
\usepackage{amsfonts}
\usepackage{amssymb}
\usepackage{graphicx}
\pagestyle{empty}
\usepackage[bottom=0.5in,headheight=0pt,headsep=0pt]{geometry}
\addtolength{\topmargin}{0pt}
\usepackage{xepersian}
\linespread{1.2}
\settextfont{XB Niloofar}

\begin{document}
\begin{center}
	بسمه تعالی
\end{center}
\begin{center}
	\textbf{پاسخ تمرین تحویلی سوم - درس ریاضی‌نویسی - دانشگاه صنعتی شریف}
	\\
	علیرضا توفیقی محمدی - رشته علوم کامپیوتر - شماره‌دانشجویی: ۹۶۱۰۰۳۶۳
\end{center}
\section*{پرسش}
یک جدول 
$n \times n$
داریم که در هر خانه از آن عددی طبیعی قرار گرفته، در هر مرحله می‌توانیم یک سطر را انتخاب‌کرده و همه‌ی اعداد آن را دوبرابر کرده یا یک ستون را انتخاب کرده و از همه‌ی اعداد آن یک واحد کم کنیم.
\\
ثابت کنید می‌توان با انجام تعدادی مرحله همه‌ی خانه‌های جدول را ۰ کرد.
\section*{پاسخ}
برای حل سوال ابتدا الگوریتمی را ارائه می‌دهیم که همه‌ی خانه‌های یک جدول 
$n \times 1$
که با اعداد طبیعی پر شده‌است را ۰ کند؛ سپس به کمک آن کل جدول را ۰ می‌کنیم.
\\
چنین الگوریتمی را در ادامه \textit{ستون صفر کن} می‌نامیم.

\subsubsection*{الگوریتم ستون صفر کن:}
فرض کنید یک جدول $n \times 1$ از اعداد طبیعی داریم، $n$ بار فرایند زیر را روی آن انجام می‌دهیم:
\\
در $i$ امین فرایند، تا زمانی که مقدار خانه‌ی $i$ ام برابر با ۱ نشده، همه‌ی سطر‌ها به غیر از سطر $i$ ام را انتخاب کرده و دوبرابر می‌کنیم و سپس ستون را در نظر گرفته و مقدار همه‌ی خانه‌های آن را منهای یک می‌کنیم.
(چون هر دفعه یک واحد از مقدار خانه‌ی $i$ ام کم می‌شود و در ابتدا طبیعی بود، پس این فرایند پایان پذیر است.)


\textbf{ادعا:}
بعد از انجام $i$ امین فرایند، همه‌ی خانه‌های ۱ تا $i$ برابر با ۱ و خانه‌های $i+1$ تا $n$ عددی طبیعی اند.
\\
\textbf{اثبات:}
برای اثبات از استقرا استفاده می‌کنیم، بعد از انجام اولین فرایند، خانه‌ی اول برابر با ۱ است زیرا در غیر اینصورت بازهم باید عمل داخل فرایند یکم را تکرار می‌کردیم.
\\
فرض کنید قبل از اجرای فرایند $i$ ام (در پایان فرایند $i-1$ ام) همه‌ی خانه‌های ۱ تا $i-1$ برابر با یک و بقیه‌ی خانه‌ها طبیعی باشند، در هر دفعه تمام خانه‌ها غیر از $i$ ام را دوبرابر می‌کنیم و مقدار کل خانه‌ها را منهای یک می‌کنیم، در این عمل خانه‌ی ۱ تا $i-1$ ابتدا دوبرابر شده و برابر با ۲ می‌شوند و سپس یک واحد از آن‌ها کم شده و برابر با ۱ می‌شوند، همچنین خانه‌های $i+1$ ام تا $n$ ام طبیعی بوده‌اند، پس دوبرابر آنها حداقل ۲ است و منهای یک آنها بازهم طبیعی است. پس در پایان هر دفعه اجرای فرایند $i$ ام مقادیر خانه‌های ۱ تا $i-1$ برابر با یک و $i+1$ تا $n$ طبیعی باقی می‌مانند. همچنین شرط پایان تکرار فرایند $i$ام برابر با ۱ شدن خانه‌ی $i$ام است، پس خانه‌ی $i$ام نیز برابر با یک می‌شود و گام استقرا ثابت و ادعای ما ثابت شد.
\\
\\
پس بعد از انجام فرایند $n$ ام همه‌ی خانه‌ها برابر با ۱ هستند.
\\
حال اگر این ستون را منهای یک کنیم، همه‌ی خانه‌ها ۰ می‌شوند و الگوریتم \textit{ستون صفر کن} ساخته شد.
\subsubsection*{لم:}
\textbf{
	اگر جدولی $n\times n$ از اعداد حسابی داشته باشیم که ستونی از اعداد طبیعی داشته باشد و الگوریتم ستون صفر کن را برای خانه‌های آن ستون انجام دهیم، به جدولی می‌رسیم که اعداد آن ستون صفر بوده و دیگری اعداد طبیعی جدول طبیعی باقی مانده و اعداد ۰ جدول نیز ۰ باقی می‌مانند.
}
\\
\textbf{اثبات:}
اثبات به سادگی امکان پذیر است، طبق الگوریتم ستون جمع کن، همه‌ی خانه‌های آن ستون ۰ می‌شوند، همچنین چون فقط عمل منهای یک برای آن ستون انجام می‌شود تنها عملی که روی خانه‌های دیگر ستون‌ها انجام می‌شود ضربدر ۲ است، که ضربدر ۲، اعداد طبیعی را طبیعی و اعداد ۰ را ۰ نگه می‌دارد.
\\
\subsubsection*{حل مسئله}
جدول 
$n \times n$
از اعداد طبیعی داریم، حال فرایند زیر را $n$ بار انجام می‌دهیم.
\\
در مرحله‌ی $i$ ام برای ستون $i$ ام الگوریتم \textit{ستون صفر کن} را اجرا می‌کنیم.
\\
طبق لم اعداد ستون‌های $i+1$ ام تا $n$ ام طبیعی باقی می‌ماند و اعداد ستون‌های نخست تا $i-1$ صفر باقی می‌ماند و اعداد ستون $i$ ام هم که در الگوریتم ستون صفر کن، صفر شدند پس در پایان فرایند $i$ ام خانه‌های ستون‌های ۱ تا $i$ برابر با ۰ می‌شوند پس در پایان فرایند $n$ ام تمام خانه‌های جدول ۰ می‌شوند و حکم ثابت می‌شود.
\end{document}