\documentclass[12pt,a4paper]{article}
\usepackage{multirow}
\usepackage{rotating}
\usepackage{graphicx}
\usepackage{amsmath}
\usepackage{amsfonts}
\usepackage{amssymb}
\usepackage{graphicx}
\pagestyle{empty}
\usepackage[bottom=0.5in,headheight=0pt,headsep=0pt]{geometry}
\addtolength{\topmargin}{0pt}
\usepackage{xepersian}
\linespread{1.2}
\settextfont{XB Niloofar}

\begin{document}
\begin{center}
	بسمه تعالی
\end{center}
\begin{center}
	\textbf{پاسخ تمرین تحویلی ششم - درس ریاضی‌نویسی - دانشگاه صنعتی شریف}
	\\
	علیرضا توفیقی محمدی - رشته علوم کامپیوتر - شماره‌دانشجویی: ۹۶۱۰۰۳۶۳
\end{center}
\section*{پرسش}
۱۳۹۶ آب‌نبات قرمز و ۲۰۱۷ آب‌نبات آبی داریم، دو رویکرد داریم:
\begin{enumerate}
	\item یک آب‌نبات از کیسه درآورده و می‌خوریم و به ۲ می‌رویم.
	\item
	یک آب‌نبات از کیسه در می‌آوریم.
	\begin{itemize}
		\item اگر همرنگ قبلی بود می‌خوریم و دوباره رویکرد ۲ را اجرا می‌کنیم.
		\item اگر همرنگ قبلی نبود، به کیسه برمی‌گردانیم و رویکرد ۱ را اجرا می‌کنیم.
	\end{itemize}
\end{enumerate}
در اولین گام رویکرد ۱ را اجرا کرده و تا زمانی که شکلاتی مانده‌است ادامه می‌دهیم. احتمال اینکه آخرین شکلات خورده شده قرمز باشد چقدر است؟
\section*{پاسخ}
به جای ارائه‌ی احتمال برای ۱۳۹۶ و ۲۰۱۷ برای $n$ شکلات آبی و $m$ شکلات قرمز که $n, m \geq 1$ اند مسئله را حل می‌کنیم.
\\
همچنین گاها به جای آب‌نبات از واژه‌های شکلات یا مهره استفاده شده که منظور همان آب‌نبات است!
\\
توابع $f, r, b$ را به صورت روبرو تعریف می‌کنیم:
\begin{itemize}
	\item $f(n, m)$ احتمال اینکه با $n$ شکلات آبی و $m$ شکلات قرمز رویکرد ۱ را اجرا کرده و آخرین شکلات خورده شده قرمز شود.
	\item $r(n, m)$
	احتمال اینکه در حالتی با $n$ شکلات آبی و $m$ شکلات قرمز رویکرد ۲ را اجرا کنیم و فرضمان بر این باشد که شکلات قبلی قرمز بود و آخرین شکلات خورده شده قرمز شود.
	\item $b(n, m)$
	احتمال اینکه در حالتی با $n$ شکلات آبی و $m$ شکلات قرمز رویکرد ۲ را اجرا کنیم و فرضمان بر این باشد که شکلات قبلی آبی بود و آخرین شکلات خورده شده قرمز شود.
\end{itemize}
با توجه به تعاریف داده شده می‌توان برای $f$ روابط زیر را گفت:
$$ f(n, 0) = 0, (n \neq 0)$$
$$ f(0, n) = 1, (n \neq 0)$$
$$ f(n, m) = \frac{n}{n+m} b(n-1, m) + \frac{m}{n+m} g(n, m-1), (n,m > 0)$$
که رابطه‌ی اول در حالتی است که هیچ قرمزی نداریم و بدیهتا جواب ۰ است، حالت دوم حالتی است که فقط قرمز داریم و بدیهتا جواب ۱ است و حالت سوم حالتی است که هم قرمز و هم آبی داریم، اگر اولین برداشتنمان آبی باشد به احتمال $\frac{n}{n+m}$ آبی را برداشته ایم و احتمال اینکه آخرین آب‌نبات در ادامه‌ی مسیر قرمز شود برابر با $b(n-1, m)$ است پس احتمال این حالت برابر با 
$\frac{n}{n+m} b(n-1, m)$
است. به طور مشابه مقدار 
$\frac{m}{n+m} g(n, m-1)$
برای اولین شکلات قرمز برداشتن نیز به دست می‌آید.
\\
همچنین تابع $r$ به صورت زیر تعریف می‌شود:
$$r(n, 0) = 0, (n \neq 0)$$
$$r(0, n) = 1, (n \neq 0)$$
$$r(n, m) = \frac{n}{n+m} f(n, m) + \frac{m}{n+m} r(n, m-1), (n,m > 0)$$
که دو پایه برای حالتی که هیچ آبی‌ای یا هیچ قرمزی نداریم اند و حالت سوم دو حالت داریم، یا مهره‌ای آبی بر می‌دارد که در این صورت مهره را برگردانده و به رویکرد یک می‌رود که احتمال آن برابر با $\frac{n}{n+m} f(n, m)$ و یا مهره‌ای قرمز برداشته که آن‌را می‌خورد و باز به رویکرد ۲ می‌شود که در این صورت احتمال آن برابر با 
$\frac{m}{n+m} r(n, m-1)$
است که چون «یا» است پس احتمال برابر با مجموع این دو احتمال است.
\\
به طریق مشابه برای تابع $b$ داریم:
$$b(n, 0) = 0, (n \neq 0)$$
$$b(0, n) = 1, (n \neq 0)$$
$$b(n, m) = \frac{m}{n+m} f(n, m) + \frac{n}{n+m} b(n-1, m), (n,m > 0)$$
\\
\\
ابتدا با توجه به فرمول‌های به‌دست آمده می‌توانیم برای $n,m \leq 2$ مقادیر $f, r, b$ را محاسبه کنیم که به شرح زیر است:
$$
f(0, 1) = f(0, 2) = 1
$$
$$
f(1, 0) = f(1, 0) = 0
$$
$$
f(1, 1) = f(1, 2) = f(2, 1) = \frac{1}{2}
$$
$$
r(0, 1) = r(0, 2) = 1
$$
$$
r(1, 0) = r(1, 0) = 0
$$
$$
r(1, 1) = \frac{1}{4}, r(2, 1) = \frac{2}{6}
$$
$$
r(1, 2) = \frac{2}{6}, r(2, 2) = \frac{5}{12}
$$
$$
b(0, 1) = b(0, 2) = 1
$$
$$
b(1, 0) = b(1, 0) = 0
$$
$$
b(1, 1) = \frac{3}{4}, b(2, 1) = \frac{4}{6}
$$
$$
b(1, 2) = \frac{4}{6}, b(2, 2) = \frac{7}{12}
$$
\\
حال به حل سوال می‌پردازیم:
\subsubsection*{لم ۱: 
$b(n, m) = 1 - r(m, n)$
}
برای اثبات به تعریف $r$ و $b$ بر می‌گردیم، در تعریف $b$ گفتیم فرض ما این است که قبلا یک مهره که تعداد آنها پارامتر اول است خورده بودیم و حال می‌خواهیم احتمال اینکه آخرین مهره هم از نوع پارامتر دوم شود حساب کنیم، این احتمال برابر با ۱ منهای احتمال اینکه آخرین مهره از نوع پارامتر اول شود است، حال اگر قرمز و آبی را جابه جا ببینیم و تعداد آنها را نیز جا به جا کنیم، جواب هیچ تغییری نمی‌کند ولی احتمال برابر با ۱ منهای احتمال اینکه قبلا مهره از پارامتر دوم با تعداد پارامتر اول قدیم خورده بودیم و آخرین مهره پارامتر دوم شود است که این به معنی عبارت بالاست. پس لم ثابت شد.
\\
\subsubsection*{حل مسئله}
پس طبق لم برای $n, m > 0$ داریم:
$$f(n, m) = \frac{n}{n+m}(1-r(m, n-1)) + \frac{m}{n+m}r(n, m-1)$$
حال با استقرا بر روی ‌$n+m$ ثابت می‌کنیم برای هر $n, m$ طبیعی 
($n, m > 0$)
دو حکم زیر را داریم:
$$
f(n, m) = \frac{1}{2}
$$
$$
r(n, m) = \frac{\binom{n+m}{m}-1}{2\times \binom{n+m}{m}}
$$
که با ساده‌سازی $r(n, m)$ به عبارت زیر می‌رسیم:
$$
r(n, m) = \frac{(n+m)! - n!m!}{2\times(n+m)!}
$$
\\
\textbf{پایه‌ی استقرا:}
حکم برای $n+m = 2$ برای $r(1, 1)$ و $f(1, 1)$ طبق محاسبات صفحه‌ی دوم درست است.
\\
\textbf{فرض استقرا:}
برای همه‌ی $n+m < k$ ها برابری‌های زیر صدق می‌کنند:
$$
f(n, m) = \frac{1}{2}
$$
$$
r(n, m) = \frac{\binom{n+m}{m}-1}{2\times \binom{n+m}{m}}
$$
\\
\textbf{حکم استقرا:}
حال باید ثابت کنیم با توجه به فرض استقرا حکم بالا برای 
$n+m = k$
نیز برقرار است.
\\
برای این‌کار حالت بندی زیر را انجام می‌دهیم:
\begin{enumerate}
	\item \textbf{اگر $n = 1$ باشد:}
	\\
	در این صورت 
$$f(1, m) = \frac{1}{1+m}(1-r(m, 0)) + \frac{m}{1+m}r(1, m-1)$$
حال طبق فرض استقرا و مقداردهی پایه‌ی $r$:
$$\rightarrow
f(1, m) = \frac{1}{1+m} + 
\frac{m}{1+m} \times \frac{(m)! - (m-1)!}{2\times m!}
$$
با مخرج مشترک گیری داریم:
$$\rightarrow
f(1, m) = \frac{
	2 \times m! + m \times (m! - (m-1)!)
}{
	2 \times (m+1)!
}
$$
$$\rightarrow
f(1, m) = \frac{
	2 \times m! + m \times m! - m!
}{
	2 \times (m+1)!
}
$$
با فاکتورگیری از $m!$ داریم:
$$\rightarrow
f(1, m) = \frac{
	m!(2 + m - 1)
}{
	2 \times (m+1)!
}
$$
$$\rightarrow
f(1, m) = \frac{
	(m+1)!
}{
	2 \times (m+1)!
}
$$
که این برابر است با

$$\rightarrow
f(1, m) = \frac{1}{2}
$$
که $f$ به‌درستی حساب شد و به سراغ $r$ می‌رویم:
$$
r(1, m) = \frac{1}{1+m} f(1, m) + \frac{m}{1+m} r(1, m-1)
$$
که طبق اثبات بالا و فرض استقرا داریم:
$$
\rightarrow r(1, m) = \frac{
	m! + m\times m! - m \times (m-1)!
}{
	2\times (1+m) \times m!
}
$$
$$
\rightarrow r(1, m) = \frac{
	m! + m!\times (m-1)
}{
	2\times (m+1)!
}
$$
با اضافه و کم کردن $m!$ در صورت داریم:

$$
\rightarrow r(1, m) = \frac{
	m! + m!\times (m-1) + m! - m!
}{
	2\times (m+1)!
}
$$
پس با فاکتور گیری $m!$های مثبت در صورت داریم:
$$
\rightarrow r(1, m) = \frac{
	(m+1)! - m!
}{
	2\times (m+1)!
}
$$
که حکم برای $r$ نیز ثابت شد.
\item
\textbf{اگر $m=1$ باشد.}
پس داریم:
$$f(n, 1) = \frac{n}{n+1}(1-r(1, n-1)) + \frac{1}{n+1}r(n, 0)$$
که طبق پایه‌ی $r$ و فرض استقرا داریم:
$$
\rightarrow f(n, 1) = \frac{
	n \times 2 \times n! - n \times (n!-(n-1)!)
}{
(n+1) \times 2 \times n!
}
$$
$$
\rightarrow f(n, 1) = \frac{
	n \times 2 \times n! - n! \times (n-1)
}{
	2 \times (n+1)!
}
$$
$$
\rightarrow f(n, 1) = \frac{
	n! \times (2\times n - n + 1)
}{
	2 \times (n+1)!
}
= \frac{(n+1)!}{2 (n+1)!}
$$
$$ \rightarrow f(n, 1) = \frac{1}{2}$$
که حکم برای $f$ ثابت شد. حال به سراغ $r$ می‌رویم، طبق تعریف داریم:
$$
r(n, 1) = \frac{n}{n+1} f(n, 1) + \frac{1}{n+1} r(n, 0)
$$
که طبق پایه و حکم بالا داریم:
$$
\rightarrow r(n, 1) = \frac{n}{2\times (n+1)} (*)
$$
از طرفی داریم:
$$
\frac{(n+1)! - n!}{2\times(n+1)!} = 
\frac{n!(n+1 - 1)}{2\times(n+1)!} =
\frac{n}{2\times(n+1)} = r(n, 1)
$$
که حکم برای $r$ نیز ثابت شد.
\item 
\textbf{اگر $n, m > 1$ باشند:}
\\
در این صورت برای $f$ داریم:
$$
f(n, m) = \frac{n}{n+m}(1-r(m, n-1)) + \frac{m}{n+m}r(n, m-1)
$$
طبق فرض استقرا برای $r$ داریم:
$$
\rightarrow f(n, m) = \frac{n}{
	n+m
}(1-\frac{(n+m-1)! - (n-1)!m!}{2\times(n+m-1)!}) + \frac{m}{n+m} \frac{(n+m-1)! - n!(m-1)!}{2\times(n+m-1)!}
$$
با مخرج مشترک گیری داریم:
$$
\rightarrow f(n, m) = \frac{
	n \times 2 \times (n+m-1)! - n \times (n+m-1)! + n!m!
	+ m \times (n+m-1)! - n!m!
}{
	(n+m) \times 2 \times (n+m-1)!
}
$$
$$
\rightarrow f(n, m) = \frac{
	n \times 2 \times (n+m-1)! - n \times (n+m-1)!
	+ m \times (n+m-1)!
}{
	2 \times (n+m)!
}
$$
با فاکتورگیری از $(n+m-1)!$ در صورت داریم:

$$
\rightarrow f(n, m) = \frac{
	(n+m-1)! \times (2n - n + m)
}{
	2 \times (n+m)!
} = \frac{
(n+m-1)! \times (n + m)
}{
2 \times (n+m)!
} = \frac{
(n+m)!
}{
2 \times (n+m)!
}
$$
پس
$$f(n, m) = \frac{1}{2}$$
حال برای $r$ داریم:
$$r(n, m) = \frac{n}{n+m} f(n, m) + \frac{m}{n+m} r(n, m-1)$$
طبق اثبات بالا و فرض استقرا داریم:
$$
\rightarrow r(n, m) = \frac{n}{n+m} \times \frac{1}{2} +
 \frac{m}{n+m} \frac{(n+m-1)!-n!(m-1)!}{2(n+m-1)!}$$
 حال با مخرج مشترک گیری داریم:
$$
\rightarrow r(n, m) = \frac{
 	n \times (n+m-1)! + m\times (n+m-1)! - n!m!
}{
	2 \times (n+m)!
}
$$
حال با فاکتورگیری از $(n+m-1)!$ داریم:
$$
\rightarrow r(n, m) = \frac{
	(n+m-1)!\times(n+m) - n!m!
}{
	2 \times (n+m)!
}
$$
$$
\rightarrow r(n, m) = \frac{
	(n+m)! - n!m!
}{
	2 \times (n+m)!
}
$$
و حکم برای $r$ در این حالت نیز ثابت شد.
\end{enumerate}
پس حکم استقرا برای هر ۳ حالت ثابت شد و حکم ثابت شد.
پس $f(n, m) = \frac{1}{2} (n, m > 0)$ پس برای $n = 2017 , m = 1396$ نیز حکم برقرار است و $f(2017, 1396) = 0.5$ پس احتمال قرمز آمدن  در آخرین مرحله برابر با $\frac{1}{2}$ است.
\end{document}



