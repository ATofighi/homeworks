\documentclass[12pt,a4paper]{article}
\usepackage{multirow}
\usepackage{rotating}
\usepackage{graphicx}
\usepackage{amsmath}
\usepackage{amsfonts}
\usepackage{amssymb}
\usepackage{graphicx}
%\pagestyle{empty}
%\usepackage[bottom=0.5in,headheight=0pt,headsep=0pt]{geometry}
%\addtolength{\topmargin}{0pt}
\DeclareMathOperator{\rank}{rank}
\DeclareMathOperator{\im}{Im}
\newcommand{\dori}[1]{{\langle\langle {#1} \rangle\rangle_{T}}}
\usepackage{xepersian}
\linespread{1.2}
\settextfont{XB Niloofar}
\setdigitfont{XB Niloofar}

\begin{document}
\begin{center}
	بسمه تعالی
\end{center}

\begin{center}
	\textbf{
		پاسخ سری هفتم تمرین‌ها
		- درس جبرخطی ۱ - دانشگاه صنعتی شریف}
	\\
	علیرضا توفیقی محمدی - رشته علوم کامپیوتر - شماره‌ی دانشجویی: ۹۶۱۰۰۳۶۳
\end{center}
\section{تمرین 19 از بخش هشتم}
\subsection{7}
ابتدا ثابت می‌کنیم
$\ker p(T)$
زیر فضایی ناوردا از $T$ است. برای اینکار عضو دلخواه $v \in \ker p(T)$ در نظر بگیرید.
$$
p(T)(T(v)) = T(p(T)(v)) = T(0) = 0 \implies T(v) \in \ker p(T)
$$
و قسمت اول ثابت شد.

حال چون 
$m_{T_{\ker p(T)}} | m_T$
پس 
$m_{T_{\ker p(T)}} = p(x)^k, k \leq b$

چون 
$T(\ker p(T)) \subseteq \ker p(T)$ 
پس
$p(T_{\ker p(T)}) = 0$
و در نتیجه
$m_{T_{\ker p(T)}} | p \implies k \leq 1$
از طرفی $m_{T_{\ker p(T)}}$ نمی‌تواند اسکالر باشد پس $k=1$ و 
$m_{T_{\ker p(T)}}(x) = p(x)$
\subsection{8}
اگر $W_1, ..., W_k$ مستقل خطی باشند،
$$\forall w_i \in W_i: w_1 + .... + w_k = 0 \implies w_1 = ... = w_k = 0$$
پس اگر عضو‌های 
$v_i \in W_i \cap \ker p(T)$
در نظر بگیریم که
$v_1 + ... + v_k = 0$
چون $v_i \in W_i$ است پس
$v_1 = ... = v_k = 0$
و در نتیجه 
$W_1 \cap \ker p(T), ..., W_k \cap \ker p(T)$
مستقل خطی است.

حال به طرف دیگر قضیه می‌پردازیم، با استقرا روی $k$ تعداد زیرفضاهای $T$-دوری تلاش به اثبات این سمت می‌کنیم.

حکم: اگر $T$ عملگری روی فضای برداری $V$ باشد که $m_T(x) = p(x)^b$ و $p$ یک چندجمله‌ای اول باشد و $W_1, ..., W_k$ زیرفضاهای $T$-دوری در $V$ باشند، آنگاه اگر ‌$W_1 \cap \ker p(T), ..., W_k \cap \ker p(T)$ مستقل خطی باشد نتیجه می‌شود $W_1, ..., W_k$ مستقل خطی است.

پایه: حکم برای $k = 1$ واضخ است زیرا هر زیرفضا به شکل تنها یک مجموعه‌ی مستقل خطی از زیرفضاها را تشکیل می‌دهد.

فرض کنید حکم برای $k = n-1$ برقرار است و قرار می‌دهیم $n=k$، چون $W_i$ ها $T$-دوری اند و $m_T(x) = p(x)^b$ پس $m_{W_i}(x) = p(x)^{q_i}$ است،
$\max{q_1, ..., q_k} = Q$
در نظر بگیرید و بدون خدشه به کلیت مسئله فرض کنید $q_1 = Q$ است، چون $p^{q_i}$ چندجمله‌ای مینیمال $W_i$ است پس 
$p(T)(W_i)^{q_i} = 0, p(T)(W_i)^{q_i-1}\neq 0$
. حال بردارهای $w_1, ..., w_k$ را در نظر بگیرید که $w_i \in W_i$ و $w_1 + ... + w_k = 0$ باشد.

در این صورت چون $W_i$ $T$-دوری است، 
$p(T)(W_i)^{Q-1} \subseteq W_i$
 و چون $p(T)(W_i)^Q = 0$ است پس 
$\im p(T)(W_i)^{Q-1} \subseteq \ker p(T)$
پس اگر بردارهای 
$u_1 = p(T)(w_1)^{Q-1}, ..., u_k = p(T)(w_k)^{Q-1}$
را در نظر بگیریم داریم $u_1 + ... + u_k = 0$ و در نتیجه $u_1 = ... = u_k = 0$ پس 
$$u_1 = p(T)(w_1)^{q_1 - 1} = 0 \implies w_1 = 0$$
پس $w_2 + ... + w_k = 0$ و طبق فرض استقرا $W_2, ..., W_k$ نیز مستقل خطی اند پس 
$$w_2 = ... = w_k = 0$$
پس $W_1, ..., W_k$ مستقل خطی اند.
\subsection{9}
می‌دانیم 
$m_{T_\dori{v}}|m_T$
 پس $a \leq b$ وجود دارد که 
$m_{T_\dori{v}} = p(x)^a$

و به طریق مشابه 
$m_{T_\dori{p(T)(v)}} = p(x)^b$
همچنین در کلاس ثابت شد (و همچنین تمرین ۱ همین قسمت) که بعد یک فضای $T$-دوری برابر با درجه‌ی چند جمله‌ای مینیمال آن است پس داریم:
$$\dim \dori{v} = a \times \deg p$$
و
$$\dim \dori{p(T)(v)} = b \times \deg p$$
پس تنها کافی‌ست ثابت کنیم
$$
b\times \deg p + \deg p = a \times \deg b \iff b+1 = a
$$
که این حکم نیز نسبتا واضح است، زیرا 
$$
\forall u \in \dori{v}: p^{b+1}(T)(u) = p^b(p(T)(u)) = 0 \implies a \leq b+1
$$
$$
\forall u \in \dori{v}: p^b(T)(u) = p^{b-1}(p(T)(u)) \neq 0 \implies a \geq b+1
$$
پس $a = b+1$ و حکم ثابت شد.
\subsection{10}
حل نشد :(
\end{document} 