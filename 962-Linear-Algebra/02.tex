\documentclass[12pt,a4paper]{article}
\usepackage{multirow}
\usepackage{rotating}
\usepackage{graphicx}
\usepackage{amsmath}
\usepackage{amsfonts}
\usepackage{amssymb}
\usepackage{graphicx}
\pagestyle{empty}
%\usepackage[bottom=0.5in,headheight=0pt,headsep=0pt]{geometry}
%\addtolength{\topmargin}{0pt}
\usepackage{xepersian}
\linespread{1.2}
\settextfont{XB Niloofar}

\begin{document}
\begin{center}
	بسمه تعالی
\end{center}
\begin{center}
	\textbf{
		پاسخ سری دوم تمرین‌ها
		-
		- درس جبرخطی ۱ - دانشگاه صنعتی شریف}
	\\
	علیرضا توفیقی محمدی - رشته علوم کامپیوتر - شماره‌دانشجویی: ۹۶۱۰۰۳۶۳
\end{center}
\section{سوال ۴}
چون 
$<v_1, ..., v_{n+2}>$
زیرفضای از $R^n$ است، پس بعد آن حداکثر $n$ است پس حداقل ۲ تا از $v_i$ها زائد بوده و می‌توان به صورت ترکیب خطی‌ای از بقیه‌ی عناصر نوشت، فرض کنید این دو عضو 
$v_{n+1}$
 و 
$v_{n+2}$
اند،پس داریم:
\[
v_{n+1} = \sum_{i=1}^n t_i v_i
\rightarrow A = v_{n+1} - \sum_{i=1}^n t_i v_i = 0 \hspace*{1cm} (1)
\]
\[
v_{n+2} = \sum_{i=1}^n s_i v_i
\rightarrow B = v_{n+2} - \sum_{i=1}^n s_i v_i = 0 
\hspace*{1cm}(2)
\]
\[
T = 1-\sum_{i=1}^{n}t_i
\]
\[
S = 1-\sum_{i=1}^{n}s_i
\]
اگر مجموع ضرایب عبارت (۱) را $T$ و مجموع ضرایب عبارت (۲) را $S$ می‌نامیم. اگر یکی از $T$ و $S$ برابر با صفر بود که همان عبارت جواب مسئله است زیرا مجموع ضرایب آن ۰ شده و ضریب جمله‌ی اول آن ناصفر(یک) است.

حال اگر 
$S \neq 0, T \neq 0$
در این داریم:
\[
S\times A - T \times B = 0 \rightarrow
S\times (v_{n+1} - \sum_{i=1}^n t_i v_i)
- T \times (v_{n+2} - \sum_{i=1}^n s_i v_i) = 0
\]
\[
\rightarrow -T\times v_{n+2} + S\times v_{n+1} 
+ \sum_{i=1}^n (T s_i - S t_i) v_i = 0
\]
که در ترکیب خطی بالا مجموع ضرایب برابر است با:
\[
-T + S + \sum_{i=1}^n T.s_i - S.t_i =
-T (1- \sum_{i=1}^n s_i) + S (1 - \sum_{i=1}^n t_i)
= -T\times S + S \times T = 0
\]
مجموع ضرایب ترکیب خطی نیز ۰ شد و همچنین ضریب $v_{n+1}$ برابر با $S \neq 0$ است پس این ترکیب خطی شرایط مسئله را دارد و حکم ثابت شد.

\section{سوال ۵}
طبق ۴، اعداد حقیقی 
$a_1, ..., a_{n+2}$
وجود دارند که بعضی از آن‌ها ناصفر بوده و 
\[
a_1 + \cdots + a_{n+2} = 0 \text{  و  }
a_1.u_1 + \cdots + a_{n+2}.u_{n+2} = 0
\]
حال عبارت زیر را در نظر بگیرید:
\[
a_1.u_1 + \cdots + a_{n+2}.u_{n+2} = 0
\]
چون مجموع ضرایب این ترکیب‌ خطی صفر است، پس مجموع ضرایب مثبت با مجموع ضرایب منفی برابر است، این مجموع را $s$ در نظر بگیرید، همچنین بردار‌هایی که ضریب آن‌ها منفی است را به سمت چپ تساوی منتقل کنید و سپس دو طرف معادله را در 
$\frac{1}{s}$
ضرب کنید. هر طرف تمام ضرایب مثبت و مجموع ضرایب هر طرف برابر با 
$\frac{s}{s} = 1$
است، پس هر طرف یک نقطه از یک پوش محدب است پس دو پوش محدب  پیدا کردیم که باهم اشتراک دارند.

\section{سوال ۶}
شرط لازم و کافی برای اینکه بردار 
$v = a_1 v_1 + \cdots + a_n v_n$
باید داشته باشد تا 
\\
$S = \{v_1, ..., v_{i-1}, v, v_{i+1}, ..., v_n\}$
یک پایه برای $V$ باشد این است که 
$a_i \neq 0$.

برای اینکار ثابت می‌کنیم اگر $a_i = 0$ باشد، $\{v_1, ..., v_{i-1}, v, v_{i+1}, ..., v_n\}$ پایه‌ای برای $V$ نیست و همچنین اگر 
$a_i \neq 0$
آنگاه 
$\{v_1, ..., v_{i-1}, v, v_{i+1}, ..., v_n\}$
پایه‌ای برای $V$ است.
\\
\textbf{اگر $a_i = 0$ باشد،}
در این صورت $v$ را می‌توان به صورت ترکیب خطی‌ای از بقیه‌ی عناصر $S$ است پس $S$ مستقل خطی نیست پس نمی‌تواند یک پایه برای $V$ باشد. (همچنین حتی $S$ مولدی برای $V$ نیز نیست!)
\\
\\
اگر 
$a_i \neq 0$
در این صورت اثبات می‌کنیم $S$ مستقل خطی است. فرض کنید:
\[
b_1 v_1 + ... + b_{i-1} v_{i-1} + b_i v + b_{i+1} v_{i+1} + ... + b_n v_n = 0
\]

اگر $b_i = 0$ باشد در این صورت:
\[
b_1 v_1 + ... + b_{i-1} v_{i-1} + b_{i+1} v_{i+1} + ... + b_n v_n = 0
\]
که چون 
$\{v_1, ..., v_{i-1}, v_{i+1}, ..., v_n\}$
مستقل خطی است، پس 
\[
b_1 = ... = b_{i-1} = b_{i+1} = ... = b_n = 0
\]
که در این حالت مسئله حل شد.
\\
حال اگر 
$b_i \neq 0$

\[
-b_i v =  b_1 v_1 + ... + b_{i-1} v_{i-1} + b_{i+1} v_{i+1} + ... + b_n v_n
\]
\[
\rightarrow 
- b_i a_1 v_1 + ... + - b_i a_n v_n = b_1 v_1 + ... + b_{i-1} v_{i-1} + b_{i+1} v_{i+1} + ... + b_n v_n
\]
\[
\rightarrow 
- b_i a_i v_i = (b_1 + b_i a_1) v_1 + ... + (b_{i-1} + b_i a_{i-1}) v_{i-1} + (b_{i+1} + b_i a_{i+1}) v_{i+1} + ... + (b_n + b_i a_n) v_n
\]
چون 
$b_i \neq 0 , a_i \neq 0$

\[
\rightarrow 
 v_i = \frac{b_1 + b_i a_1}{- b_i a_i} v_1 + ... + \frac{b_{i-1} + b_i a_{i-1}}{- b_i a_i} v_{i-1} + \frac{b_{i+1} + b_i a_{i+1}}{- b_i a_i} v_{i+1} + ... + \frac{b_n + b_i a_n}{- b_i a_i} v_n
\]
پس $v_i$ را توانستیم به صورت ترکیب خطی‌ای از 
$v_1, ..., v_{i-1}, v_{i+1}, ..., v_n$
بنویسیم که این با پایه بودن
$v_1, ..., v_n$
در تناقض است پس این حالت رخ نمی‌دهد.
\section{سوال ۷}
برای اینکه ویژگی مورد نیاز برای $v$ را پیدا کنیم، فرض کنید $v$ ای دلخواه داریم، برای اینکه 
$S = \{v_1 - v, \cdots, v_n-v \}$
یک پایه برای $V$ باشد، باید دو شرط زیر صدق کند:
\begin{enumerate}
	\item 
	اولا $S$ مولدی برای $V$ باشد.
	\item
	ثانیا $S$ مستقل خطی باشد.
\end{enumerate}
با توجه به اینکه 
$<S> \subseteq V$
اگر شرط دوم برقرار باشد آنگاه 
$\dim(<S>) = \dim(V)$
و نتیجه می‌گیریم که 
$<S> = V$
است و شرط اول را نتیجه می‌دهد. پس تنها بررسی شرط دوم لازم است.

فرض کنید اعداد 
$t_1, ..., t_n$
موجود باشند به‌طوری که:
\[
t_1 (v_1 - v) + ... + t_n (v_n - v) = 0 \rightarrow \sum_{i=1}^n t_i v_i = (\sum_{i=1}^n t_i) v
\]
حال دو حالت داریم:
\begin{enumerate}
	\item 
\textbf{	اگر 
	$\sum_{i=1}^n t_i = 0$
	باشد:}
\\
\[
\sum_{i=1}^n t_i v_i = 0
\]
و چون 
$\{v_1, ..., v_n\}$
مستقل خطی است نتیجه می‌گیریم
$t_1 = t_2 = ... = t_n = 0$
و درنتیجه 
$\{v_1 - v, ..., v_n - v\}$
مستقل خطی است.
\item
\textbf{
اگر 
$\sum_{i=1}^n t_i \neq 0$
باشد:
}
\\
آنگاه 
\[v = \frac{\sum_{i=1}^n t_i v_i}{\sum_{i=1}^n t_i}\]
\[
\rightarrow v = \sum_{i=1}^n s_i v_i, (s_i = \frac{t_i}{\sum_{i=1}^n t_i} \rightarrow \sum_{i=1}^n s_i = 1)
\]
\end{enumerate}
چون ما می‌خواهیم $\{v_1 - v, ..., v_n - v\}$ مستقل خطی شوند، پس باید نتیجه‌ی این عبارت یعنی 
$v = \sum_{i=1}^n s_i v_i, \sum_{i=1}^n s_i = 1$
برقرار نشود. پس شرط 
$v = a_1 v_1 + ... +a_n v_n$
 برای اینکه شرط مسئله برقرار شود این است که
$\sum_{i=1}^n a_i \neq 1$
شود، همچنین هر $v = a_1 v_1 + ... +a_n v_n$ دلخواه که 
$\sum_{i=1}^n a_i \neq 1$
شود را در نظر بگیریم، چون حالت دوم در استقلال خطی 
$\{v_1 - v, ..., v_n - v\}$ 
رخ نمی‌دهد، پس $\{v_1 - v, ..., v_n - v\}$  حتما مستقل خطی بوده و همچنین اگر  $v = a_1 v_1 + ... +a_n v_n$ ای داشتیم که 
$\sum_{i=1}^n a_i = 1$
بود، 
$\{v_1 - v, ..., v_n - v\}$ 
حتما وابسته‌ی خطی است زیرا:
\[
a_1 (v_1 - v) + ... + a_n (v_n - v) =a_1 v_1 + ... + a_n v_n - (a_1 + ... + a_n) v = v - v = 0
\]
در حالی که چون 
$\sum_{i=1}^n a_i = 1$
بوده پس حداقل یکی از $a_i$ ها ناصفر است و 
\[\{v_1 - v, ..., v_n - v\}\]
وابسته‌ی خطی است.
\\
\\
 پس شرط لازم و کافی برای 
$v = a_1 v_1 + ... +a_n v_n$
این است که 
$\sum_{i=1}^n a_i \neq 1$
.
\\
که این از نظر هندسی به این معنی است که $v$ بر روی صفحه‌ی تعمیم‌یافته‌ی تشکیل شده از نقاط $v_1, ..., v_n$ نباشد.
\section{سوال ۸}
خیر، به طور مثال دو مجموعه‌ی زیر را در نظر بگیرید:
\[S_1 = {e_1, ..., e_n}\]
\[S_2 = {2e_1, ..., 2e_n}\]
هردوی این مجموعه‌ها مولد $R^n$ اند، پس داریم:
\[<S_1> \cap <S_2> = R^n \cap R^n = R^n\]
در حالی که
\[<S_1 \cap S_2> = <\emptyset> = {0} \]
پس اگر $n \neq 0$ باشد در مثال ما
$<S_1> \cap <S_2> \neq <S_1 \cap S_2>$
\section{سوال ۹}
$W_1 \cup W_2$
یک زیرفضای 
$R^n$
است، اگر و تنها اگر 
$W_1 \subseteq W_2$
یا
$W_2 \subseteq W_1$
.
\\
برای اثبات ادعای بالا اگر یکی زیرمجموعه‌ی دیگری باشد آن‌گاه اجتماع این دو نیز برابر با زیرفضای بزرگتر می‌شود و در نتیجه خود یک زیرفضای برداری است.
\\
حال فرض کنید 
$W_1 \nsubseteq W_2$ 
و
$W_2 \nsubseteq W_1$
در این صورت 
\[
\exists v \in W_1 \land v \notin W_2, \exists u \in W_2 \land u \notin W_1
\]
حال اگر 
$W_1 \cup W_2$
یک زیرفضای برداری باشد،‌ آنگاه
$<W_1 \cup W_2> = W_1 \cup W_2$
همچنین از قبل می‌دانیم:
$<W_1 \cup W_2> = W_1 + W_2$
پس باید
$W_1 \cup W_2 = W_1 + W_2$
همچنین چون 
$v \in W_1, u \in W_2$
است، نتیجه می‌گیریم:
$v+u \in W_1 + W_2 = W_1 \cup W_2$
پس $v+u$ باید در یکی از $W_1$ یا $W_2$ باشد، بدون خدشه به کلیت مسئله فرض کنید در $W_1$ باشد پس چون $W_1$ یک زیرفضای برداری است داریم:
$v+u \in W_1, v \in W_1 \rightarrow -v \in W_1 \rightarrow v+u-v = u \in W_1$
که این با 
$u \notin W_1$
در تناقض است و $W_1 \cup W_2$ یک زیرفضای برداری نیست.
\\
\\
همچنین در مورد $W_1^c$ چون 
$0 \in W_1$
پس
$0 \notin W_1^c$
پس $W_1^c$ یک زیرفضای برداری نیست.
\section{سوال ۱۰}
در قسمت نخست سوال فرض می‌کنیم
$W_1 \subseteq W$ 
 است و 
 باید دو چیز را ثابت کنیم:
\\
\textbf{اول اینکه 
$W \cap W_1$
و 
$W \cap W_2$
مستقل خطی اند.}
که برای اثبات آن فرض کنید 
$v_1 \in W \cap W_1, v_2 \in W \cap W_2$
داریم به‌طوری که
$v_1 + v_2 = 0$
است.
چون 
$v_1 \in W \cap W_1$ 
بود نتیجه می‌گیریم 
$v_1 \in W_1$
و به طور مشابه نتیجه می‌گیریم 
$v_2 \in W_2$
است و از استقلال خطی $W_1, W_2$ نتیجه می‌گیریم 
$v_1 = v_2 = 0$
و ثابت شد $W \cap W_1$
و 
$W \cap W_2$
مستقل خطی اند.
\\\\
\textbf{دوم اینکه 
$W = (W \cap W_1) \oplus (W \cap W_2)$
است.}
\\
برای اینکار ثابت می‌کنیم هر عضو طرف چپ در طرف راست و هر عضو طرف راست در طرف چپ است.
\\
عضو دلخواه $v$ از طرف راست را در نظر بگیرید داریم:
$v = u + w, u \in W \cap W_1, w \in W \cap W_2$
پس
$v = u + w, u,w \in W \rightarrow v \in W$
و ثابت شد هر عضو طرف راست در طرف چپ است.
\\
حال عضو دلخواه $w$ از طرف چپ را در نظر بگیرید:
\[
w \in W \subseteq V \rightarrow w = v + u, v \in W_1, u \in W_2
\]
از طرفی:
\[
v \in W_1 \subseteq W \rightarrow v \in W
\rightarrow v \in W \cap W_1
\]
همچنین چون 
$u \in W_2$
بود داریم:
\[
v \in W, v+u \in W \rightarrow u \in W \rightarrow u \in (W \cap W_2)
\]
پس از دو عبارت قبل نتیجه می‌گیریم:
\[
v + u \in (W \cap W_1) + (W \cap W_2) \rightarrow
w \in (W \cap W_1) \oplus (W \cap W_2)
\]
و ثابت شد هر عضو طرف چپ نیز در سمت راست است و حکم ثابت شد.
\\
\\
حال اگر $W$ شامل $W_1, W_2$ نباشد، مثال نقضی را در $R^2$ مطرح می‌کنیم که قابل تعمیم به $R^n$ 
$(n > 1)$
 است.
\\
فرض کنید 
$W_1 = <e_1>, W_2 = <e_2>, V = W_1 \oplus W_2 = R^2$
است.
\\
حال
$W = <(1, 1)>$
در نظر بگیرید.
\[
W\cap W_1 = {0}, W \cap W_2 = {0} \rightarrow 
W \neq (W \cap W_1) \oplus (W \cap W_2)
\]
و مثال نقض درست است.
\end{document}