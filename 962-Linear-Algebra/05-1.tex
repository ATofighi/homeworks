\documentclass[12pt,a4paper]{article}
\usepackage{multirow}
\usepackage{rotating}
\usepackage{graphicx}
\usepackage{amsmath}
\usepackage{amsfonts}
\usepackage{amssymb}
\usepackage{graphicx}
%\pagestyle{empty}
%\usepackage[bottom=0.5in,headheight=0pt,headsep=0pt]{geometry}
%\addtolength{\topmargin}{0pt}
\DeclareMathOperator{\rank}{rank}
\DeclareMathOperator{\im}{Im}
\usepackage{xepersian}
\linespread{1.2}
\settextfont{XB Niloofar}
\setdigitfont{XB Niloofar}

\begin{document}
\begin{center}
	بسمه تعالی
\end{center}

\begin{center}
	\textbf{
		پاسخ سری پنجم تمرین‌ها
		- درس جبرخطی ۱ - دانشگاه صنعتی شریف}
	\\
	علیرضا توفیقی محمدی - رشته علوم کامپیوتر - شماره‌ی دانشجویی: ۹۶۱۰۰۳۶۳
\end{center}

\section{تمرین 16 سری سوم}
\subsection{
$\rank A + \rank B - n \leq \rank AB$
}
برای اثبات 
$\ker A$،
$\ker B$ و
$\ker AB$
را در نظر بگیرید.
ادعا می‌کنیم:
$$
\dim(\ker AB) \leq \dim(\ker A) + \dim(\ker B)
$$
زیرا 
$$
\ker AB = \ker B + \ker A|_{\im B}
\implies
$$$$
\dim(\ker AB) = \dim(\ker B) + \dim(\ker A|_{\im B}) - 
\dim(\ker A|_{\im B} \cap \ker B)
$$$$= \dim(\ker B) + \dim(\ker A|_{\im B}) \leq
\dim(\ker B) + \dim(\ker A)
$$
و چون 
$\dim \ker T + \rank T = n$
داریم:
$$n-\rank AB \leq n-\rank A + n - \rank B
\implies 
\rank A + \rank B - n \leq \rank AB
$$
\subsection{$\rank AB \leq \min\{\rank A, \rank B\}$}
اولا
$$\rank AB = \rank A|_{\im B} \leq \rank A$$
\\
ثانیا 
$$\ker B \subseteq \ker AB \implies
\dim\ker B \leq \dim\ker AB \implies$$$$
n-\rank B \leq n - \rank AB \implies
\rank AB \leq \rank B$$
پس
$$\rank AB \leq \min\{\rank A, \rank B \}$$
و حکم ثابت شد.

\subsection{$\rank AB + \rank BC \leq \rank ABC + \rank B$}
برای اثبات ابتدا ادعا می‌کنیم
$$\im AB = \im A|_{\im B}$$
که تقریبا واضح است، زیرا:
$$ v \in \im AB \iff \exists u: AB(u) = v \iff A(B(u)) = v $$$$\iff w = B(u) \in \im B; A(w) = v \iff v \in \im A|_{\im B}$$

همچنین می‌دانیم:
$$
\dim \im A|_{X} + \dim \ker A|_{X} = \dim X
$$
پس
$$
\rank A|_{\im B} + \dim \ker A|_{\im B} = \dim \im B = \rank B
$$
پس
$$
\rank AB + \dim \ker A|_{\im B} = \rank B \text{(1)\space\space\space\space\space}
$$
همچنین می‌دانیم:
$$
V \subseteq U \implies \dim \ker A|_V \leq \dim \ker A|_U
$$
و همچنین می‌دانیم 
$\im BC \subseteq \im B$
پس:
$$
\dim \ker A|_{\im BC} \leq \dim \ker A|_{\im B}
$$
حال از (۱) و نتیجه‌ی بالا استفاده می‌کنیم، داریم:
$$
\rank B = \rank AB + \dim \ker A|_{\im B} \geq 
\rank AB + \dim \ker A|_{\im BC}$$
پس
$$\rank B - \dim \ker A|_{\im BC} \geq \rank AB$$
حال 
$\rank BC$
را به دو طرف تساوی اضافه می‌کنیم:
$$\rank B - \dim \ker A|_{\im BC} + \rank BC\geq \rank AB + \rank BC$$
و طبق ادعای ابتدای این بخش (با قرار دهی $BC$ به حای $B$ در آن ادعا) داریم:
$$\rank B + \rank ABC\geq \rank AB + \rank BC$$
\subsection{$\rank(A + B) \leq \rank A + \rank B$}
داریم:
$$
\im (A+B) \subseteq \im A + \im B
$$
پس
$$
\rank(A+B) \leq \dim(\im A + \im B) =
\rank A + \rank B - \dim(\im A \cap \im B)
$$
$$
\leq \rank A + \rank B
$$
یعنی
$$
\rank (A+B) \leq \rank A + \rank B
$$

\section{تمرین ۱ سری چهارم}
پاسخ هر قسمت را با 
$\{u_1, ..., u_n\}$
نشان می‌دهیم.
\subsection{
$\{v_2, v_1, v_3, ..., v_n\}$
}
برای $i > 2$ مشخص است که 
$u^{*}_i = v^{*}_i$.

برای 
$u_1$
داریم:
$u_1(v_2) = 1, u_1(v_i) = 0 (i \neq 2)$
پس
$u_1 = v^{*}_2$
و به طریق مشابه 
$u_2 = v^{*}_1$.

\subsection{
$\{rv_1, v_2, ..., v_n\}$
}
برای $i=1$ داریم:
$u_1(rv_1) = ru_1(v_1) = 1, u_1(v_i) = 0 (i \neq 1)$
پس 
$$u_1(v_1) = r^{-1}, u_1(v_i) = 0 (i \neq 1)$$
پس
$u_1 = r^{-1}v^*_1$.

همچنین برای $i > 1$ داریم:
$$u_i(v_i) = 1, u_i(v_j) = 0 (j > 1, j \neq i), u_i(rv_1) = r u_i(v_1) = 0 \implies u_i(v_1) = 0$$
پس $u_i = v^*_i$
\subsection{
$\{v_1 + r v_2, v_2, v_3, ..., v_n\}$
}
برای $i > 2$ داریم 
$u_i = v^*_i$.
کافی است قرار دهیم:
$u_1 = v^*_1, u_2 = -rv^*_1 + v^*_2$

در این صورت برای 
$i > 2$
داریم:
$$u_1(v_i) = u_2(v_i) = 0$$

و همچنین
$$
u_1(v_1+rv_2) = v^*_1(v_1 + rv_2) = v^*_1(v_1) + rv^*_1(v_2) = 1, u_1(v_2) = v^*_1(v_2) = 0
$$
$$
u_1(v_1+rv_2) = (-rv^*_1 + v^*_2)(v_1 + rv_2) = (-rv^*_1 + v^*_2)(v_1) + r(-rv^*_1 + v^*_2)(v_2) = -r + r = 0$$
$$(-rv^*_1 + v^*_2)(v_2) = -rv^*_1(v_1) + v^*_2(v_2) = 0 + 1 = 1
$$
پس شرایط را دارد.
\subsection{
$\{t_1 v_1 + t_2 v_2 + ... + t_n v_n, v_2, v_3, ..., v_n\}$
}
$u_1 = t_1^{-1}v^*_1$
و برای $i>1$
$u_i = - t_i t_1^{-1} v^*_1 + v^*_i$
تعریف می‌کنیم. داریم:
$$u_1(t_1v_1) = (t_1^{-1}v^*_1)(t_1v_1) = t_1^{-1}t_1 = 1$$
$$ i > 1: u_1(v_i) = (t_1^{-1}v^*_1)(v_i) = t_1^{-1}v^*_1(v_i) = 0$$
$$ i > 1: u_i(t_1 v_1 + ... + t_n v_n) = 
(- t_i t_1^{-1} v^*_1 + v^*_i)(t_1 v_1 + ... + t_n v_n)
$$$$= - t_i t_1^{-1} v^*_1(t_1 v_1 + ... + t_n v_n) + v^*_i(t_1 v_1 + ... + t_n v_n)$$$$
- t_i t_1^{-1} t_1 + t_i = -t_i + t_i = 0
$$
$$
i, j > 1, j \neq i: u_i(v_j) = (- t_i t_1^{-1} v^*_1 + v^*_i)(v_j) = - t_i t_1^{-1} v^*_1(v_j) + v^*_i(v_j) = - t_i t_1^{-1} 0 + 0 = 0
$$
$$ i > 1: u_i(v_i) = (- t_i t_1^{-1} v^*_1 + v^*_i)(v_i) = - t_i t_1^{-1} v^*_1(v_1) + v^*_i(v_i) = 
- t_i t_1^{-1} 0 + 1 = 1
$$
\subsection{
$\{a_1 v_1 + a_2 v_2, b_1 v_1 + b_2 v_2, v_3, ..., v_n\}$
}
اگر یکی از 
$a_1, a_2, b_1, b_2$
صفر بود، مسئله به راحتی از طبق بخش‌های قبل حل می‌شود پس فرض می‌کنیم همه‌ی این متغیرها ناصفر اند.
فرض کنید 
$u_1 = p v^*_1 + q v^*_2, u_2 = c v^*_1 + dv^*_2$
است. باید داشته باشیم:
$$
u_1(a_1 v_1 + a_2 v_2) = 1, u_1(b_1 v_1 + b_2 v_2) = 0
$$
$$
u_2(a_1 v_1 + a_2 v_2) = 0, u_2(b_1 v_1 + b_2 v_2) = 1
$$
با حل این معادله‌ها خواهیم داشت:
$$
q = (-a_1b_2b_1^{-1} + a_2)^{-1}, p = -b_2b_1^{-1}q
$$
و
$$
d = (b_1 a_2a_1^{-1} + b_2)^{-1}, c = a_1^{-1} a_2 d
$$
همچنین 
$-a_1b_2b_1^{-1} + a_2 \neq 0$
زیرا در غیر اینصورت:
$$-a_1b_2b_1^{-1} + a_2 = 0 \implies -a_1b_2 + a_2 b_1 = 0 \implies a_1b_2 = a_2b_1$$
و به طریق مشابه 
$b_1 a_2a_1^{-1} + b_2 \neq 0$
پس متغیر‌ها به درستی تعریف می‌شوندو همچنین اگر برای $i > 2$ تعریف کنیم
$u_i = v^*_i$
مسئله حل می‌شود.
\section{تمرین ۴ سری چهارم}
\subsection{قسمت 7}
برای اثبات دو لم (که یکی قسمت ۶ سوال است) استفاده می‌کنیم.
\subsubsection{لم ۱}
\textbf{
اگر $f$ تابعکی خطری روی 
$M_{n\times n}(F)$
باشد که 
$\forall A,B \in M_{n\times n}(F): f(AB) = f(BA)$
آنگاه $f$ مضربی از $tr$ است.
}

اثبات: $e_{ij}$ را ماتریس استاندارد با خانه‌ی $i, j$ یک در نظر می‌گیریم. تنها کافی است ثابت کنیم
$f(e_{ii}) = f(e_{jj})$
و
$i\neq j: f(e_{ij}) = 0$

می‌دانیم
$e_{ij} = e_{ik}e_{kj}$
(به سادگی ضرب را انجام می‌دهیم و نتیجه می‌گیریم.)
پس:
$$f(e_{ii}) = f(e_{ij}e_{ji}) = f(e_{ji}e_{ij}) = f(e_{jj})$$
و همچنین برای $i \neq j$:
$$
f(e_{ij}) = f(e_{ii}e_{ij}) = f(e_{ij}e_{ii}) = f(0) = 0
$$
پس $f$ مضربی از $tr$ است.
\subsubsection{لم ۲}
\textbf{
اگر $A$ یک ماتریس باشد، ماتریس‌های وارون‌پذیر $X$ و $Y$ وجود دارند که 
$A = X+Y$
}

اثبات. حکم معادل این گزاره را برای نگاشت‌ها ثابت می‌کنیم. فرض کنید $T$ یک علمگرخطی روی فضای $n$ بعدی $V$ باشد. 
$v_1, ..., v_m$
را پایه‌ای برای $\ker T$ در نظر بگیرید و آن‌ را به پایه‌ای برای $V$ مثل 
$v_1, ..., v_m, v_{m+1}, ..., v_n$
گسترش دهید. حال می‌دانیم 
$T(v_{m+1}), ..., T(v_n)$
پایه‌ای برای $\im T$ است. فرض کنید این پایه را نیز به پایه برای $V$ مثل
$$u_1, ..., u_m, u_{m+1} = T(v_{m+1}), ..., u_n = T(v_n)$$
گسترش می‌دهیم. حال با فرض $2 \neq 0$ نگاشت $S$ را به شکل زیر تعریف می‌کنیم:
$$\forall 1 \leq i \leq m: S(v_i) = u_i \times 2^{-1},
\forall m+1 \leq i \leq n: S(v_i) = u_i$$
و همچنین نگاشت $U$ را تعریف می‌کنیم:
$$\forall 1 \leq i \leq m: U(v_i) = u_i \times 2^{-1}, \forall m+1 \leq i \leq n: U(v_i) = -u_i$$
که $S$ و $U$ هردو وارون پذیرند و $S+U = T$ و حکم معادل ثابت و حکم اصلی ثابت شد.
\subsubsection{مسئله اصلی}
فرض کنید $A,B$ دو ماتریس دلخواه اند و طبق لم ۲ 
$B = B_1 + B_2$
که $B_1, B_2$ ماتریس‌هایی وارون‌پذیر باشند.
داریم:
$$f(BA) = f((B_1+B_2)A) = f(B_1A + B_2A) = f(B_1A) + f(B_2 A) $$$$= f(B_1^{-1}B_1AB_1) + f(B_2^{-1}B_2AB_2) = f(AB_1) + f(AB_2) = f(AB_1 + AB_2) $$$$= f(A(B_1+B_2)) = f(AB)$$
پس حکم لم ۱ برای این مسئله نیز برقرار است و در نتیجه $f$ مضربی از $tr$ است.

\subsection{قسمت ۹}
تنها کافی‌ست ثابت کنیم هسته‌ی این تابع صفر است. فرض کنید مقدار این تابع روی تابع $A$ صفر شده‌است پس:
$$
f_A = 0 \implies \forall X: f_A(X) = 0 \implies tr(AX) = 0
$$
حال اگر به جای $X$ ماتریس $e_{ji}$ را قرار دهیم،
$A e_{ji}$
ماتریسی است که فقط یک خانه از قطراصلی آن ناصفر است و مقدار آن خانه برابر با $A_{ij}$ است (زیرا می‌توانیم $A$ را به صورت 
$A = \sum_{i=1}^{n}\sum_{j=1}^{n}A_{ij}e_{ij}$
بنویسیم و ضرب را انجام دهیم.
)پس
$$
tr(Ae_{ji}) = 0 \implies A_{ij} = 0
$$
پس همه‌ی خانه‌های $A$ صفر است پس $A = 0$ پس کرنل این نگاشت خطی ۰ است و چون بعد مبدا و مقصد یکسان است، یک‌به‌یک و پوشاست. 
\subsection{قسمت ۱۱}
فرض کنید $A$ ماتریس با ویژگی‌های مسئله باشد.
اگر $i\neq j$ باشد می‌دانیم $tr(e_{ji}) = 0$ است پس: (استدلالی برای ضرب پایین در قسمت قبل کردیم)
$$tr(A e_{ji}) = 0 \implies A_{ij} = 0$$
همچنین برای $i \neq j$ ماتریس 
$X = e_{ii} - e_{jj}$
را در نظر بگیرید، 
$tr(X) = 0$
است پس داریم:
$$ tr(A X)= 0 \implies tr(A (e_{ii} - e_{jj})) = 0 \implies tr(A e_{ii}) = tr(Ae_{jj}) \implies A_{ii} = A_{jj}$$
پس $A$ مضربی از همانی است.
\end{document} 