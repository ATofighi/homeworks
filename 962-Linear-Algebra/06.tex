\documentclass[12pt,a4paper]{article}
\usepackage{multirow}
\usepackage{rotating}
\usepackage{graphicx}
\usepackage{amsmath}
\usepackage{amsfonts}
\usepackage{amssymb}
\usepackage{graphicx}
%\pagestyle{empty}
%\usepackage[bottom=0.5in,headheight=0pt,headsep=0pt]{geometry}
%\addtolength{\topmargin}{0pt}
\DeclareMathOperator{\rank}{rank}
\DeclareMathOperator{\im}{Im}
\usepackage{xepersian}
\linespread{1.2}
\settextfont{XB Niloofar}
\setdigitfont{XB Niloofar}

\begin{document}
\begin{center}
	بسمه تعالی
\end{center}

\begin{center}
	\textbf{
		پاسخ سری ششم تمرین‌ها
		- درس جبرخطی ۱ - دانشگاه صنعتی شریف}
	\\
	علیرضا توفیقی محمدی - رشته علوم کامپیوتر - شماره‌ی دانشجویی: ۹۶۱۰۰۳۶۳
\end{center}

\section{تمرین 6 از بخش پنجم}
چون مجموعه‌جواب‌های $AX = 0$ و $A'X = 0$ یکی هستند، پس  طبق قضیه‌ی اثبات‌شده در جزوه این دو دستگاه هم‌ارز بوده و ماتریس $A$ هم‌ارز سطری با ماتریس $A'$ است و چون $A$ و $A'$ ماتریس‌های ساده‌سطری هستند طبق تمرین جزوه $A = A'$.
\section{تمرین ۱۰ از بخش پنجم}
\subsection{لم}
\textbf{
اگر $D_n$ تعداد جایگشت‌های پریش به طول $n$ باشد، برای $n$های زوج $D_n$ فرد است.
}
\\
از درس ریاضیات گسسته می‌دانیم برای $n > 2$:
$D_n = (n-1)(D_{n-1}+D_{n-2})$.
حال با استقرا روی $n$ ثابت می‌کنیم $D_n$ برای $n$های فرد، زوج و برای $n$های زوج فرد است.
\\
پایه:
برای $D_1 = 0$ و $D_2 = 1$ و حکم برقرار است.
\\
حال فرض کنید حکم برای همه‌ی $n$های کمتر از $k$ برقرار باشد.
\\
$D_k$
را در نظر بگیرید، اگر $k$ زوج بود آن‌گاه $k-1$ فرد و $D_{k-1}$ طبق فرض استقرا زوج و $D_{k-2}$ طبق فرض استقرا فرد است. پس 
$(k-1)(D_{k-1}+D_{k-2})$
فرد است پس $D_k$ فرد است.
\\
اگر $k$ فرد بود آن‌گاه $k-1$ زوج بوده و $(k-1)(D_{k-1}+D_{k-2})$ نیز زوج خواهد بود و $D_k$ زوج می‌شود.
\\
پس گام استقرا نیز ثابت شد و لم ثابت شد.
\subsection{مسئله‌ی اصلی}
حال به کمک رابطه‌ی 
$$\det A = \sum_{\sigma \in S^n} \epsilon_\sigma a_{\sigma(1)1}\dots a_{\sigma(n)n}$$
سعی بر اثبات حکم می‌کنیم؛ تنها کافی است برای محاسبه‌ی دترمینان، جملات مثبت در سیگمای سمت راست را در نظر بگیریم و چون جملات روی قطر صفر است برای هر $i$ مقدار $a_{ii} = 0$ است. پس اگر برای $i$ ای 
$\sigma(i) = i$
شود مقدار 
$\epsilon_\sigma a_{\sigma(1)1}\dots a_{\sigma(n)n}$
نیز برابر با ۰ می‌شود و در غیر این‌صورت چون برای $i\neq j$ طبق صورت سوال 
$a_{ij} = \pm1$
است و $\epsilon_\sigma = \pm 1$ پس حاصل ضرب آن‌ها نیز مثبت یا منفی یک است.
پس $\det A$ برابر با جمع $D_n$ تا ($n = 2k$) مثبت یا منفی یک است و اگر فضا را حقیقی یا مختلط در نظر بگیریم چون $n$ زوج است طبق لم $D_n$ فرد است و در نتیجه $\det A$ برابر با جمع تعداد فردی مثبت و منفی یک است و در فضای حقیقی و مختلط جمع تعداد فردی مثبت یا منفی یک نمی‌تواند صفر شود پس $\det A \neq 0$ است و در نتیجه $A$ وارون‌پذیر است.

اما اگر میدان ما $\mathbb{R}$ نباشد، حکم لزوما برقرار نیست، مثلا در 
$\mathbb{Z}_3$

$\det \begin{bmatrix}
	0&1&1&1\\
	1&0&1&1\\
	1&1&0&1\\
	1&1&1&0
\end{bmatrix} = 0$
است پس
$\begin{bmatrix}
0&1&1&1\\
1&0&1&1\\
1&1&0&1\\
1&1&1&0
\end{bmatrix}$
 وارون پذیر نیست.
 
\section{تمرین 7 از بخش شش}
\subsection{}
$\det X = \det\begin{bmatrix}
	A&B\\
	0&D
	\end{bmatrix} = \det A . \det D$:

برای اثبات از رابطه‌ی جایگشتی دترمینان یعنی 
$$\det X = \sum_{\sigma \in S^n} \epsilon_\sigma a_{\sigma(1)1}\dots a_{\sigma(n)n}$$
استفاده می‌کنیم، اندازه‌ی ماتریس $X$ را $n\times n$ در نظر گرفتیم که برای اینکه رابطه معنی داشته باشد باید $n$ زوج باشد مثلا $2k$ باشد و ماتریس‌های $A, B, D$ باید $k\times k$ باشند.
\\
از سیگمای بالا تنها باید جملات ناصفر را در نظر بگیریم، پس تنها $\sigma$هایی را در نظر می‌گیریم که برای $i \leq k$ داشته‌باشیم: $\sigma(i) \leq k$ و در نتیجه برای $i > k$ خواهیم داشت $\sigma(i) > k$. پس چنین تابع $\sigma$ را می‌توانیم به با دو تابع $\sigma_1, \sigma_2 \in S^k$ نشان داد بدین صورت که:

$$\sigma(i) = \begin{cases}
\sigma_1(i)&i \leq k \\
\sigma_2(i)+k&i > k
\end{cases}$$
همچنین چون علامت جایگشت به زوجیت تعداد نابه‌جایی‌های جایگشت ربط دارد و هیچ نابه‌جایی‌ای بین نیمه‌ی چپ و راست وجود ندارد می‌توان گفت
$\epsilon_\sigma = \epsilon_{\sigma_1} \times \epsilon_{\sigma_2}$.
حال به بازنویسی عبارت دترمینان برمی‌گردیم:
$$\det X = \sum_{\sigma \in S^n} \epsilon_\sigma a_{\sigma(1)1}\dots a_{\sigma(n)n}
$$$$
 = \sum_{\sigma_1, \sigma_2 \in S^k} \epsilon_{\sigma_1}\epsilon_{\sigma_2} a_{\sigma_1(1) , 1}\dots a_{\sigma_1(k) , k} a_{(k+\sigma_2(1)) , 1}\dots a_{(k+\sigma_2(k)) , k}
$$
$$
= \sum_{\sigma_1 \in S^k} \sum_{\sigma_2 \in S^k} \epsilon_{\sigma_1}\epsilon_{\sigma_2} a_{\sigma_1(1) , 1}\dots a_{\sigma_1(k) , k} a_{(k+\sigma_2(1)) , 1}\dots a_{(k+\sigma_2(k)) , k}
$$
$$
= \sum_{\sigma_1 \in S^k} \epsilon_{\sigma_1} a_{\sigma_1(1) , 1}\dots a_{\sigma_1(k) , k} \sum_{\sigma_2 \in S^k} \epsilon_{\sigma_2} a_{(k+\sigma_2(1)) , 1}\dots a_{(k+\sigma_2(k)) , k}
$$
از طرفی 
$$\det D = \sum_{\sigma_2 \in S^k} \epsilon_{\sigma_2} a_{(k+\sigma_2(1)) , 1}\dots a_{(k+\sigma_2(k)) , k}$$
و
$$\det A = \sum_{\sigma_1 \in S^k} \epsilon_{\sigma_1} a_{\sigma_1(1) , 1}\dots a_{\sigma_1(k) , k}$$
است پس:
$$
\det X = \sum_{\sigma_1 \in S^k} \epsilon_{\sigma_1} a_{\sigma_1(1) , 1}\dots a_{\sigma_1(k) , k} \det D = \det D . \det A = \det A . \det D
$$
و حکم ثابت شد.
\subsection{}
$$
\det \begin{bmatrix}EA&EB\\C&D\end{bmatrix} =
 \det E . \det \begin{bmatrix}
A&B\\C&D
\end{bmatrix}
$$
برای اثبات تنها کافی‌است به دترمینان
$\begin{bmatrix}E&0\\0&I\end{bmatrix}\times
\begin{bmatrix}A&B\\C&D\end{bmatrix}$
نگاه کنیم.
از طرفی با ضرب بلوکی ماتریس‌ها داریم:
$$\begin{bmatrix}E&0\\0&I\end{bmatrix}\times
\begin{bmatrix}A&B\\C&D\end{bmatrix} = \begin{bmatrix}EA&EB\\C&D\end{bmatrix}$$
و از طرفی طبق قسمت قبل 
$$\det \begin{bmatrix}E&0\\0&I\end{bmatrix} = \det E . \det I = \det E$$
پس داریم:
$$\det \left(\begin{bmatrix}E&0\\0&I\end{bmatrix}\times
\begin{bmatrix}A&B\\C&D\end{bmatrix}\right) = \det \begin{bmatrix}EA&EB\\C&D\end{bmatrix}$$
و
$$\det \left(\begin{bmatrix}E&0\\0&I\end{bmatrix}\times
\begin{bmatrix}A&B\\C&D\end{bmatrix}\right) = 
\det \begin{bmatrix}E&0\\0&I\end{bmatrix}\times
\det \begin{bmatrix}A&B\\C&D\end{bmatrix} = \det E . \det \begin{bmatrix}A&B\\C&D\end{bmatrix} $$
پس
$$
\det \begin{bmatrix}EA&EB\\C&D\end{bmatrix} = \det E . \det \begin{bmatrix}A&B\\C&D\end{bmatrix}
$$
و حکم ثابت شد.
\subsection{}
$$
\det \begin{bmatrix}A&B\\C+EA&D+EB\end{bmatrix} = \det \begin{bmatrix}A&B\\C&D\end{bmatrix}
$$
مشابه قسمت اول سوال می‌توان نشان داد
$\det \begin{bmatrix}A&0\\B&D\end{bmatrix} = \det A . \det B$.
\\
حال به دترمینان ضرب زیر نگاه می‌کنیم:
$$\begin{bmatrix}I&0\\E&I\end{bmatrix}\times
\begin{bmatrix}A&B\\C&D\end{bmatrix}$$
از طرفی 
$$\begin{bmatrix}I&0\\E&I\end{bmatrix}\times
\begin{bmatrix}A&B\\C&D\end{bmatrix} = \begin{bmatrix}A&B\\C+EA&D+EB\end{bmatrix}$$$$ \implies
\det \left(\begin{bmatrix}I&0\\E&I\end{bmatrix}\times
\begin{bmatrix}A&B\\C&D\end{bmatrix} \right)= \det \begin{bmatrix}A&B\\C+EA&D+EB\end{bmatrix}
$$
و از طرفی
$$
\det \left(\begin{bmatrix}I&0\\E&I\end{bmatrix}\times
\begin{bmatrix}A&B\\C&D\end{bmatrix} \right)=
\det \begin{bmatrix}I&0\\E&I\end{bmatrix} .
\det \begin{bmatrix}A&B\\C&D\end{bmatrix}$$$$ = \det I.\det I . \det \begin{bmatrix}A&B\\C&D\end{bmatrix} = 1.\det \begin{bmatrix}A&B\\C&D\end{bmatrix} = \det \begin{bmatrix}A&B\\C&D\end{bmatrix}
$$
پس
$$
\det \begin{bmatrix}A&B\\C+EA&D+EB\end{bmatrix} = \det \begin{bmatrix}A&B\\C&D\end{bmatrix}
$$
و حکم ثابت شد.
\end{document} 