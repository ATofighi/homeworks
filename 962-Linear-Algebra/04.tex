\documentclass[12pt,a4paper]{article}
\usepackage{multirow}
\usepackage{rotating}
\usepackage{graphicx}
\usepackage{amsmath}
\usepackage{amsfonts}
\usepackage{amssymb}
\usepackage{graphicx}
%\pagestyle{empty}
%\usepackage[bottom=0.5in,headheight=0pt,headsep=0pt]{geometry}
%\addtolength{\topmargin}{0pt}
\usepackage{xepersian}
\linespread{1.2}
\settextfont{XB Niloofar}

\begin{document}
\begin{center}
	بسمه تعالی
\end{center}
\begin{center}
	\textbf{
		پاسخ سری چهارم تمرین‌ها
		-
		- درس جبرخطی ۱ - دانشگاه صنعتی شریف}
	\\
	علیرضا توفیقی محمدی - رشته علوم کامپیوتر - شماره‌ی دانشجویی: ۹۶۱۰۰۳۶۳
\end{center}
\section{تمرین ۱۶ سری دوم}
$\mathbb{R}$
با میدان 
$\mathbb{Q}$
را در نظر بگیرید، این فضا با جمع و ضرب معمولی خود یک فضای برداری است این فضا را $A$ بنامید.
\\
حال فضای برداری $\mathbb{C}$ با میدان $Q$ را در نظر بگیرید، این فضا با جمع و ضرب معمولی خود نیز یک فضای برداری است، این را $B$ بنامید.
\\
ثابت می‌کنیم 
$\dim(A) = \dim(B)$،
برای اینکار واضح است که بعد این هیچ کدام از این دو میدان متناهی نیست، (زیرا در این صورت 
$|\mathbb{R}| = |\mathbb{Q}| = |\mathbb{N}|$
می‌شد.
)
حال یک پایه برای $B$ در نظر بگیرید، اگر این پایه از تعدادی عضو به صورت $a + bi$ تشکیل شده‌بود، قسمت اول این اعضا یعنی $a$ را در نظر بگیرید، این یک مولد برای $A$ است پس 
$\dim(A) \leq \dim(B)$
. همچنین اگر یک پایه برای $A$ در نظر بگیرید، به ازای هر عضو این پایه مثل $a$ عضو های $a$ و $ai$ را در نظر بگیرید، این $2\dim(A)$ عضو یک مولد برای $B$ هستند، پس
$\dim(B) \leq 2\dim(A) = \dim(A)$
.
پس $\dim(A) = \dim(B)$.
\\
چون این دو فضا دارای یک بعد اند، پس نگاشت خطی 
$T: \mathbb{C} \rightarrow \mathbb{R}$
 که یک به یک و پوشاست روی این دو فضای برداری وجود دارد.\\
چون این نگاشت خطی است داریم:
$$
T(x+y) = T(x) + T(y)$$
و چون یک به یک و پوشاست $T^{-1}$ نیز خطی است.
\\
همچنین فضای برداری $\mathbb{C}$ روی میدان $\mathbb{C}$ را می‌شناسیم که ضرب اسکالر روی آن تعریف می‌شود.
\\
حال برای یک عدد مختلط $c$ و یک عدد حقیقی $X$ تعریف می‌کنیم:
$$
c.X := T(c.T^{-1}(Y))
$$
پس طبق تعریف بالا برای عدد مختلط $c$ و عدد مختلط $X$ خواهیم داشت:
$$
c.X = Y \iff c.T(X) = T(Y)
$$
\\
حال اعداد مختلط $s,t$ و عدد حقیقی $V$ و $U$ را در نظر بگیرید و 
$X = T^{-1}(V)$ و
$Y = T^{-1}(U)$
بنامید. داریم:
$$
(st).X = s.(t.X) \Rightarrow (st).T(X) = s.T(t.X) = s.(t.T(X)) \Rightarrow (st).V = s.(t.V)
$$
همچنین
$$
1.X = X \Rightarrow 1.T(X) = T(X) \Rightarrow 1.V = V
$$
همچنین
$$
t.(X+Y) = t.X + t.Y \Rightarrow t.T(X+Y) = T(t.X + t.Y) \Rightarrow
$$$$
t.(T(X) + T(Y)) = T(t.X) + T(t.Y) \Rightarrow t.(V + U) = t.V + t.U
$$
و همچنین
$$
(s+t).X = s.X + t.X \Rightarrow (s+t).T(X) = T(s.X + t.X) = T(s.X) + T(t.X)$$$$ \Rightarrow (s+t).V = s.T(X) + t.T(X) = s.V + t.V
$$
پس ضرب اسکالر تعریف شده ویژگی‌های لازم برای فضای برداری را دارد، چون جمع، جمع معمولی $\mathbb{R}$ است، جمع هم ویژگی‌های لازم را دارد و فضای $\mathbb{R}$ روی میدان $\mathbb{C}$ با ضرب اسکالر تعریف شده یک فضای برداری است.

\section{تمرین ۷ سری سوم}
فرض کنید 
$T: V \rightarrow V$
 یک نگاشت خطی باشد که 
$[T]_\alpha^\alpha = A$
باشد.
کافی است نگاشت خطی $S$ را بیابیم که
$T\circ S \circ T = T$
باشد.
\\
فرض کنید 
$\alpha = \{v_1, ..., v_n\}$
باشد حال مجموعه‌ی 
$\{T(v_1), ..., T(v_n)\}$
را در نظر بگیرید. با حذف عضوهای زائد این مجموعه می‌توانیم به زیرمجموعه‌ای از آن برسیم که مستقل خطی‌ست، فرض کنید این مجموعه $m$ عضوی است و بدون خدشه به کلیت فرض کنید $m$ عضو اول مجموعه‌ی بالاست یعنی
$\{T(v_1), ..., T(v_m)\}$
مستقل خطی است. حال مجموعه‌ی بالا را به یک مجموعه‌ی مستقل خطی برای $V$ گسترش می‌دهیم یعنی
$\left\langle
T(v_1), ..., T(v_m), u_{m+1}, ..., u_n\right\rangle= V$
حال اگر برای $i \leq m$،
$u_i = T(v_i)$
بگیریم؛
$\beta = \{u_1, ..., u_n\}$
یک پایه برای $V$ است و 
$\{u_1, ..., u_m\}$
یک پایه برای 
$\text{Im}T$
است.
\\
حال می‌توانیم نگاشت خطی $S$ را با مشخص کردن مقادیر آن برای این پایه بسازیم، به شکل زیر تعریف می‌کنیم:
$$i \leq m: S(u_i) = v_i$$
$$i > m: S(u_i) = 0$$
حال برای $v = a_1 v_1 + ... + a_n v_n$ سعی در محاسبه‌ی $T(v)$ و $T(S(T(v))$ می‌کنیم.
 چون $T(v)$ در $\text{Im} T$ است پس به شکل ترکیبی خطی از
$\{u_1, ..., u_m\}$
است و داریم:
$$T(v) = b_1.u_1 + ... + b_m u_m$$
از طرفی:
$$
T(S(T(v))) = T(S(b_1.u_1 + ... + b_m u_m)) = T(b_1 S(u_1) + ... + b_m S(u_m)) $$$$= T(b_1 v_1 + ... + b_m v_m) = b_1 T(v_1) + ... + b_m T(v_m) = b_1 u_1 + ... + b_m u_m = T(v)
$$
است، یعنی 
$T \circ S \circ T = T$.
\\
حال 
$B = [S]_\alpha^\alpha$
در نظر بگیرید، داریم:
$$
ABA = [T]_\alpha^\alpha [S]_\alpha^\alpha [T]_\alpha^\alpha =
[T \circ S \circ T]_\alpha^\alpha  = [T]_\alpha^\alpha = A
$$
و حکم ثابت شد.
\section{تمرین 8 سری سوم}
فرض کنید $V$ فضایی $n$ بعدی، $U$ فضایی $m$ بعدی و $W$ فضایی $p$ بعدی روی میدان $\mathbb{R}$ باشد. فرض کنید 
$T: V \rightarrow U$
نگاشتی خطی باشد که 
$[T]^\alpha_\beta = A$
و
$S: V \rightarrow W$
نگاشتی خطی باشد که 
$[S]^\alpha_\gamma = B$
باشد. چون به ازای هر $X$ اگر $AX = 0$ باشد نتیجه داریم $BX = 0$ پس
$\ker T \subseteq \ker S$
است. 
برای اثبات حکم تنها کافی است نگاشتی مثل 
$X: U \rightarrow W$
بیابیم که 
$S(v) = X(T(v))$
باشد.
\\
فرض کنید 
$\{v_1, ..., v_r\}$
پایه‌ای برای $\ker T$ باشد، گسترش این پایه به $\ker S$ را 
$\{v_1, ..., v_r, ..., v_k\}$
 و گسترش این پایه به $V$ را 
$\{v_1, ..., v_r, ..., v_k, ..., v_n\}$
در نظر بگیرید.
\\
می‌دانیم 
$\{T(v_{r+1}), ..., T(v_n)\}$
پایه‌ای برای $\text{Im} T$ است. این پایه به پایه‌ای برای $U$ گسترش می‌دهیم مثلا پایه‌ی
$\alpha = \{u_1 = T(v_{r+1}), ..., u_{n-r} = T(v_n), u_{n-r+1}, ..., u_m\}$.
حال نگاشت خطی $X$ را با مقدار دهی مقادیر آن برای پایه‌ی $\alpha$ می‌سازیم:
$$i \leq n-r: X(u_i) = S(v_{r+i})$$
$$i > n-r: X(u_i) = 0$$
حال برای 
$v = a_1 v_1 + ....+ a_n v_n \in V$
مقدار $X(T(v))$ را محاسبه می‌کنیم:
$$
X(T(v)) = X(T(a_1 v_1 + ... + a_n v_n))
= X(a_1 T(v_1) + ... + a_r T(v_r) + ... + a_n T(v_n))
$$$$
= X(a_1 0 + ... + a_r 0 + a_{r+1} T(v_{r+1}) + ... a_n T(v_n))
= X(a_{r+1} u_1 + ... + a_n u_{n-r})
$$$$
= a_{r+1} X(u_1) + ... + a_n X(u_{n-r})
= a_{r+1} S(v_{k+1}) + ... + a_n S(v_n)
$$$$
= a_1. 0 + ... + a_r.0 + a_{r+1} S(v_{k+1}) + ... + a_n S(v_n)
$$$$
= a_1. S(v_1) + ... + a_r.S(v_r) + a_{r+1} S(v_{k+1}) + ... + a_n S(v_n)
$$$$
= S(a_1 v_1 + ... + a_n v_n) = S(v)
$$
پس $X\circ T = S$ پس اگر 
$[X]^\beta_\gamma = C$
باشد داریم: 
$$CA = [X]^\beta_\gamma [T]^\alpha_\beta = [X\circ T]^\alpha_\gamma = [S]^\alpha_\gamma = B$$
و حکم ثابت شد.
\section{تمرین 25 سری سوم}
با برهان خلف ثابت می‌کنیم هیچ $A$ای وجود ندارد که 
$\{I, A, A^2, ...\}$
مولد تمام ماتریس‌های $n\times n$ شود.
فرض کنید $A$ ای پیدا کردیم که هر ماتریس دلخواه مثل $B$ و $C$ را می‌توان به صورت ترکیب خطی‌ای از $A^i$ها نوشت مثلا
$$B = b_1 A^{0} + b_2 A^{1} + ... +  b_m A^{m}$$
و
$$C = c_1 A^{0} + c_2 A^{1} + ... +  c_m A^{m}$$

حال ادعا می‌کنیم $BC = CB$ است، برای اینکار به محاسبه هر کدام می‌پردازیم:
\\
چون $A^x A^y = A^y A^x = A^{x+y}$ پس:
$$
BC = (b_1 A^{0} + b_2 A^{1} + ... +  b_m A^{m})(c_1 A^{0} + c_2 A^{1} + ... +  c_m A^{m})
$$$$
= \sum_{i=0}^{2m} \sum_{j=\max(0, i-m)}^{\min(m, i)}(a_j b_{i-j}) A^{i}
$$
و از طرفی
$$
CB = (c_1 A^{0} + c_2 A^{1} + ... +  c_m A^{m})(b_1 A^{0} + b_2 A^{1} + ... +  b_m A^{m})
$$$$
= \sum_{i=0}^{2m} \sum_{j=\max(0, i-m)}^{\min(m, i)}(a_j b_{i-j}) A^{i}
$$
پس $CB = BC$ پس اگر این مجموعه مولد تمام ماتریس‌ها شود ضرب خاصیت جابه‌جایی خواهد داشت، درحالی که می‌دانیم ضرب ماتریس‌های $n\times n$ خاصیت جابه‌جایی ندارد که این تناقض است. پس فرض خلف باطل و هیچ $A$ ای وجود ندارد که $\{I, A, A^2, ...\}$
مولد تمام ماتریس‌های $n\times n$ شود.
\section{تمرین 29 سری سوم}
اگر بعد $V$ متناهی نباشد این مسئله مثال نقض دارد.
\\
به طور مثال اگر $V$ فضای برداری چندجمله‌ای‌ها باشد، $T$ را عملگر خطی در نظر بگیرید که یک چند جمله‌ای را در $x$ ضرب می‌کند یعنی
$$T(a_0 + a_1 x + ... + a_n x^n) = a_0 x + a_1 x^1 + ... + a_n x^{n+1}$$
همچنین $U$ را عملگری در نظر بگیرید که جمله‌ی ثابت را حذف کرده و همه را تقسیم بر $x$ می‌کند یعنی
$$U(a_0 + a_1 x + ... + a_n x^n) = a_1 + a_2 x + ... + a_n x^{n-1}$$
نگاشت $T$ پوشا و نگاشت $U$ یک‌به‌یک نیست پس هیچ کدام وارون پذیر نیستند. همچنین
$$UT(a_0 + a_1 x + ... + a_n x^n) = U(a_0 x + a_1 x^2 + ... + a_n x^{n+1}) = a_0 + a_1 x + ... + a_n x^n$$
است پس این نگاشت همانی بوده و وارون پذیر است و مثال نقضی برای این مسئله است.

پس فرض می‌کنیم $V$ با بعد متناهی است و همچنین منظور از عملگر عملگرخطی است.
\\
دو حالت داریم، اگر $T$ وارون پذیر باشد، آنگاه $T^{-1}$ نیز یک نگاشت خطی بوده و طبق مسئله‌ی 28 داریم $T^{-1}TU$ نیز خطی بوده و در نتیجه
$$T^{-1}TU = (T^{-1}T)U = I_vU = U$$
خطی است و این حالت حل شد.
\\
حال فرض کنید $T$ وارون پذیر نیست، پس $T$ پوشا نیست و چون $T$ پوشا نیست، پس $TU$ نیز نمی‌تواند پوشا باشد که این با وارون پذیر بودن $TU$ در تناقض است پس این حالت رخ نمی‌دهد.
\\
پس هر دو حالت حل شد و مسئله حل است.
\section{تمرین ۳۰ سری سوم}
خیر لازم نیست وارون پذیر باشند، به طور مثال $V = Z$ را 
$\mathbb{R}^n$
و $W$ را
$\mathbb{R}^{n+1}$
در نظر بگیرید.
$\forall v \in V: T(v) = v$
و
$\forall v \in W: U(v) = v$
در نظر بگیرید، چون $V$ و $U$ دارای بعد مبدا و مقصد متفاوت اند پس وارون‌پذیر نیستند.
\\
از طرفی 
$UT(v) = U(v) = v$
است که یک عملگر همانی روی $V=Z$ است پس وارون پذیر است.
چون $TU$ وارون پذیر و $T$ و $U$ وارون پذیر نیستند پس لازم نیست وارون‌پذیر باشند.

\section{تمرین 31 سری سوم}
فرض کنید $X$ نگاشتی خطی است که روی هر $T$ ای جابه جا می‌شود. یعنی $XT = TX$.
\\
اگر $X$ مضربی از نگاشت همانی نباشد، پایه‌ای مثل 
$\{v_1, ..., v_n\}$
برای $V$ وجود دارد که 
$$X(v_1) = r.v_1 + k$$
است که 
$$k \neq 0, k \in \langle v_2, ..., v_n\rangle$$
باشد. (اثبات این در پایان راه حل)
\\
در این صورت
$$V = \langle v_1 \rangle \oplus \langle v_2, ..., v_n \rangle$$
است. $T$ را نگاشت تصویر در نظر بگیرید که
$$u = u_1 + u_2; u_1 \in \langle v_1 \rangle, u_2 \in \langle v_2, ..., v_n \rangle \rightarrow T(u) = u_1$$
در این صورت:
$$TX = XT \rightarrow XT(v_1) = TX(v_1) = T(X(v_1)) = T(r.v_1 + k) = r.v_1$$
از طرفی
$$XT(v_1) = X(v_1) = r.v_1 + k$$
پس
$r.v_1 = r.v_1 + k \rightarrow k = 0$
که این با فرض $k \neq 0$ در تناقض است.
\\
\\
حال به اثبات ادعایی که در ابتدای مسئله شد می‌پردازیم.
\\
$\{v_1, ..., v_n\}$
را یک پایه‌ی دلخواه در نظر می‌گیریم اگر عضوی مثل $v_i$ وجود داشت که 
$$X(v_i) = r.v_i + k$$
باشد و 
$$k \neq 0, k \in \langle v_1, ..., v_{i-1}, v_{i}, ..., v_n\rangle$$
باشد که $v_i$ را به اول پایه می‌آوریم و حکم برقرار می‌شود، در غیر اینصورت هر عضو مثل $v_i$ داشته باشیم، 
$X(v_i) = r_i v_i$
است. چون این نگاشت مضربی از نگاشت همانی نیست $i$ ای وجود دارد که
$r_i \neq r_1$
بدون خدشه به کلیت مسئله فرض کنید $r_2 \neq r_1$.
\\
حال طبق تمرین 7 سری نخست، مجموعه‌ی 
$\{v_1 + v_2, v_2 + v_2, ..., v_n + v_2\}$
نیز یک پایه برای $V$ است و اما
$$X(v_1 + v_2) = X(v_1) + X(v_2) = r_1.v_1 + r_2.v_2 = r_1(v_1 + v_2) + (r_2 - r_1). v_2$$
است، که 
$(r_2 - r_1). v_2 \notin \langle v_1 + v_2 \rangle$
پس 
$$(r_2 - r_1). v_2 \neq 0, (r_2 - r_1). v_2 \in \langle v_2 + v_2, ..., v_n + v_2 \rangle$$
 و حکم برقرار است.
 
\section{تمرین 36 سری سوم}
\subsection{لم ۱}
\textbf{اگر در میدان $F$،
$x$
ای داشتیم که
$x^2 = 1$
بود، آنگاه $x=1$ یا $x=-1$.}

برای اثبات تنها کافی‌است یک را به طرف چپ ببریم، داریم:
$$
x^2 = 1 \implies x^2-1 = 0 \implies (x-1)(x+1) = 0 \implies x-1 = 0 \text{ or } x+1 = 0$$$$ \implies x = 1 \text{ or } x = -1
$$
%\subsection{لم 2}
%\textbf{
%اگر $V$ یک فضای برداری با بعد متناهی و 
%$T: V \rightarrow V$
%یک نگاشت باشد که 
%$T^2 = T$
%است، پایه‌ای مانند
%$\{v_1, ..., v_k, v_{k+1}, T(v_{k+1}), ..., v_n, %T(v_n)\}$
%وجود دارد.
%}
%\\
%برای اثبات الگوریتم زیر را در نظر بگیرید، فرض کنید مجموعه‌ی پایه‌هایی باشد که به ازای هر عضو $T$ آن عضو نیز در فضای تولید شده توسط پایه باشد.الآن $S$ باشد، در ابتدا %$S$ تهی است.
%%\\
%یک عضو از $V$ که در زیرفضای
%$\langle S \rangle$
%نباشد در نظر بگیرید این عضو را $u$ در نظر بگیرید. $u$ را به $S$ اضافه کرده، حال $T(u)$ را در نظر می‌گیریم، اگر %$T(u)$ در $\langle S \rangle$ نبود آن‌را نیز اضافه %می‌کنیم، چون بعد $V$ محدود است، این فرایند پایان می‌یابد.

\subsection{لم 2}
\textbf{
اگر $V$ یک فضای برداری با بعد متناهی و $T: V \rightarrow V$ یک نگاشت خطی باشد که 
$T^2 = T$
است، پایه‌ای مانند 
$\{v_1, ..., v_n\}$
وجود دارد که
$T(v_i) = v_i$
یا
$T(v_i) = -v_i$
}

اثبات این لم را بلد نیستم ولی به‌نظر با کمک لم ۱ درست است.
\\
\\
حال طبق لم‌ بالا برای هریک از $V$ پایه‌ای داریم که اثر $T$ روی آن همانی یا منفی همانی است. فرض کنید برای مجموعه‌ی $S_1$ همانی و برای مجموعه‌ی $S_2$ منفی همانی باشد. بدین صورت مجموعه‌های $T_1$ و $T_2$ برای $U$ بدست می‌آید.
\\
اگر هر کدام از $S_1$ و $S_2$ با $T_1$ و $T_2$ اشتراک داشته باشند، در این صورت یک فضای یک بعدی تشکیل شده از بردار اشتراک آن‌ها وجود دارد که هر دو نگاشت در آن همانی باشند. پس فرض کنیم چنین اتفاقی نمی‌افتد. پس
$S_1 \cap T_1 = S_1 \cap T_2 = S_2 \cap T_1 = S_2 \cap T_2 = \emptyset$
همچنین چون $S_1$ و $S_2$ جدا از هم‌اند و $T_1$ و $T_2$ جدا از هم‌اند داریم:
$$
|S_1| + |S_2| = |T_1| + |T_2| = \dim(V) = 2n+1
$$
حال طبق اصل لانه‌کبوتری تعداد اعضای یکی از $S_1$ و $S_2$ از $n+1$ بیشتریا مساوی است، بدون خدشه به کلیت مسئله فرض کنید $S_1$ این ویژگی‌را دارد و به طور مشابه فرض کنید 
$|T_1| \geq n+1$
.
حال چون $S_1$ و $S_2$ جدا از هم بودند داریم:
$$|T_1 \cup S_1| \leq 2n+1$$
$$|T_1 \cup S_1| = |T_1| + |S_1| \geq 2n+2$$
که این تناقض است.
\section{تمرین 37 سری سوم}
برای حل فرض کنید ابعاد ماتریس $A_{ij}$ برابر با 
$a_i \times b_j$
است و ابعاد ماتریس 
$B_{ij}$
برابر با
$b_i \times c_j$
است. در این صورت ماتریس بزرگ شامل همه‌ی 
$A_{ij}$
ها را $X$ می‌نامیم و ابعاد آن برابر با
$\sum_{i=1}^m a_i \times \sum_{i=1}^n b_i$
است، همچنین ماتریس بزرگ شامل همه‌ی $B_{ij}$‌ها را $Y$ می‌نامیم که ابعاد آن برابر با 
$\sum_{i=1}^n b_i \times \sum_{i=1}^p c_i$
است. اگر $Z=XY$ باشد و
$(Z)_{xy} = (C_{ij})_{vu}$
باشد تنها کافی‌ست ثابت کنیم
$(\sum_{k=1}^m A_{ik} B_{kj})_{vu} = (Z)_{xy}$
است.
$$Z_{xy} = \sum_{i=1}^{\sum_{i=1}^m b_i} X_{xi} Y_{iy}$$
از طرفی
$$(\sum_{k=1}^m A_{ik} B_{kj})_{vu}
= \sum_{k=1}^m (A_{ik} B_{kj})_{vu}
= \sum_{k=1}^m \sum_{l=1}^{b_k}(A_{ik})_{vl}(B_{kj})_{lu}
= \sum_{i=1}^{\sum_{i=1}^m b_i} X_{xi} Y_{iy}
$$
(جمله‌ی یکی مانده به آخر در واقع همان حاصل سطر $x$ام در ماتریس نخست که سطر $v$ ام از بلوک‌های در سطربلوکی $i$ ام در ماتریس بلوک،بلوک شده است در ستون $y$ ام در ماتریس دومم که ستون $u$ام از بلوک‌های در ستون بلوکی $j$ام در ماتریس بلوک‌بلوک شده‌ی دوم است می‌باشد)

پس 
$(\sum_{k=1}^m A_{ik} B_{kj})_{vu} = (Z)_{xy}$
ثابت شد و حکم ثابت است.
\\
\\
همچنین اگر منظور سوال به این سبک بلوک بندی نباشد، مثال نقضی برای سوال داریم به این صورت که اگر دو ماتریس که هر کدام ۲ در ۲ باشند داشته باشیم، ماتریس اول به ترتیب دارای ماتریس‌های به ابعاد $1\times5$، $1\times 1$ در بالا و $1\times 5$ و $1\times 1$ در پایین و ماتریس دوم به ابعاد $5\times 1$ و $5\times 5$ در بالا و $1\times 5$ و $1\times 1$ در پایین باشد، ابعاد $A_{ik}$ و $B_{kj}$ مناسب ضرب هستند ولی در ضرب یک درایه به جمع چند ماتریس  می‌رسیم که ابعاد یکسانی ندارند و مثال نقضی برای سوال است.
\section{تمرین 38 سری سوم}
چون هر زیرفضا شامل 0 است و 0 یک عملگر وارون‌پذیر نیست، پس مجموعه‌ی همه‌ی عملگرهای وارون پذیر شامل 0 نیست و زیرفضا نیست.

همچنین اگر زیرفضایی داشته باشیم که همه‌ی اعضای آن وارون پذیر باشد، پس ۰ نیز باید وارون پذیر باشد که این تناقض است.پس گزاره‌ی دوم به انتفای مقدم درست است. گزاره‌ی سوم نیز به همین شکل درست می‌شود!
\\
\\
حال اگر صفر را بیخیال شویم:
\\
\subsection{قسمت اول}
$I_v$
یک عملگر خطی وارون پذیر است، همچنین $A$ را ماتریسی در نظر بگیرید که همه عناصر قطر اصلی به غیر از سطر آخر ۱ باشد، عنصر سطر آخر قطر اصلی $-1$ و بقیه‌ی عناصر ۰ باشد. $A$ نیز یک ماتریس وارون پذیر است و نمایش دهنده‌ی نگاشت $T$ است که وارون پذیر است. حال $I_v+T$ را در نظر بگیرید. این نگاشت عضو آخر پایه‌ی نمایش ماتریس $A$ را به صفر می‌برد پس وارون پذیر نیست. پس مجموعه‌ی همه‌ی عملگرهای خطی وارون پذیر زیرفضا نیست.
\section{تمرین 40 سری سوم}
\subsection{لم}
\textbf{
اگر $A$ یک ماتریس پوچ‌توان با مرتبه‌ی $k$ 
که
$k \geq 1$
باشد و $B^2 = A$، آنگاه $B$ یک ماتریس پوچ‌توان با مرتبه‌ی بزرگ‌تر از 
$k$
است.
}
طبق فرض داریم 
$A^k = 0, A^{k-1} \neq 0$
پس 
$B^{2k} = 0, B^{2k-2} \neq 0$
پس $B$ یک ماتریس پوچ توان با مرتبه‌ی بزرگتریا مساوی 
$2^k-1$
است، از طرفی چون $k > 1$ است، 
$2k-1 > k$
است و لم ثابت شد.

حال به اثبات مسئله می‌پردازیم:

فرض خلف می‌کنیم، فرض کنید برای هر ماتریس $X$، ماتریسی مثل $Y$ وجود دارد که 
$Y^2 = X$.

همچنین در یک فضای $n$ بعدی، ماتریس پوچ توانی با مرتبه‌ی بیشتر از $n$ نداریم. (زیرا در این صورت این صورت، نگاشت این ماتریس دارای یک زنجیره مستقل خطی به طول بیشتر از $n$ از پایه‌هاست که با بعد $n$ بودن در تناقض است.)

چون $n>1$ است، ماتریسی داریم که پوچ توان با مرتبه‌ی حداقل یک باشد. مثلا ماتریس 
$A_1$
 که 
$(A_1)_{12} = 1$
 و برای بقیه‌ی درایه‌ها ۰ باشد.

حال چون طبق فرض خلف برای هر ماتریسی ماتریس دیگری وجود دارد که توان دو آن برابر با ماتریس اول شود، ماتریسی مثل $A_2$ وجود دارد که 
$(A_2)^2 = A_1$
و همچنین طبق لم $A_2$ نیز پوچ توان و مرتبه‌ی آن حداقل 2 است. حال فرض کنید ماتریس $A_i$ با مرتبه‌ی حداقل $i$ یافتیم، طبق فرض خلف ماتریس 
$A_{i+1}$
وجود دارد که
$(A_{i+1})^2 = A_i$
است و طبق لم $A_{i+1}$ پوچ توان با مرتبه‌ی بیشتر از مرتبه‌ی $A_i$ یعنی با مرتبه‌ی حداقل $i+1$ است. پس طبق استقرا ماتریسی مثل $A_{n+1}$ یافت می‌شود که پوچ‌توان و با مرتبه‌ی حداقل $n+1$ است، که این با $n$ بعدی بودن فضای برداری در تناقض است، پس فرض خلف باطل و حکم ثابت شد.
\section{تمرین 41 سری سوم}
تنها برگشت قضیه‌ی دوشرطی سوال را اثبات می‌کنم.
\\
فرض کنید $I, T, ..., T^k$ وابسته خطی باشند، در این صورت اسکالرهای
$a_0, ..., a_k$
وجود دارند که حداقل یکی از آن‌ها صفر نیست و 
$a_0.I + a_1.T + ... + a_k T^k = 0$
پس به ازای هر $v$ داریم:
$$a_0.I(v) + a_1.T(v) + ... + a_k T^k(v) = 0 \rightarrow a_0.v + a_1.T(v) + ... + a_k T^k(v) = 0$$
پس
$v, T(v), ..., T^k(v)$
وابسته‌ی خطی اند و حکم ثابت شد.
\end{document}