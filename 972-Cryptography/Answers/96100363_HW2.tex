\documentclass[12pt]{article}
\usepackage{HomeWorkTemplate}
\usepackage{circuitikz}
\usepackage{tikz}
\usepackage{float}
\usepackage{mathtools}
\usepackage{xepersian}
\usetikzlibrary{arrows,automata}
\usetikzlibrary{circuits.logic.US}
\settextfont{XB Niloofar}
\newcounter{problemcounter}
\newcounter{subproblemcounter}
\linespread{1.2}
\setcounter{problemcounter}{1}
\setcounter{subproblemcounter}{1}
\newcommand{\grade}[1]{\textbf{(#1 نمره)}}
\newcommand{\problem}[1]
{
\section*{
مسأله‌ی
\arabic{problemcounter} 
\stepcounter{problemcounter}
\setcounter{subproblemcounter}{1}
#1
}
}
\newcommand{\subproblem}{
\subsection*{\alph{subproblemcounter})}\stepcounter{subproblemcounter}
}
\newcommand{\n}{

\null

}
\begin{document}

\handout
{مقدمه‌ای بر رمزنگاری}
{}
{دانشجو: علیرضا توفیقی محمدی}
{سری 2}
{شماره‌ی دانشجویی: 96100363}

\problem{} % 1
هیچ کدام خوب نیست، 
\\
اولا هر الگوریتم رمزنگاری که بخواهد امنیت کامل را داشته باشد، لازم است اندازه‌ی فضای \lr{cipther-text} از متن اصلی بیشتر باشد، همچنین در الگوریتم‌های معمول رمزنگاری سعی بر این است که متن رمز‌شده تاجای ممکن به یک متن رندم شبیه باشد و این دو توزیع \lr{PPT} یکسان داشته‌باشند، پس رمزنگاری و سپس کامپرس کردن در مورد اول می‌تواند باعث کاهش امنیت و در مورد دوم اصلا کارا نیست.
\\
همچنین ابتدا فشرده‌سازی و سپس رمزنگاری نیز می‌تواند باعث لو رفتن برخی از اطلاعات رمز اصلی شود.

می‌دانیم که طول پیام را به سادگی نمی‌توان مخفی نگه داشت، حال چون الگوریتم‌های فشرده‌سازی براساس پیام اصلی طول‌های مختلفی را می‌سازند (رشته‌های کاملا رندم را قادر به فشرده‌سازی نیستند و رشته‌های واقعی را فشرده‌تر می‌کنند.) پس یک متن کاملا رندم و یک متن غیررندم قابل تشخیص هستند و امنیت آن زیر سوال می‌رود.


\problem{} % 2
برای حل مسئله می‌دانیم اگر $m_1, m_2$ دو متن اصلی باشند که با کلید $k$ از طریق رمز \lr{OTP} رمز شده‌باشند و رمز‌شده‌ي آن‌ها به ترتیب $c_1, c_2$ باشد، آنگاه:
$$
m_1 \oplus m_2 = c_1 \oplus c_2
$$
از همین ویژگی برای حدس $c_2$ استفاده می‌کنیم.
داریم:
$$
m_1 = 61 74 74 61 63 6b 20 61 74 20 64 61 77 6e, m_2 = 61 74 74 61 63 6b 20 61 74 20 64 75 73 6b
$$
$$
\implies m_1 \oplus m_2 = 140405
$$
% 61 74 74 61 63 6b 20 61 74 206461776e
% 61 74 74 61 63 6b 20 61 74 206475736b
حال از طرفی داریم:
$$
m_1 \oplus m_2 = c_1 \oplus m_2 \implies c_2 = m_1 \oplus m_2 \oplus c_1 $$
$$
\implies c_2 = 140405 \oplus 09e1c5f70a65ac519458e7e53f36 = 09e1c5f70a65ac519458e7f13b33
$$

\problem{} % 3
\subproblem{}
برای اثبات کافی است ثابت کنید برای هر $m \in M, c \in C$ داریم:
$$
Pr\{C = c | M = m\} = Pr\{M+K = c | M = m\} = Pr\{K = c-m\} = \frac{1}{|K|}
$$
مشاهده می‌شود که متن رمزی مستقل از متن اصلی دارای توزیع یک‌نواخت است.
\subproblem{}
طبق قضیه‌ی شانون برای اینکه یک سیستم رمز امنیت کامل داشته‌باشد باید $|M| \leq |K|$.
\\
پس تنها کلمات یک حرفی با کلید یک‌حرفی قادر به داشتن امنیت کامل هستند.
\subproblem{}
اولا طبق قضیه‌ی شانون باید اندازه‌ی فضای کلید به اندازه‌ی فضای متن اصلی باشد، پس طول کلید باید حداقل $n$ باشد. همچنین کلید به طول $n$ امنیت کامل را رقم می‌زند زیرا برای هر $m = m_1m_2... m_n \in M$ و $c = c_1 c_2 ... c_n \in C$ داریم:
$$
Pr(C = c | M = m) = Pr(c = (m_1 + K_1)(m_2+K_2)...(m_n+K_n) | M = m_1 m_2 ... m_n)
$$$$
= Pr(K = (c_1 - K_1)(c_2 - K_2)...(c_n - K_n) ) = \frac{1}{|K|}
$$
که متن رمزی مستقل از متن اصلی دارای توزیع یک‌نواخت است، پس دارای امنیت کامل است.

(دقت کنید منظور از جمع و منهاها در این پاسخ سوال جمع و تفریق در پیمانه‌ی ۲۶ است.)\
\problem{} %4
اولا اگر سیستمی امنیت کامل داشته باشد، یعنی به ازای هر $c \in C$ و $m_1, m_2 \in M$ داریم:
$$
Pr[C = c | M = m_1] = Pr[C = c | M = m_2] 
$$
پس هیچ حمله‌کننده‌ای نمی‌تواند با احتمال بهتر از $\frac{1}{2}$ تشخیص دهد $c$ رمزشده‌ی $m_1$ است یا $m_2$.

همچنین اگر سیستمی امنیت کامل در آزمایش تشخیص داشته‌باشد، یعنی هر حمله‌کننده‌ای در نظر بگیریم، احتمال درست گفتنش $\frac{1}{2}$ است.

حال به ازای هر $c \in C$ و $m_1, m_2 \in M$، حمله‌کننده‌ی 
$A_{m_1, m_2, c}$
را به این گونه تعریف می‌کنیم که دو متن $m_1, m_2$ را می‌دهد، حال اگر چالشگر رمز $c$ را داد، حمله‌کننده یک و در غیر اینصورت یک عدد تصادفی از ۰ و ۱ را برمی‌گرداند.

می‌دانیم طبق فرض احتمال درست گفتن این حمله‌کننده نیز $\frac{1}{2}$ است. از طرفی فرض کنید احتمال اینکه چالشگر رمز $c$ را در صورت گرفتن $m_1, m_2$ بدهد $p$ باشد، در این صورت برای احتمال درست گفتن حمله کننده داریم:
$$
p * \frac{Pr[C = c | M = m_1]}{Pr[C = c | M = m_1] + Pr[C = c | M = m_2]} + (1-p)\times 0.5 = \frac{1}{2}
$$$$
\implies \frac{Pr[C = c | M = m_1]}{Pr[C = c | M = m_1] + Pr[C = c | M = m_2]} = \frac{1}{2}
$$$$
\implies Pr[C = c | M = m_1] = Pr[C = c | M = m_2] 
$$
اگر $p \neq 0$ می‌توان نتیجه‌ی بالا را گرفت، همچنین داریم:
$$
p = \frac{Pr[C = c | M = m_1] + Pr[C = c | M = m_2]}{2}
$$
پس اگر $p = 0$ شود، آنگاه هر دو احتمال برابر با صفر می‌شود و این حالت نیز حل می‌شود.

پس طبق استدلال بالا به ازای هر $m_1, m_2, c$ ثابت شد 
$Pr[C = c | M = m_1] = Pr[C = c | M = m_2]$
و در نتیجه امنیت کامل داریم.
\newpage
\problem{} %5
\subproblem{}
می‌توان تعریف را به شکل زیر ارائه داد:
$$
\forall A \text{\lr{(which is PPT)}}:
$$$$
 Pr[(pk, sk) \leftarrow Gen(1^n); m \leftarrow M; c \leftarrow Enc_{pk}(m); A(pk, c) = m] \leq \frac{1}{|M_n|} (1 + \epsilon(n))
$$
که $\epsilon$ یک تابع ناچیز است.

\subproblem{}
برای اینکار فرض می‌کنیم حمله‌کننده $A$ وجود دارد که $pk, c$ را گرفته و با احتمال بیشتر از 
$\frac{1}{|M_n|} (1 + \epsilon(n))$
 به ازای هر تابع ناچیزی $\epsilon$ای درست حدس می‌زند.

حال یک حمله کننده مثل $A'$ برای مسئله‌ی \lr{DDH} می‌سازیم، برای اینکار اگر $A'$ ورودی $G, q, g, u, v, w$ را گرفت، مقادیر زیر را بسازد:
$$
pk = (u, G, q, g)
$$$$
m \leftarrow M
$$$$
c =  (v, m . w)
$$$$
m' = A(pk, c)
$$
حال اگر $m' = m$ بود، یک را بر میگرداند و در غیر اینصورت یک عدد تصادفی از ۰ و ۱ برمی‌گرداند.

حال احتمال درست حدس زدن را حساب می‌کنیم:
$$
Pr > \frac{1}{2} \times (\frac{1}{|M_n|} (1 + \epsilon(n)) + \frac{1}{2}\times (1-\frac{1}{|M_n|} (1 + \epsilon(n)))) 
$$$$
+ \frac{1}{2} \times (\frac{1}{q}\times (\frac{1}{|M_n|} (1 + \epsilon(n)) + \frac{1}{2}\times (1-\frac{1}{|M_n|} (1 + \epsilon(n)))) + \frac{q-1}{q} \times \frac{1}{2})
$$$$
\geq \frac{1}{2}\left(
\frac{1}{2}\times (1+\frac{1}{|M_n|} (1 + \epsilon(n))) + \frac{1}{2}
\right) = \frac{1}{2} +\frac{1}{4|M_n|} (1 + \epsilon(n))
$$
\problem{} %6
درست نیست، اگر یک سیستم دارای امنیت کامل باشد و فضای آن بالای ۳ عضو داشته باشد، برای هر توزیع $M$ و برای $m \in M, c \in C$ باید شرط زیر برقرار باشد:
$$
Pr\{M = m | C = c\} = Pr\{M = m\}
$$
حال سه عضو دلخواه $M$ مثل $m_1, m_2, m_3$ را در نظر بگیرید و فرض کنید توزیع $M$ به این حالت است که 
$$Pr\{M = m_1\} = Pr\{M = m_3\} = \frac{1}{2}, Pr\{M \notin \{m_1, m_3\}\} = 0$$

اما با توزیع داریم:
$$
Pr\{M = m_1 | C = c\} = Pr\{M = m_1\} = \frac{1}{2}
$$$$
Pr\{M = m_2 | C = c\} = Pr\{M = m_1\} = 0
$$
پس
$$
Pr\{M = m_1 | C = c\} \neq Pr\{M = m_2 | C = c\}
$$
پس حکم برقرار نیست.

\problem{} % 7
اگر $|M| = |C| = 1$ حکم سوال برقرار نیست و سیستم رمزی داریم که امنیت کامل دوپیامه را دارد.

اما اگر $|M| > 1$:
\\
فرض کنید $k \in K$ کلیدی دلخواه است و همچنین $m_1 \in M$ نیز پیامی دلخواه است.
\\
حال مجموعه مقادیر ممکن برای $Enc_k(m_1)$ را $D$ بنامید. ادعا می‌کنیم $C-D \neq \emptyset$. زیرا طبق شرط صحت 
$Pr[Dec_k[Enc_k(m_1)] = m_1] = 1$
و در نتیجه:
$\forall d \in D: Pr[Dec_k[d] = m_1] = 1$

و اگر $C = D$ باشد، آن‌گاه شرط صحبت برای بقیه‌ی اعضای $M$ نقض می‌شود.

حال $c_2 \in C-D$ و $c_1 \in D$ در نظر بگیرید که 
$Pr[C_1 = c_1 \land C_2 = c_2] > 0$.
( این کار قابل انجام است زیرا $c_2$ را می‌توان برای $m \in M, m \neq m_1$ یکی از مقادیر $Enc_k(m)$ در نظر گرفت.)

حال $m_2 = m_1$ در نظر بگیرید؛ ادعا می‌کنیم $m_1, m_2, c_1, c_2$ ساخته‌شده شرایط مسئله را نقض می‌کنند. زیرا از طرفی 
$$Pr[M_1 = m_1 \land M_2 = m_2] \neq 0$$
است و از طرفی چون $c_2 \notin D$ و در نتیجه احتمال $Pr[M_2 = m_2 \land C_2 = c_2] = 0$ است، پس:
$$
Pr[M_1 = m_1 \land M_2 = m_2 | C_1 = c_1 \land C_2 = c_2] = 0
$$

پس حکم سوال برای هر سیستم رمز دلخواهی نقض می‌شود.

\problem{} % 8
برای اثبات وجود چنین سیستم رمزی، کافی است سیستم رمزی مانند \lr{OTP} که امنیت کامل دارد را در نظر بگیریم، اما در کلید آن به جای ساختن کلید $n$ بیتی، یک کلید $n-t$ بیتی بسازیم.  سپس در فرایند رمزنگاری و بازگشایی رمز، ابتدا یک رشته‌ی $t$ بیتی تصادفی ساخته و کلید کاری خود را با کنار هم قرار دادن کلید اصلی و رشته‌ی تصادفی در نظر بگیریم و فرایند \lr{XOR} را با آن انجام دهیم.

در این صورت اندازه‌ي فضای کلید برابر با $\frac{|M|}{2^t}$ خواهد شد، همچنین مشابه استدلال برای \lr{OTP} دارای امنیت کامل است.

همچنین شرط صحت جدید را دارد، زیرا به احتمال $2^{-t}$ رشته‌ی تصادفی در الگوریتم رمزنگاری و رمزگشایی یکسان شده و طبق شرط صحت \lr{OTP} با این شرط به احتمال یک رمزنگاری ورمزگشایی همانی جواب می‌دهد. پس احتمال اینکه رمزنگاری و رمزگشایی با یک کلید همانی جواب دهد حداقل $2^{-t}$ است.

همچنین باند آن $\frac{|M|}{2^t}$ است که اثباتی برای آن ندارم.
\problem{} % 9

\problem{} % 10
\subproblem{}
$$
Dec_{k_0||k_1}(c_0||c_1) = 
\begin{cases}
k_0 \oplus c_0 ,& \text{ if } k_0 \oplus c_0 = k_1 \oplus c_1\\
\bot,              & \text{otherwise}
\end{cases}
$$
\subproblem{}
بله، داریم:
$$
Pr[C=00|M=0]=Pr[C=11|M=0]=Pr[C=00|M=1]=Pr[C=00|M=1] = \frac{1}{3}
$$$$
Pr[C=01|M=0]=Pr[C=10|M=0]=Pr[C=01|M=1]=Pr[C=10|M=1] = \frac{1}{6}
$$
پس:
$$
\forall c \in C, m \in M:
Pr[C = c | m = m] = Pr[C = c]
$$
و طبق لم ۱ از جزوه امنیت کامل داریم.
\problem{} % 11

\problem{} % 12

\problem{} % 13

\problem{} % 14
از فرض‌های کتاب می‌دانیم که فضای کلید متناهی است، همچنین چون طبق قضیه‌ی شانون برای برقرار امنیت کامل باید $|K| \geq |M|$ باشد، پس فضای پیام نمی‌تواند نامتناهی باشد.
\problem{} % 15
اگر حمله‌کننده‌ای وجود داشته باشد که بتواند آزمایش دوم را با احتمال بهتر از نصف به‌اضافه‌ی هر تابع جزئی حل کند، آن‌گاه حمله‌کننده‌ای وجود دارد که $m_0, m_1$ را به شکل قطعی انتخاب می‌کند و بازهم از آزمایش دوم با موفقیت بیرون می‌آید (به این شکل که  یکی از حالت‌های انتخاب تصادفی $m_0, m_1$ از حمله کننده‌ی نخست این ویژگی را دارد.)

این $m_0, m_1$ ویژگی اول را ندارند.

حال اگر $m_0$ و $m_1$ ای وجود داشته باشند که قابل تشخیص محاسباتی باشند، حمله‌کننده‌ای می‌سازیم که همین $m_0, m_1$ را انتخاب می‌کنند و از روش تشخیص محاسباتی $m_0, m_1$ چالش را با \lr{advantage} غیر ناچیز پاسخ می‌دهد.
\problem{} % 16
در تعریف اول، $m_0, m_1$ در واقع دنباله‌هایی از متن‌های اصلی هستند (به ازای هر $n$ یک متن اصلی). پس تعداد کل $m_0, m_1$ها ناشمارا است، همچنین چون هر حمله کننده را می‌توان با یک الگوریتم نشان داد و تعداد الگوریتم‌ها شمارا است، پس $m_0, m_1$ ای وجود دارد که هیچ حمله‌کننده‌ای قادر با ساختن آن‌ها نیست. با این حساب کافی‌است سیستم رمزی بسازیم که برای این $m_0, m_1$ قابل تشخیص و برای بقیه غیرقابل تشخیص باشد.
\end{document}
