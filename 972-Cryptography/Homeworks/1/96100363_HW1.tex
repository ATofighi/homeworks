\documentclass[12pt]{article}
\usepackage{HomeWorkTemplate}
\usepackage{circuitikz}
\usepackage{tikz}
\usepackage{float}
\usepackage{mathtools}
\usepackage{xepersian}
\usetikzlibrary{arrows,automata}
\usetikzlibrary{circuits.logic.US}
\settextfont{XB Niloofar}
\newcounter{problemcounter}
\newcounter{subproblemcounter}
\linespread{1.2}
\setcounter{problemcounter}{1}
\setcounter{subproblemcounter}{1}
\newcommand{\grade}[1]{\textbf{(#1 نمره)}}
\newcommand{\problem}[1]
{
\section*{
مسأله‌ی
\arabic{problemcounter} 
\stepcounter{problemcounter}
\setcounter{subproblemcounter}{1}
#1
}
}
\newcommand{\subproblem}{
\subsection*{\alph{subproblemcounter})}\stepcounter{subproblemcounter}
}
\newcommand{\n}{

\null

}
\begin{document}

\handout
{مقدمه‌ای بر رمزنگاری}
{}
{دانشجو: علیرضا توفیقی محمدی}
{سری 1}
{شماره‌ی دانشجویی: 96100363}

\problem{} % 1
مسئله را به این شکل حل می‌کنیم که رشته‌ها را دوتا دوتا \lr{XOR} کرده و بایت‌هایی که بیت ۳۲ را دارند یکی از آن‌ها کاراکتر حروف الفبا نیست، بر این اساس جایگاه اسپیس‌ها را حدس زده، سپس از روی جایگاه اسپیس‌ها یک کلید می‌سازیم و مقادیر رمزگشایی را با آن‌ها چاپ می‌کنیم. سپس کلید را جوری تغییر می‌دهیم تا رمزگشایی‌ها منطقی شوند.
\\
کد در \lr{1.py} قرار دارد.
\problem{} % 2
با توجه به اینکه باید طول بلوک مقسوم علیه‌ای از رمز باشد، حدس می‌زنیم طول بلوک ۷ باشد، سپس با کمک تعداد دوحرفی‌های پرتکرار و اینکه حروف \lr{x} باید آخر رشته باشند، رمز را رمزگشایی می‌کنیم. کد در \lr{2.py} قرار دارد.

\problem{} % 3
\subproblem{} % 3.1
سیستم رمز جایگزینی نیست چراکه واریانسِ فرکانس حروف با واریانس فراکنس حروف انگلیسی متفاوت است.
\subproblem{}
حدس می‌زنیم سیستم رمز ویژنر باشد، با محاسبه‌ی واریانس فراکناس حروف حرف‌های $i$ام بلوک‌ها حدس می‌زنیم که طول بلوک $5$ باشد، حال با کمک حمله به سیستم سزار رمزگشایی می‌کنیم.
\\
کد در \lr{3.py} قرار دارد.
\problem{} % 4
\end{document}
